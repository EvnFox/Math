\documentclass[11pt]{homework}
\usepackage{commath}
\newcommand{\R}{\mathbb{R}}
\newcommand{\N}{\mathbb{N}}
\newcommand{\Q}{\mathbb{Q}}
\newcommand{\Z}{\mathbb{Z}}
\newcommand{\ran}{\operatorname{ran}}
\newcommand{\dom}{\operatorname{dom}}
\newcommand{\eps}{\varepsilon}
\newcommand{\ssd}{\bigtriangleup}
\newcommand{\pow}{\mathcal{P}}

% TODO: replace these with your information
% TODO: replace these with your information
\newcommand{\hwname}{Evan Fox}
\newcommand{\hwemail}{efox20@uri.edu}
\newcommand{\hwtype}{Homework}
\newcommand{\hwnum}{1}
\newcommand{\hwclass}{MTH 316}
\newcommand{\hwlecture}{}
\newcommand{\hwsection}{}

\begin{document}
\maketitle

\question
Prove that the function $\phi : G \to G$ defined by $\phi (g) = g^{-1}$ is an automorphism of 
$G$ iff $G$ is Abelian. 

\begin{proof}
    First Assume that $G$ is Abelian, then define the function $\phi(g) = g^{-1}$ for all $g \in G$. Then it is clear that $\phi$ is 
    a injection since if $\phi(g_1) = \phi(g_2)$ then $g_1^{-1} = g_2^{-1}$; thus $g_1 = g_2$. To see surjectivity let $h \in G$ and it 
    is clear $\phi(h^{-1}) = h$. It then follows that $\phi$ is a bijection. We now must show that $\phi$ is an isomorphism. Observe 
    \[ \phi(ab) = (ab)^{-1} = b^{-1}a^{-1} = a^{-1}b^{-1} = \phi(a)\phi(b). \]  
    Where the third equality is justified since $G$ is abelian by assumption. Thus $\phi$ is an automorphism of $G$.  
    Conversly, Assume that $\phi$ as defined above is an automorphism of $G$. Then for all $a, b \in G$ we have $\phi(ab) = \phi(a)\phi(b)$. 
    Then consider 
    \[ ab = \phi((ab)^{-1}) = \phi(b^{-1}a^{-1}) = \phi(b^{-1})\phi(a^{-1}) = ba \]
    Thus $G$ is Abelian and we are done. 
    
\end{proof}

\question 
Let $\phi$ and $\psi$ be isomorphisms from $G$ to $\bar{G}$. Define 

\[H = \{ g \in G | \phi(g) = \psi(g) \}\]

show $H \leq G$. 

\begin{proof}
We use the two step subgroup test. Notice that since all isomorphisms map the identity in $G$ to the identity in $\bar{G}$, we know 
$e \in H$. Now suppose $a, b \in H$. Then 
\[\phi(a) = \psi(a) \textit{ and } \phi(b) = \psi(b)\]

Then we must have 

\[ \phi(ab) = \phi(a)\phi(b) = \psi(a)\psi(b) = \psi(ab)\]

so $ab \in H$. Now

\[ \phi(a^{-1}) = \phi(a)^{-1} = \psi(a)^{-1} = \psi(a^{-1}) \] 

Thus $a^{-1} \in h$ and this completes the proof. 


    
\end{proof}

\end{document}