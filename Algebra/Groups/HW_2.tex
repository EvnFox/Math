\documentclass[11pt,largemargins]{homework}
\usepackage{commath}
\newcommand{\R}{\mathbb{R}}
\newcommand{\N}{\mathbb{N}}
\newcommand{\Q}{\mathbb{Q}}
\newcommand{\Z}{\mathbb{Z}}
\newcommand{\ran}{\operatorname{ran}}
\newcommand{\dom}{\operatorname{dom}}
\newcommand{\eps}{\varepsilon}
\newcommand{\ssd}{\bigtriangleup}
\newcommand{\pow}{\mathcal{P}}

% TODO: replace these with your information
% TODO: replace these with your information
\newcommand{\hwname}{Evan Fox}
\newcommand{\hwemail}{efox20@uri.edu}
\newcommand{\hwtype}{Homework}
\newcommand{\hwnum}{2}
\newcommand{\hwclass}{MTH 316}
\newcommand{\hwlecture}{}
\newcommand{\hwsection}{}

\begin{document}
\maketitle

\question
Let $|G| = \infty$, Prove $G$ has infinitly many subgroups.

    
\begin{proof}
    We consider the case where $G$ contains at least one element of infinite order seperatly. 
    So Assume $|G| = \infty $ and that all elements of $G$ have finite order. Then we let 

    \[S = \{\langle a \rangle | a \in G\} \] 

    If $S$ is infinite we have found an infinite collection of subgroups and are done. So assume $S$ is finite. Then since $S$ contains 
    all cyclic subgroups of $G$ the union $\bigcup_{x \in S} x = G$ must hold since $a\in \langle a \rangle$ and for all $a \in G$ either $\langle a \rangle$ is 
    an element of $S$ or it is equivalent to an element of $S$. But every element of $G$ has finite order and thus all the cyclic subgroups 
    have finite order. But then the finite union of finite sets must be finite, and since the union of $S$ is equal to $G$ this implys that 
    $G$ is finite; a contradiction. Hence $S$ must be an infinite family of subgroups. 


    Now 
    we consider the cases where $G$ contains at least one element of infinite order. So let $h \in G$ and $|h| = \infty$. Now consider $\langle h \rangle$. It is 
    clear the subgroup generated by $h$ is both cyclic and infinite so it will sufice to show that $h$ has infinite subgroups. Since the order of $h$ 
    is infinite we have $h^n \neq h^m$ for all $n \neq m$ since otherwise we would have $h^{n-m} = e$ implying the order of $h$ is finite.
    Now let $n, m \in \N$ with $n < m$ and assume $\langle h^n \rangle = \langle h^m \rangle$. Then $h^n \in \langle h^m \rangle$ so $h^n = (h^m)^t$ for $t \in \N$. 
    But this implys $m < n$ a contradiction. So then we must have $\langle h^n \rangle \neq \langle h^m \rangle$ for $0 < n < m$. Thus we can easily create infinitly many 
    subgroups of $\langle h \rangle$ and it follows that $G$ has infinitly many subgroups. 
    
\end{proof}

\question
Let $G$ be a group such that the only subgroups of $G$ are the trivial subgroup and $G$ itself. Prove $|G|$ is prime 

\begin{proof}
    It is clear by the previous result that $G$ must be finite, if it were infinite then it must have infinitly many subgroups. 
We first prove that $G$ is cyclic and then we use the fundamental theorem of finite cyclic groups. Let $|G| = n$. Note for all non idenity elements 
$g \in G$ we must have $\langle g \rangle = G$, otherwise $g$ would generate a proper subgroup of $G$. 
Thus $G$ is cyclic. We then have by the FTFCG that there exists a uniqiue subgroup of order $k$ for each $k \in \N$ such that $k|n$.
Since the only subgroups of of $G$ have orders $1$ and $n$ it follows that the only divisors of $n$ are one and itself. Thus $n$ is 
prime.  
    
\end{proof}




\end{document}