
\documentclass[11pt,largemargins]{homework}
\usepackage{commath}
\newcommand{\R}{\mathbb{R}}
\newcommand{\N}{\mathbb{N}}
\newcommand{\Q}{\mathbb{Q}}
\newcommand{\Z}{\mathbb{Z}}
\newcommand{\ran}{\operatorname{ran}}
\newcommand{\dom}{\operatorname{dom}}
\newcommand{\lcm}{\operatorname{lcm}}
\newcommand{\eps}{\varepsilon}
\newcommand{\ssd}{\bigtriangleup}
\newcommand{\pow}{\mathcal{P}}

% TODO: replace these with your information
% TODO: replace these with your information
\newcommand{\hwname}{Evan Fox}
\newcommand{\hwemail}{efox20@uri.edu}
\newcommand{\hwtype}{Homework}
\newcommand{\hwnum}{6}
\newcommand{\hwclass}{MTH 316}
\newcommand{\hwlecture}{}
\newcommand{\hwsection}{}

\begin{document}
\maketitle

\question
Let $G$ be a group and define $H = \{ (g,g) | g \in G \}$
\begin{alphaparts}
    \questionpart
    Show $H \leq G \oplus G$

    \begin{proof}
        Note the the idenitity in $G \oplus G$ is $(e,e)$ and since $e \in G$ we have $(e, e) \in H$. Then 
        let $(a, a), (b, b) \in H$. Again since $b \in G$ we must have $b^{-1} \in G$ which implys $(b^{-1}, b^{-1}) \in H$ and this is 
        clearly the inverse of $(b, b) \in H$. We note $(a,a) (b, b) = (ab, ab)$
        Then similarly since $a, b \in G$ we have $ab \in G$ so $(ab, ab) \in H$. 
        Hence by the two step subgroup test we have that $H$ is a subgroup of $G \oplus G$. 
    \end{proof}

    \questionpart
    Prove $H \cong G$. 

    \begin{proof}
        Define $\phi :H \to G$ such that $(g,g) \mapsto g$. Injectivity is clear. For $h \in G$ we can see 
        $h$ is mapped to by $(h, h) \in H$. Hence $\phi $ is a bijection. Then 
        \[\phi((a,a)(b,b)) = \phi(ab,ab) = ab = \phi(a,a) \phi(b,b)\]
        and this completes the proof.
    \end{proof}
\end{alphaparts}

\question
For prime $p$ show that $\Z_p \oplus \Z_p$ has $p + 1$ subgroups of order $p$

\begin{proof}
    We start by counting the number of elements of order $p$ in $\Z_p \oplus \Z_p$. Each non-idenity element in $\Z_p$ has 
    order $p$. Then an element of $\Z_p \oplus \Z_p$, say $(a, b)$ has order $p$ only if $\lcm(|a|,|b|) = p$, but since the only possible 
    orders for $a$ and $b$ are 1 and $p$, every case must give us an lcm of $p$ unles both $a,b$ have order 1 which can only occure when 
    they are both the idenitity. That is, every element of $\Z_p \oplus \Z_p$ has order $p$ except for the idenitity. Then since 
    $|\Z_p \oplus \Z_p| = p^2$, there must exist $p^2 - 1$ elements of order $p$. Every subgroup of order $p$ is cyclic with $p - 1$ generators 
    so we have counted each subgroup $p- 1$ times. Then the number of subgroups of order $p$ is given by 

    \[\frac{p^2 - 1}{p - 1} = \frac{(p - 1)(p + 1)}{p - 1} = p + 1\]

    Hence there must be $p + 1 $ distinct subgroups of order $p$. 
\end{proof}

\end{document}