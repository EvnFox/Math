\documentclass[11pt,largemargins]{homework}
\usepackage{commath}
\newcommand{\R}{\mathbb{R}}
\newcommand{\N}{\mathbb{N}}
\newcommand{\Q}{\mathbb{Q}}
\newcommand{\Z}{\mathbb{Z}}
\newcommand{\ran}{\operatorname{ran}}
\newcommand{\dom}{\operatorname{dom}}
\newcommand{\st}{\operatorname{stab}}
\newcommand{\eps}{\varepsilon}
\newcommand{\ssd}{\bigtriangleup}
\newcommand{\pow}{\mathcal{P}}

% TODO: replace these with your information
% TODO: replace these with your information
\newcommand{\hwname}{Evan Fox}
\newcommand{\hwemail}{efox20@uri.edu}
\newcommand{\hwtype}{Homework}
\newcommand{\hwnum}{3}
\newcommand{\hwclass}{MTH 316}
\newcommand{\hwlecture}{}
\newcommand{\hwsection}{}

\begin{document}
\maketitle


\question
Let $G$ be a group of permutations on a set A. For $a \in A$ define $\st(a) = \{ \sigma \in G | \sigma(a) = a \} $. Prove 
$\st(a) \leq G$. 

\begin{proof}
    It is clear that $\epsilon(a) = a$ for all $a \in A$; thus $\epsilon \in \st(a)$. Now we use the two step subgroup test so let 
    $\sigma, \tau \in \st(a)$. Then 
    \[(\sigma \circ \tau)(a) = \sigma(\tau(a)) = \sigma(a) = a \] 
    since both $\sigma$ and $\tau$ fix $a$. Now we need to show that if $\sigma$ fixes $a$ then $\sigma^{-1}$ also fixes $a$. 
    since 

    \[ \sigma(a) = a \] 

    we can take $\sigma^{-1}$ of both sides 

    \[ \sigma^{-1}(\sigma(a)) = \sigma^{-1}(a) \] 
    \[ a = \sigma^{-1}(a) \] 
    and this completes the proof. 
    
\end{proof}
\newpage

\question 
Let $\sigma, \tau$ be permutations. Prove $\sigma \tau$ is even if and only if $\sigma$ and $\tau$ are both 
even or both odd. 

\begin{proof}
    ($\Leftarrow$) Assume that $\sigma$ and  $\tau$ are both even or both odd. We have 
    \[ \sigma = \alpha_1 \alpha_2 . . . \alpha_s \] 
    and 
    \[\tau = \gamma_1 \gamma_2 . . . \gamma_t \] 

    Where $\alpha_i$ and $\gamma_i$ are two cycles and $s, t$ have the same parity. We can write $\sigma \tau $ as

    \[ \sigma \tau = \alpha_1 . . . \alpha_s \gamma_1 . . . \gamma_t \]


    then it is clear that $\sigma \tau$ can be written as a product of $t + s$ two cycles. Since an even number plus an even number is 
    even and an odd number plus an odd number is also even, $\sigma \tau$ can be written as an even number of two cylces. 

    ($\Rightarrow$) To prove the opposite direction we use the contrapositive, so assume without loss of generality that $\sigma$ is even 
    and $\tau$ is odd; we prove that $\sigma \tau$ is odd. Just like before we can decompose $\sigma$ and $\tau$ into two cycles of the form 
    \[ \sigma = \alpha_1 \alpha_2 . . . \alpha_s \] 
    and 
    \[\tau = \gamma_1 \gamma_2 . . . \gamma_t \] 
    where $s$ is even and $t$ is odd. We may then write $\sigma \tau$ 

    \[ \sigma \tau = \alpha_1 . . . \alpha_s \gamma_1 . . . \gamma_t \]

    So $\sigma \tau$ can be written as a product of $s + t$ two cycles where $s$ is even and $t$ is odd. Since an odd number 
    plus an even number is odd the result follows. 
    
\end{proof}


% Your content

\end{document}