\documentclass{article}

\usepackage{import}
\usepackage{pdfpages}
\usepackage{transparent}
\usepackage{xcolor}
\usepackage{amsmath, amsthm, amsfonts, amsthm,amssymb,mathtools, mathrsfs}
\usepackage{graphicx} 
\usepackage{tikz, tikz-cd} 
\usepackage{hyperref} 
\usepackage[margin=.75in]{geometry} 
\usepackage[shortlabels]{enumitem}

\newcommand{\R}{\mathbb{R}}
\newcommand{\Q}{\mathbb{Q}}
\newcommand{\Z}{\mathbb{Z}}
\newcommand{\N}{\mathbb{N}}
       %	\newcommand{\incfig}[2][1]{%
%	    \def\svgwidth{#1\columnwidth}
%	    \import{./figures/}{#2.pdf_tex}
%}
%\pdfsuppresswarningpagegroup=1

\newtheorem{prb}{Problem}

\title{MTH 436 Homework 4}
\author{Evan Fox} 
\date{May 1, 2023}


\begin{document}
\maketitle
Morgan and I worked together on this homework. 
	\begin{prb}
		
	\end{prb} 

\begin{proof}
	Let $ \mathcal{A}$ be a set of closed subsets of $\R$ such that at least one element is bounded, say $C \in \mathcal{A}$. Then it follows 
	from the Hine-Borel theorem that this set is compact. 
	We prove the contrapositive, that is, suppose that for all $F_1, \dots , F_n \in \mathcal{A}$, we have the intersection 
	$\bigcap^n F_i \neq \varnothing $. Then we prove theat $\bigcap_{F \in \mathcal{A}} F \neq \varnothing$. 
	We do this by recalling the finite intersection propertery from 525. We know a space $X$ is compact if and only if for all collections 
	of closed sets having the finite intersection property, the intersection over the whole collection is non empty. 
	First note that $C$ (The compact set in $\mathcal{A}$ above) must have a non-trivial intersection with every other $A \in \mathcal{A}$, by assumption. 	Then consider the collection $\mathcal{B} = \{C \cap A \, | \, A \in \mathcal{A}\}$ considered as subsets of the compact metric space $C$. Then 
	this is a collection of closed sets (in $C$) and it has the finite intersection property, since $\mathcal{A}$ does. 
	Hence, by the theorem stated above $\bigcap_{B \in \mathcal{B}} B \neq \varnothing$. Which is to say, 
	$\bigcap_{A \in \mathcal{A}} \Big(C \cap A \Big) = \bigcap_{A \in \mathcal{A}} A \neq \varnothing$ as desired.  
	\end{proof}

\begin{prb}
		
\end{prb}
\begin{proof}
	\begin{enumerate}[(a)]
		\item $F$ is defined as the complement of a union of open sets and hence is closed 
			\item Let $I$ be an interval contained in $F$ and $x, y \in I$; further suppose that  $x < y$.  
				Then by density there exists $r_m$ with $x < r_m < y$, but then by definition $F$ cannot contain the interval 
				$(r_m - \frac{1}{2^m}, r_m + \frac{1}{2^m})$, hence neither can $I$, hence $I$ cannot be an interval. It follows $x = y$. 
			\item We have 
				\[
| \bigcup^\infty (r_k - \frac{1}{2^k}, r_k + \frac{1}{2^k} ) | = \sum^\infty |(r_k - \frac{1}{2^k}, r_k + \frac{1}{2^k} )|  = \sum^\infty \frac{1}{2^{k-1}} 
= 2 
				\]
				Since $|\R| = \infty$, it follows that $|F| = \infty$. 
	\end{enumerate}
\end{proof}
\begin{prb}
	
\end{prb}

\begin{proof}
	Consider the space $(\R, \{ \varnothing, \R\})$. Then the only continuous functions into $\R$ with the usually sigma algebra of borel sets are 
	the constant functions; If $f$ takes more than one value so that $f(x_1) \neq f(x_2)$, let $B$ be a borel set containing $f(x_1)$ but not $f(x_2)$. Then 
	the preimage of $B$ will be some proper subset of $\R$, hence, not in our sigma algbra. Construct the function, 
	\[
		f(x) =  \begin{cases}
			a & x \in [0, \infty) \\ 
			-a & x \in (-\infty, 0)
	         	\end{cases}
	\]
	Then this will produce the desired result. It is not constant so it is not measurable, but $|f|$ is constant.  

\end{proof}




 

\begin{prb}
	
\end{prb}

\begin{proof}
	\begin{enumerate}[(a)]
		\item 
			Let $E \in \mathcal{S}$, then $E \in \mathcal{T}$ and $E \subset X$ by assumption, then $E \cap X = E$. 
			Conversly, it is clear that $F \cap X \subset X$, and a $\sigma$-algebra is closed under intersections, 
			so $F \cap X \in \mathcal{T}$ hence it is also in $\mathcal{S}$. 
		\item	Let $\mathcal{S} = \{ F \cap X | F \in \mathcal{T} \}$.  
			$\varnothing \in S$ since $\varnothing \in \mathcal{T}$, Then $X \setminus (F \cap X) = (Y \setminus F) \cap X $, and 
			$Y \setminus F \in \mathcal{T}$ since $\mathcal{T}$ is a $\sigma$-algebra hence 
			this set is closed under complementation. Then given a collection of sets of the form $F \cap X$ for $F \in \mathcal{T}$. 
			we have 
			\[
				\bigcup^\infty F_i \cap X  = X \cap \Big( \bigcup^\infty F_i \Big) \in \mathcal{S}	
			\]
 	\end{enumerate}
\end{proof}


\begin{prb}
	
\end{prb}

\begin{proof}
	Let $f: B \to \R$ be a borel measurable function defined on a borel set $B$. Define $g$ as in the problem. 
	Let $C$ be a borel set and write $C = (C \cap f(B)) \cup (C \cap \{0\})$. 
	Then 
	\[
		g^{-1}(C) = g^{-1}(C \cap f(B)) \cup g^{-1}(C \cap \{0\})
	\]

	Since $C \cap f(B) \subset f(B)$ and $f$ is borel measurable, the first primage on the RHS must be a borel set. 
	Then if $C \cap \{0\} = \varnothing$ the second priamge is trivial and if $C \cap \{0\} = \{0\}$ the second primage is 
	$\R \setminus B$ wich is a borel set. Then since the union of borel sets is a borel set, $g$ is borel measureable. 
\end{proof}

\begin{prb}
	
\end{prb}

\begin{proof}
Let $f(x) > 0$ for all $x$, then we can write 
\[
	f(x)^{g(x)} = e^{\ln(f(x)^{g(x)})} = e^{g(x) \ln(f(x))}
\]
Where the RHS is $S$ measurable by the product and composistion results obtained in class. Note that $f(x) > 0$ is required so the $ \ln(f(x))$ is defined.  
\end{proof}

\begin{prb}
	
\end{prb}

\begin{proof}
 Let $\mu$ and $\nu$ be measures. Then $(\mu + \nu)(\varnothing) = 0 + 0 = 0$. let $E$  be a collection of disjoint sets. We have 
 \[
	 (\mu + \nu)(\bigcup E) = \mu(\bigcup E) + \nu( \bigcup E) = \sum^{\infty} \mu(E) + \sum^\infty \nu(E) = \sum (\mu + \nu)(E) \]

\noindent Where the last equality follows from basic facts about 
infinite series, hence $(\mu + \nu)$ is a measure. 
\end{proof}
\end{document}
