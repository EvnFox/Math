\documentclass[11pt,largemargins]{homework}
\usepackage{commath}
\newcommand{\R}{\mathbb{R}}
\newcommand{\N}{\mathbb{N}}
\newcommand{\Q}{\mathbb{Q}}
\newcommand{\Z}{\mathbb{Z}}
\newcommand{\T}{\mathfrak{T}}
\newcommand{\B}{\mathfrak{B}}
\newcommand{\C}{\mathfrak{C}}
\newcommand{\ran}{\operatorname{ran}}
\newcommand{\dom}{\operatorname{dom}}
\newcommand{\eps}{\varepsilon}
\newcommand{\ssd}{\bigtriangleup}
\newcommand{\pow}{\mathcal{P}}

% TODO: replace these with your information
% TODO: replace these with your information
\newcommand{\hwname}{Evan Fox}
\newcommand{\hwemail}{efox20@uri.edu}
\newcommand{\hwtype}{}
\newcommand{\hwnum}{}
\newcommand{\hwclass}{MTH 513}
\newcommand{\hwlecture}{}
\newcommand{\hwsection}{}

\newcommand {\mat}  [1] {\left[\begin{array}{#1}}
    \newcommand {\rix}      {\end{array}\right]}
    
\begin{document}
\maketitle


\question
Show that there doesn't exist a rational number $s$ such that $s^2 = 6$. 


\begin{proof}
    First we prove that $6|a^2 \implies 6|a$. Suppose that $6$ does not divide $a$ and let $a = p_1p_2 \dots p_n$ 
    be the prime factorization of $6$, then we know that $3$ and $2$ cannot both be common factors,  
    But then since $2$ and $3$ are prime they cannot be the square of a prime number, hence $2$ and $3$ cannot both 
    appear in $a^2 = p_1^2 p_2^2 \dots p_n^2$, hence $6$ does not divide $a^2$. It now follows by contrapositive that 
    if $6|a^2$ then $6|a$. 

    Now assume there exists $a, b \in \Z$ such that 
    \[6 = \left(\frac{a}{b}\right)^2 \]
    If $a$ and $b$ have any common factors we may cancel them out, so we assume $(a, b) = 1$. 
    Then $6b^2 = a^2$ implies $6|a$ so we may fix $m \in \Z$ such that $6m = a$. Then 
    $6b^2 = (6m)^2$ implies $b^2 = 6m^2$ and $6|b$; thus $a$ and $b$ share a common factor, 
    a contradiction. 

\end{proof}

\question
If $a, b \in \R$ show that $|a + b| = |a| + |b| $ iff $ab \geq 0$. 


\begin{proof}
    For the forward direction we use contrapositive, assume $ab < 0$ then without loss of generality assume $a > 0$ and $b<0$, we show that 
    \[|a +b | \neq |a| + |b| \] 
    we have by our assumptions on $a$ and $b$ that $|a| + |b| = a - b$. 
    Now there are three cases (by tricotomy) for what $|a+b|$ can map to, if $|a + b| = 0$
    we are done since $| a| + |b| > 0$. 
    If $a + b$ is positive $|a + b| = a + b$, but then $a + b = a - b$ would imply $b = 0$, contradicting our assumption on $b$.
    If $a + b$ is negative $|a + b| = -a -b$, but now $-a -b = a - b$ would force $a = 0$, contradicting our assumption on $a$. 
    Hence, in every case equality does not hold. This proves the first direction.
    
    Now assume $ab \geq 0$, if either is equal to zero we are done. If they are both positive it follows since $a + b$ will be positive giving 
    \[|a + b| = a + b = |a| + |b|.\]
    Then if they are both negative, their sum will be negative so,
    \[|a + b| = -a - b = |a| + |b|\]
    completing the opposite direction. 
\end{proof}

\question 
Find all $x \in \R$ that satisfy the inequality 
\[ 4 < |x + 2| + |x - 1| < 5. \] 


\begin{proof}
    The terms $x + 2$, $x - 1$ have different signs only if $x \in (-2, 1)$ but it is clear 
    no such $x$ is a solution, so assume $x \notin (-2, 1)$, then the terms have the same sign and by the above we may add them 
    to get 
    \[4 < |2x +1 | < 5\]
    which we may split into $4 < |2x + 1| $ and $|2x + 1| < 5$. 
    For the first case we have 
    \[4 < |2x + 1| \implies 4 < 2x + 1 \vee 2x + 1 < -4\]
    which gives $3/2 < x$ and $x < -5/2$; written in interval notation as $(-\infty, -5/2) \cup (3/2, \infty)$. 
    Now for 
    \[|2x + 1| < 5 \implies -5 < 2x + 1 < 5 \implies -6/2 < x < 4 = (-3, 2)\]
    We need both conditions to be satisfied so we must take the intersection over our two solutions, 
    \[(-\infty, -5/2) \cup (3/2, \infty) \cap (-3, 2) = (-3, -5/2) \cup (3/2, 2)\]
\end{proof}

\question  
\begin{alphaparts}
\questionpart

\begin{proof}
    Assume without loss of generality that $b < a$, then $|a - b| = a - b$. 
    Thus 
    \[\frac{1}{2}(a + b + |a - b|) = \frac{1}{2}(a + b + a -b ) = \frac{1}{2}(2a) = a\]
    We also have 
    \[\frac{1}{2}(a + b - |a - b|) = \frac{1}{2}(a + b -(a - b)) = \frac{1}{2}(2b) = b\]
    and we are done.  
\end{proof}

\questionpart 
Prove $\min\{a, b, c\} = \min\{\min\{a, b\}, c\}$

\begin{proof}
    Note $\min\{a, b\}$ is either $a$ or $b$, then if $c$ where such that $c < a$ and $c < b$, it is clear. Now suppose $a < b$ and $a < c$. then 
\[ \min\{a, b, c\} = a = \min\{a, c \] 
\[ = \min\{\min\{a, b\}, c\}.\]
Then $b < a$ and $b < c$ is the same as above.

\end{proof}


\end{alphaparts}


\question 

\begin{proof}
$\inf S_4 = \frac{1}{2}$, $\frac{1}{2}$ is an element of $S_4$ that occurs for $n = 2$, for $n \neq 2$, if $n$ is odd we will be adding to 1 which 
gives $\frac{1}{2} < 1 + 1/n$. If $n>2$ is even we have $1/n < 1/2$ which implies $1 - 1/2 < 1 - 1/n$, so $1/2$ is the minimal element of $S_4$. It follows that 
if a set contains a minimal element it is the infinium since we have $1/2 < x$ for all $x \in S_4$ it is a lower bound and since it is an element of $S_4$, given any lower bound $l$, we must 
have $l \leq 1/2$. 

$\sup S_4 = 2$, I proceed with a very similar argument as above, $2$ appears in $S_4$ when $n=1$, for any $n>1$ either we are subtracting from $1$, or adding a number smaller than $1$ to $1$, in either case 
we get something less than $2$. Thus $2$ is the maximal element of the set and hence it must be the supremum. 
\end{proof}


\question 
Let $A$ and $B$ be bounded nonempty subsets of $\R$, and let $A + B = \{a + b: a \in A, b \in B\}$. 
Prove that $\sup(A+B) = \sup(A) + \sup(B)$ 

\begin{proof}
    Let $\alpha = \sup (A)$ and $\beta = \sup (B)$. Then for all $a, b$ we have 
    $a \leq \alpha$ and $b \leq \beta$, thus 
    \[a + b \leq \alpha + \beta, \forall a \in A, b \in B. \] 
    So $\alpha + \beta $ is an upper bound for the set $A + B$. But then for $\epsilon > 0$ there exists $b_0 \in B$ such that 
    $ \beta - \frac{1}{2}\epsilon < b_0$ and $a_0 \in A$ such that $a - \frac{1}{2}\epsilon < a_0$, so $\alpha + \beta - \epsilon < a_0 + b_0$,
    Thus $\alpha + \beta = \sup( A + B)$. 
    A very similar argument works for infinium. Let $a = \inf(A)$ and $b = \inf(B)$. Then $a + b$ is a lower bound for $A + B$ for the same reasons stated above. 
    But then given $\epsilon > 0$, I can find elements $a_0, b_0$ in $A$ and $B$ respectively such that $a + b + \epsilon > a_0 + b_0$, so 
    $a + b = \inf(A + B)$. 
\end{proof}

\question 
I first reprove a result from class. If every element of a set $B$ is an upper bound for a set $A$, then $\inf(B)$ is an upper bound for $A$. 
\begin{proof}
    Assume every element of $B$ is an upper bound for $A$, then if $\inf(B) < a$ for some $a \in A$, there exists $\epsilon = a - \inf(B) > 0$ such that $\inf(B) + \epsilon = a$, but 
    by the epsilon formulation of infinium, we have there exists an element of $b$ with $b < a$, a contradiction. 
\end{proof}

\begin{proof}
    We prove that every element of $F = \inf\{f(x) | x \in X\}$ is an upper bound for the set $G = \{g(y)| y \in Y\}$, then it will follow $\sup(G) \leq \inf(F)$. 
    Let $y_0 \in Y$ be arbitrary, then for all $x \in X$ we have 
    \[g(y_0) \leq h(x, y_0) \leq f(x).\]
    The first inequality holds since $g(y)$ is the infinium over all choices of $x$ and the second holds since for each $x \in X$ we have defined $f(x)$ to be the supremum over $y \in Y$. 
    Then for each $g(y) \in G$, we have $g(y) \leq f(x)$ for all $x \in X$, thus each element of $F$ is an upper bound for $G$ and by the above, the result follows. 
\end{proof}

\question 
If $u > 0$ and $x < y$ show there exists a rational number $x < ru< y$. 


\begin{proof}
    Let $x, y \in \R$ with $x < y$. Then fix $n \in \N$ such that $\frac{1}{n} < y - x$, from this we immediately get $0 < 1 + xn < yn$. Now choose 
    $m \in \N$ to be the integer such that $xn < m \leq xn + 1$. Then we have 
    \[xn < m \leq xn + 1 < yn \implies xn < m < yn \implies x < \frac{m}{n} < y.\]
    Now note $\frac{x}{u}, \frac{y}{u} \in \R$ so there exists $s, t \in \Z$ such that 
    \[\frac{x}{u} < \frac{s}{t} < \frac{y}{u} \implies x < \frac{su}{t} < y \]
    and we are done. 
\end{proof}
    
\end{document}