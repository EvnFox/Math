\documentclass[11pt,largemargins]{homework}
\usepackage{commath}
\newcommand{\R}{\mathbb{R}}
\newcommand{\N}{\mathbb{N}}
\newcommand{\Q}{\mathbb{Q}}
\newcommand{\Z}{\mathbb{Z}}
\newcommand{\T}{\mathfrak{T}}
\newcommand{\B}{\mathfrak{B}}
\newcommand{\C}{\mathfrak{C}}
\newcommand{\ran}{\operatorname{ran}}
\newcommand{\dom}{\operatorname{dom}}
\newcommand{\eps}{\varepsilon}
\newcommand{\ssd}{\bigtriangleup}
\newcommand{\pow}{\mathcal{P}}

% TODO: replace these with your information
% TODO: replace these with your information
\newcommand{\hwname}{Evan Fox}
\newcommand{\hwemail}{efox20@uri.edu}
\newcommand{\hwtype}{}
\newcommand{\hwnum}{}
\newcommand{\hwclass}{MTH 513}
\newcommand{\hwlecture}{}
\newcommand{\hwsection}{}

\newcommand {\mat}  [1] {\left[\begin{array}{#1}}
    \newcommand {\rix}      {\end{array}\right]}
    

\begin{document}
    \maketitle 

    \question
    Determine all accumulation points of the following sets in $\R$ and decide if 
    the sets are open or closed
    \begin{alphaparts}
        \questionpart 
        All numbers of the form $\frac{1}{n}$ for $n \in \N$ 

        ans: There is only one accumulation point, $0$. Since 
        by the archimedean property, each nbhd around $0$ will contain 
        a point of the form $\frac{1}{k}$ for a large enough choice of $k$. 
        The set is not closed since it does not contain all its limit points, 
        it is also not open since every point is isolated. 

        \questionpart 
        All numbers of the form $2^{-n} + 5^{-m}$ for $n,m \in \N$. 

        Ans: Let $A = \{2^{-n} + 5^{-m} \, | \, n, m \in \N \}$. 
       $A$ has accumulation points, $s = \frac{1}{2^{m}}$ and $t = \frac{1}{5^{n}}$ for $n,m \in \N$
       Consider a nbhd $B(s, \epsilon)$ for  $\epsilon > 0$ around $\frac{1}{2^{-m}}$, then there exists $N \in \N$ such that $\frac{1}{n} < \epsilon$ 
       by the archimedean property. Then $\frac{1}{5^n} < \epsilon$ for $n > N$. Hence $s + 5^{-n} \in B(s, \epsilon)$; 
       so $s = \frac{1}{2^m}$ is a limit point for every choice of $m \in \N$. 

       Now fix a nbhd $B(t, \epsilon)$, since $n > N \implies \frac{1}{n} < \epsilon$ we again 
       have $\frac{1}{2^n} < \frac{1}{n} < \epsilon$ so $t + \frac{1}{2^n} \in B(t, \epsilon)$, 
       so that $t$ is a limit point. 

       This set does not contain all of its limit points so it cannot be closed. Further since $A \subset \Q \subset \R$, 
       all of its points are isolated, so it is not open. 

       \questionpart 
       $A = \{(-1)^n + \frac{1}{m} \, | \, n, m \in \N\}$.
       
       Ans: $A$ has limit points $1, -1$. Let $\epsilon > 0$, and consider $B(1,\epsilon)$. 
       then by the archimedean property, we fix $M \in \N$ such that $m > M \implies \frac{1}{m} < \epsilon$. 
       Then $1 + \frac{1}{m} \in B(1, \epsilon)$. Hence $1$ is a limit point, the argument for $-1$ is the same. 


       This set also fails to contain its limit points so it cannot be closed and again it is a subset set of $\R$ completely contained 
       in the rationals, so it cannot be open. 

    \end{alphaparts}

    \question 
    The same as Exercise 3l2 for the following sets in $\R^2$ 

    \begin{alphaparts}
        \questionpart 
        All complex numbers of the form $\frac{1}{n} + \frac{i}{m}$ for $n, m \in \N$ 

        Ans: 
        All limit points are of the form $\frac{1}{n}$ for any $n \in \N$ or $\frac{i}{m}$ for any $m \in \N$. 
        


        \questionpart 
        All points $(x, y) $ such that $x^2 - y^2 < 1$ 

        Claim: Let $S = \{(x, y) | x^2 - y^2 < 1\}$. 
         $T = \{(x, y) | x^2 + y^2 \leq 1\}$.
         Then $S^\prime = T$


         \begin{proof}
            Let $t \in T$, $t = (x, y)$. Then fix $\epsilon > 0$ and consider $B(t, \epsilon)$. 
            Then if $y > 0$, note that the point $(x, y + \frac{\sqrt{\epsilon}}{2}) \in B(t, \epsilon)$, 
            since 
            \[ \| (x, y) - (x, y + \frac{\sqrt{\epsilon}}{2}) \| = \frac{\epsilon}{4} < \epsilon \]
            and then 
            \[x^2 - (y + \frac{\sqrt{\epsilon}}{2})^2 = x^2 - y^2 - \sqrt{\epsilon}y - \frac{\sqrt{\epsilon}}{2}\]
            \[\leq 1 - \sqrt{\epsilon}y - \frac{\sqrt{\epsilon}}{2} < 1\]
            since $y > 0$. If $y < 0$, consider $(x, y - \frac{\sqrt{\epsilon}}{2})$. 
            Then $t$ is a limit point of $S$. Hence $T \subset S^\prime$. 
            Now if we take a limit point of $S$ and suppose its not in $T$ we will quickly get a contradiction 
            since we are in $\R^2$, and $x^2 + y^2 > 1$ we can fix a ball around $(x, y)$ that doesnt contain a point 
            satisfying $x^2 + y^2 < 1$. 

            This set is not closed since it does not contain all its limit points. It is however an open set, 
            since for every point in $S$, one can fix a $\epsilon$ ball such that all points in the ball satisfy our condition. 

         \end{proof}

        \questionpart 
        All points $(x, y)$ such that $x > 0$. 

        Claim: limit points are all points such that $x \geq 0$. 

        \begin{proof}
            Let $(x, y) \in \R^2$ such that $x \geq 0$, then let $\epsilon > 0$. 
            Then $(x + \epsilon/2, y)$ is contained in the $\epsilon$ ball around 
            $(x, y)$ and satisfies the condition that the first coordinate be greater than $0$. 
            hence $(x, y)$ is a limit point. 


            The set does not contain all its limit points so it is not closed. 
            It is open, for $(x, y)$ with $x > 0$, take $r = \frac{1}{2} \| x \|$ and 
            consider the ball $B((x, y), r)$. This clearly, every point in this set satisfies 
            our requirement. Then let $(a, b) \in B((x, y), r)$ and let $\| (a, b) - (x, y) \| = h$, then take $l = \frac{1}{2}(r - h) \|$. 
            Then for any point $z \in B((a, b), l)$, by the triangle inequality we have 
            \[ \|z - (x, y) \| \leq \|z - (a, b) \| + \| (a, b) - (x, y) \| \]
            \[ < r - h + h = r\]
            Hence $z \in B((x, y), r)$. Thus there is a  nbhd around $(a, b)$ completely contained in $B((x, y), r)$, and thus $(a, b)$ is an interior 
            point. Since $(a, b)$ was arbitrary it follows that every point is interior and so the ball is open. Then for every point $(x, y)$ with $x > 0$, 
            we have a ball with all points having the same property, and thus the original set in question is open.
        \end{proof}
    \end{alphaparts}
\end{document}