\documentclass[11pt,largemargins]{homework}
\usepackage{commath, amsmath, amsfonts, hyperref}
\newcommand{\R}{\mathbb{R}}
\newcommand{\N}{\mathbb{N}}
\newcommand{\Q}{\mathbb{Q}}
\newcommand{\Z}{\mathbb{Z}}
\newcommand{\ran}{\operatorname{ran}}
\newcommand{\dom}{\operatorname{dom}}
\newcommand{\eps}{\varepsilon}
\newcommand{\ssd}{\bigtriangleup}
\newcommand{\pow}{\mathcal{P}}
\newcommand{\itab}{\int_a^b f(x)d\alpha(x)}
% TODO: replace these with your information
% TODO: replace these with your information
\newcommand{\hwname}{Evan Fox}
\newcommand{\hwemail}{efox20@uri.edu}
\newcommand{\hwtype}{HW}
\newcommand{\hwnum}{1}
\newcommand{\hwclass}{MTH 436: Analysis}
\newcommand{\hwlecture}{}
\newcommand{\hwsection}{}

\begin{document}
\maketitle
\question 

\begin{alphaparts} 
	\questionpart  
	\begin{proof} 
If $\itab$ exists by the given definition the for any $\eps > 0$ there exists a $\delta > 0$ such that $\| P \| < \delta$ we have that for any choice of $t_k \in [x_{k-1}, x_{k}]$, 
$|S(P, f, \alpha) - A | < \eps$ for some $A \in \R$. We want to show that there exists $P_\eps$ 
such that taking $P \supset P_\eps$ will imply the same inequality. Taking $P_\eps = P$ where 
$P$ is any partition of $[a, b]$ with norm less than $\delta$ will product the desired result, 
since refining a partition can only decrease its norm. The fact that the integrals are the same 
follows since both are converging to the same $A$. 
\end{proof}
\questionpart  

\begin{proof} 
The existence of this integral according to definition 7.1 is ensured by a theorem proved in class. 
Namely since $f$ and $\alpha$ are not \emph{both} discontinuous from the left or right, the integral exits. On the other hand, if it exists according to the definition given in the problem then they would
have to be equal; by def 7.1, we would have the integral be equal to $f(c)[\alpha(c+) - \alpha(c-)] $, since $f(c) = 0$, we get that the integral must be 0. However, for any $\delta > 0$ we can find a partition that doesnt include $c$, and so the RS sum wil become $f(t_k)[\alpha(c+) - \alpha(c-)]$. 
for $t_k \in [x_{k-1}, x_k]$ with $x_k > c$, so by choseing $t_k > c$, we can always make the sum non zero, hence the integral doesnt exits. 
\end{proof} 
\end{alphaparts} 

\question 

\begin{proof} 
	Let $I = \itab$ where $\alpha(x) =x$ and let $M= \sup\{f(x) : x \in [a, b]\} $. 
	Given $\eps > 0$, there exits a $P_\eps$ such that $U(P_\eps, f) \leq I + \eps/2$. 
	Now set $N$ as the number of subdivisions of $P_\eps$ and let $\delta = \frac{\eps}{2MN}$. 
	Then for $\|P\| < \delta$ we have that 
	\[ U(P, f) = \sum_{k= 1}^M  M_k(f) \Delta x_k = S_1 + S_2 \] 
	Where $S_1$ contains all tems such that $[x_{k-1}, x_k]$ contains no points of $P_\eps$ and $S_2$ contains all other terms. 
	Then we have that 
	\[ S_1 \leq U(P_\eps, f) \leq I + \frac{\eps}{2} \] 
	and 
	\[ S_2 \leq MN \|P\| < MN \delta < \frac{\eps}{2} \]
	Hence adding these shows that 
	\[ U(P, f) = S_1 + S_2 \leq I + \eps\] 
	. Repeating the same process for the lower sums will give $L(P, f) > I - \eps$ for $\|P\| < \delta^\prime$. 
	Then it follows 
	\[ U(P, f) - L(P, f) < I + \eps - (I - \eps) = 2e \] 
	So by Rienmanns condition, the integral exits. 
\end{proof}

\question 
\begin{alphaparts} 
	\questionpart 
\begin{proof} 
	We use Eulers summation formula for $f(x) = x^{-s}$. 
	we have 
\[ \sum_1^n f(x) = \int_1^n f(x) dx + \int_a^b f'(x)(x \lfloor x \rfloor - \frac{1}{2}) dx + \frac{f(1) + f(n)}{2} \] 
Let the first integral be dentonted $I_1$ and the secong $I_2$. Then 
\[ I_1 = \int_1^n x^{-s}dx = \frac{1}{s - 1} - \frac{n^{-(s-1)}}{s-1} \] 
For $I_2$ we distribute $f^\prime(x)$ and use linearity to obtain 
\[I_2 = \int_1^n f^\prime(x) dx - \int_1^n f^\prime(x) \lfloor x \rfloor dx - 1/2 \int_1^n f^\prime(x) dx \] 
\[ -s \int_1^n x^{-s} dx + s\int_1^n x^{-(s+1)} \lfloor x \rfloor dx - \frac{s}{2}\int_1^n x^{-(s+1)} dx \] 
Now we have split $I_2$ into three different integrals we can deal with each one their own. First 
note that the second intgral above is what we are looking for so we will leave him alone. 
Then the first integral is the same as $I_1$ and the second will evaluate to $-\frac{1}{2n^s} + \frac{1}{2}$. 
Now for the last part of Eulers formula we have  
\[\frac{f(1) + f(n)}{2} =  \frac{1}{2n^s} + \frac{1}{2} \]  
Now putting all of this together, 
\[(1 -s) \left(\frac{1}{s-1} - \frac{1}{(s-1)(n^{s-1}} \right) + s\int_1^n \frac{1}{x^{s+1}} \lfloor x \rfloor dx - \frac{1}{2n^s} + \frac{1}{2}+ \frac{1}{2n^s} + \frac{1}{2} \] 
then we get 
\[ -1 + \frac{1}{n^{s-1}} +  s\int_1^n \frac{1}{x^{s+1}} \lfloor x \rfloor dx  - \frac{1}{2n^s} + \frac{1}{2}+ \frac{1}{2n^s} + \frac{1}{2} \] 
\[ \frac{1}{n^{s-1}} +  s\int_1^n \frac{1}{x^{s+1}} \lfloor x \rfloor dx \]
as desired 
\end{proof}

\questionpart 
\begin{proof} 
	Let $f(x) = \frac{1}{x}$ then $f^\prime(x) = \frac{-1}{x^2}$. We have 
	\[ \int_1^n f(x)dx = \ln(n) \] 
	and 
	\[ \int_1^n f^\prime(x) dx = - \int_1^n \frac{1}{x^2} dx = \frac{1}{n} - 1 \] 
	Now we apply Eulers summation formula to get 
	\[ \sum_1^n f(n) = \int_1^n f(x) dx + \int_1^n f^\prime(x)(x - \lfloor x \rfloor) dx - \frac{1}{2} \int_1^nf^\prime(x) dx + \frac{f(1)+ f(n)}{2}\] 
	reducing this using the above gives 
	\[ln(n) + \int_1^n f^\prime(x) (x - \lfloor x \rfloor) dc - \frac{1}{2}[\frac{1}{n} - 1] + \frac{1}{2n} + \frac{1}{2}  \] 
	\[ = ln(n) - \int_1^n \frac{x - \lfloor x \rfloor}{x^2}dx + 1 \] 

	as desired. 
\end{proof}


\end{alphaparts}
 

\question 
\begin{proof} 

We apply integration by parts to 
\[\int_1^{2n} f(x)d(\lfloor x \rfloor - 2 \lfloor \frac{x}{2} \rfloor) \] 
and obtain 
\[\int_1^{2n} f(x)d(\lfloor x \rfloor - 2 \lfloor \frac{x}{2} \rfloor) +\int_1^{2n}(\lfloor x \rfloor - 2 \lfloor \frac{x}{2} \rfloor)  d(f(x)) = -f(1) \] 
\[ f(1) +\int_1^{2n} f(x)d(\lfloor x \rfloor - 2 \lfloor \frac{x}{2} \rfloor) + \int_1^{2n}(\lfloor x \rfloor - 2 \lfloor \frac{x}{2} \rfloor)  d(f(x)) =0  \]  
Applying reduction to a Rienmann integral to the right most term gives what we want, so we just need to show that the two left terms  satisfies the following  
\[f(1) + \int_1^{2n} f(x)d(\lfloor x \rfloor - 2 \lfloor \frac{x}{2} \rfloor) = - \sum_{k=1}^{2n} (-1)^kf(k) \] 
so that we can add to both sides and get the result. 
Using linear properties we have that the LHS integral of the above is equal to  
\[ \int_1^{2n} f(x)d(\lfloor x \rfloor) - \int_1^{2n} f(x)d(2\lfloor \frac{x}{2} \rfloor) \] 
\begin{equation} = \sum_{k=2}^{2n} f(k) -2 \int_1^{2n} f(x)d(\lfloor \frac{x}{2} \rfloor) \end{equation}
Looking into the integral on the right, we notice that we may apply a change of variables to get 
\[ -2 \int_1^{2n} f(x)d(\lfloor \frac{x}{2} \rfloor) = -2 \int_{\frac{1}{2}}^n f(2x)d(\lfloor x \rfloor) \] 
\[= -2 \int_{\frac{1}{2}}^1 f(2x) d(\lfloor x \rfloor) -2 \int_1^n f(2x) d(\lfloor x \rfloor)\]
which will enable us to reduce the intergral to a finite sum. 
Then this becomes $-2f(2) -2 \sum_{k = 2}^n f(2x) $. I.e it is $-2$ times the sum of the even terms. So going back to what we want to show we have  
\[f(1) + \sum_{k = 2}^{2n}f(k) - 2 f(2) - \sum_{k=2}^n f(2k) = (f(1) + \dots f(2k)) - (2f(2) + 2f(4) + \dots + 2f(2n)) \]
\[ = f(1) - f(2) + f(3) + \dots - f(2n) = - \sum_{k=1}^{2n} (-1)^kf(k) \] 
which proves the result

\end{proof}
\question 
\begin{proof} 
We show that 
\[ \int_1^n f^\prime(x) [x - \lfloor x \rfloor - \frac{1}{2}] dx= - \int_1^n \phi_2(x) f''(x) dx\].
Let $\phi_1(x)$ and $\phi_2(x)$ be defined as in the problem. Notice that $\phi_2^\prime(x) = \phi_1(x)$ by the first fundamental theorem of calculus. 
Now we apply integration by parts to $\int_1^n f^\prime(x) d\phi_2(x)$. Note that the LHS will be zero since $\phi_2(n) = \int_0^n \phi_1(x) dx = 0$. 
so $f^\prime(n)\phi_2(n) - f^\prime(1)\phi_2(1) = 0$. 
Then integration by parts gives us 
\[ \int_1^n f^\prime d\phi_2(x) + \int_1^n \phi_2(x) df^\prime(x) = 0\] 
\[ \int_1^n f^\prime \phi_1(x) dx + \int_1^n \phi_2(x) f''(x)dx= 0 \] 
by applying reduction to a Riemann integral. So 
\[ \int_1^n f^\prime [x - \lfloor x \rfloor - \frac{1}{2} ] dx  = -\int_1^n \phi_2(x) f''(x) dx \] 
Since integrating over $\phi_1(x) $ or $x - \lfloor x \rfloor - \frac{1}{2}$ is the same. now one can just subsitite this integral in Eulers formulat to obtain the result. 
\end{proof}
 

\question 
\begin{proof} 
	By the second mean value theorem for RS intefrals, there exists $x_0 \in [a, b] $ such that 
	\begin{equation} \int_a^b f d\beta = f(a) \int_a^{x_0} d\beta + f(b)\int_{x_0}^b d \beta \end{equation}
	Then we may apply the first fundamental theorem of calculus to see that $\beta^\prime(x) = g(x)\alpha^\prime(x)$. 
	Then 
	\[ \int_a^{x_0} d\beta = \int_a^{x_0} g(x) \alpha^\prime(x) dx = \int_a^{x_0} g(x) d\alpha(x) \] 
	and the same is true for $\int_{x_0}^bd\beta$. Hence we have 
	\[ \int_a^b fd\beta = f(a)\int_a^{x_0} gd\alpha + f(b) \int_{x_0}^b g d\alpha \] 
	which is what we wanted to prove. 
\end{proof}
\end{document} 












































