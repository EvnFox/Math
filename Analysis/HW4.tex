\documentclass[11pt,largemargins]{homework}
\usepackage{commath, amsmath, amsfonts, hyperref}
\newcommand{\R}{\mathbb{R}}
\newcommand{\N}{\mathbb{N}}
\newcommand{\Q}{\mathbb{Q}}
\newcommand{\Z}{\mathbb{Z}}
\newcommand{\ran}{\operatorname{ran}}
\newcommand{\dom}{\operatorname{dom}}
\newcommand{\eps}{\varepsilon}
\newcommand{\ssd}{\bigtriangleup}
\newcommand{\pow}{\mathcal{P}}

% TODO: replace these with your information
% TODO: replace these with your information
\newcommand{\hwname}{Evan Fox}
\newcommand{\hwemail}{efox20@uri.edu}
\newcommand{\hwtype}{HW}
\newcommand{\hwnum}{1}
\newcommand{\hwclass}{MTH 436: Analysis}
\newcommand{\hwlecture}{}
\newcommand{\hwsection}{}

\begin{document} 
\maketitle
\question 
Show that if $f$ is lipschitz with $\alpha > 1$ then 
$f$ is constant and show that if $\alpha = 1$ then 
$f$ is of bounded variation. 

\begin{proof} 
Let $\alpha >1$. Then  we have that for all $x, y \in [a, b]$, 
\begin{equation} 
\label{eq:1}
\frac{|f(x) - f(y)|}{|x-y|} \leq M|x - y|^{\alpha -1}
\end{equation}

where $\alpha - 1 > 0$ by our assumption on $\alpha$. Now let $c \in (a, b)$, 
$f$ is differentiable at $c$ if $\lim_{x \to c} \frac{f(x) - f(c)}{x- c} $ is finite for all $c$. 
Since the absoulute value of this limit is bounted above by $M|x - y|^{\alpha -1}$, a term which goes to 0 as $x \to y$, 
$f^\prime (c) = 0$ for all $c \in (a, b)$. Then it is a consequence of the mean value theorem that a function whose deriviative is zero must 
be constant. 

Now if $\alpha = 1$, equation \ref{eq:1} becomes 
\[ \frac{|f(x) - f(y)|}{|x-y|} \leq M \]
and a similliar argument to above will show that $f^\prime$ is bounded and $f$ is clearly continuous. Thus, by a theorem proved in class will be of bounded variation. 

\end{proof}

\question 

\begin{alphaparts} 
\questionpart 
(a) 

\begin{proof} 
	First $v(x) \leq p(x) + n(x)$ is clear, since for any partition $P$ of $[a,b]$, Then we have 
	\[ v(x) = \sum_{k=1}^n \Delta f_k = \sum_{k \in A(p)} \Delta f_k + \sum_{k \in B(P)}| \Delta f_k |\] 
	\[ \leq \sup \{\sum_{k \in A(P)} \Delta f_k | P \in \mathcal{P}[a, b] \} +\sup \{\sum_{k \in B(P)}| \Delta f_k| | P \in \mathcal{P}[a, b] \} \]
	\[= p(x) + n(x) \]

	Now we must prove the opposite inequality, first note that for $P_1, P_2 \in \mathcal{P}[a, b]$ we have 
	\begin{equation} \sum_{k \in A(P_1)} \Delta f_k \leq \sum_{k \in A(P_1 \cup P_2)} \Delta f_k \end{equation} 
	and 
	\begin{equation} \sum_{k \in B(P_2)} |\Delta f_k| \leq \sum_{k \in B(P_1 \cup P_2)} |\Delta f_k| \end{equation} 
	(2) follows since if I evaluate the posistive variation at some partition $P = \{x_0, \dots, x_1 \}$, then adding a single point $u$ to $P$
	lies in some interval $[x_k, x_{k+1}]$. If $f(x_k) < f(u) < f(x_{k+1})$ Then there is nothing to prove. If $f(u) < f(x_k)$, 
	then $f(x_{k-1}) - f(u) > f(u) - f(x_k)$ so that the positive variation increases and the same is true if $f(u) > f(x_{k+1})$. 
	Hence by adding a point to a partition $P$, I cannot decrease the positve variation, then by induction, I can add any finite number of points to the 
	parition and I can only increase the positive variation. (3) is similliar. 
	
	Now we want to prove $p(x) + n(x) < v(x)$, so we prove that for any partitions $P_1$ and $P_2$, there exists a parition $P$ such that 
	\begin{equation} \sum_{k \in A(P_1)} \Delta f_k + \sum_{k \in B(P_2)} |\Delta f_k | \leq \sum_{k = 1}^n |\Delta f_k | \end{equation}  
        Where the term on the right of the inequality is being evaluated at the partition $P$.	Namely, choose $P= P_1 \cup P_2$, then using (2) and (3) and
	noticing that I can add the two terms on the right of the inequality when they are being evaluated at the same partition, the desired result follows. 
\end{proof} 

\questionpart 
(d) 

\begin{proof} 
	First note that this is clear intuitively, it is saying that the value of $f(x)$ for $x \in [a,b]$ is just the initial value at the start of the interval $f(a)$ plus the positive variation from $a$ to $x$ and minus the negitive varition (we subtract since negitive variation is defined with an absoulute value so that it is positive). We prove that $p(x) - n(x) = f(x) - f(a) = \sum_{k = 1}^n \Delta f_k$. Let $P_1$ be the partition which maximizes the positive variation and let $P_2$ maximize the negitive variation on $[a, x]$. Then let $P = P_1 \cup P_2$.  Then 
	\[ p(x) - n(x) = \sum_{k \in A(P_1 \cup P_2)} \Delta f_k + \sum_{k \in B(P_1 \cup P_2) } - |\Delta f_k| \] 
\[ = \sum_{k = 1}^n \Delta f_k = f(x) - f(a) \] 
Where we dont need an absoulte value n the second line since we reintroduced the negitive to all the terms $\Delta f_k$ with $k \in B(P_1 \cup P_2)$. 
\end{proof}

\questionpart
(e) 
\begin{proof} we have 
	\[ f(x) = f(a) + p(x) - n(x) \] 
	\[ f(x) = f(a) + p(x) - p(x) - n(x) \] 
	\[ f(x) = f(a) + 2p(x) - v(x) \] 
	\[f(x) - f(a) + v(x) = 2p(x) \] 
	where the seconed step follows from (a). The other equation follows from adding and subtracting $n(x)$ rather than $p(x)$ 
\end{proof}

\questionpart 
(f) 
\begin{proof} 

	We proved in classes that if $x \in [a, b]$ is a point of continuity for $f$ then it is also a point of continuity for $v$. Then using 
	the results of the last section, we can see that $p(x) = \frac{1}{2} (f(x) + v(x) - f(a)$ and since the sum of continuous functions is continuous, 
	we are done. The same argument applies to $n(x)$. 
\end{proof} 
\end{alphaparts}
 

\question 
10 

	\begin{alphaparts} 

		\questionpart 
		First let $g_1 = im(H)$ and $g_2 = Re(H)$, these are just analogs to first and seconed projection if we considered $H$ as a paramterized cure in $\mathbb{R}^2$. $g_1$ is continuous on $[a, 2b-a]$ by the pasting lemma. Since $V_{g_1} [a, b]  = V_{f} [a, b]$ is well defined as $f$ is of bounded variation and similiarlly for $V_{g_1}[b, 2b-a] = V_{g(x)} [a, b]$ we have by the additive property of total variation, adding the previous two numbers gives the total variation of $g_1$ on the interval $[a, 2b-1]$. A similiar story is going to hold for $g_2$, hence since the components of $H$ are of bounded variation, $H$ defines a rectifiable curve.
	
\questionpart

Look at last page.  

\questionpart
I think in the second line defining $H$, they mean 2b-t not t.

\begin{proof} 
$f-g$ is continuous since $f$ and $g$ are. Then the same approach as above will show that $Im(H)$ has a well defined total variation on $[a, b]$ and on $[b, 2b -1]$ since $H$ is continuous on the union of these intervals, The total variation will be given by the additive property of variation. 

again $S$ defines a closed region in $\mathbb{R}^2$, so its interior is all points $S \setminus intS$. But then any point not on $\Gamma_0$ but in $S$, 
will be in the interior since its coordinates are given by strict inequalities $a < x < b$ and -$1/2 (f - g) < y < 1/2 f-g$
\end{proof}
\questionpart 
This is easy to see since the imaginary component of $H$ is the $y$-axis, and in the first half of the interval $[a, b]$ the imaginary component of $H$ 
is minus its value in the second half $[b, 2b-a]$, so it is fliping the curve over the $x$-axis. also note that it is zero at $a$ and $b$, since $g$ and $f$ 
agree there. 

\questionpart 
\begin{proof} 
	$S$ is a closed set, so its boundry are all points $S \setminus intS$. For any point $(x, y)$ not on $\Gamma$ if $a < x < b$ and $f(x) < y < g(x)$ then we can find a open set of $(x, y)$ contained in $S$, so it is not in the boundry. Hence the only points are the points in $S$ that do not satisfy the above, i.e they must lie on the curve $\Gamma$.
\end{proof}

\questionpart
\begin{proof} 
The curve $\Gamma_0$ (or $\Gamma$) is traced out by the imaginary component. When computing 
$\Lambda_{\Gamma_0}(P) = \sum_{k=1}^n \|(g-f)(x_k) - (g- f)(x_{k-1})\|$ at some partition $P \in \mathcal{P}[a,2b-a]$, we see that 
\[\Lambda_{\Gamma_0}(P) = \sum_{k=1}^n \|(g-f)(x_k) - (g- f)(x_{k-1})\|\] 
\[= \sum_{k=1}^n\|g(x_k) - g(x_{k-1}) \| + \sum_{k=1}^n \|f(x_k) - f(x_{k-1})\| \] 
but this is the same as the arclength of $\Gamma$ execpt I dont get to choose two different partitions for $g$ and $f$. Hence by choosing the same partition 
for $g$ and $f$ we see that $\Lambda_{\Gamma_0}$ must be obtained by some partition for $\Lambda_{\Gamma}$
\end{proof}

	\end{alphaparts}



\question 
11 
\begin{proof} 
	Let $f$ be absolutely continuous, and let $\epsilon > 0$. Then using the definition for $n =1$, we see that it says given a subinterval $(a_1, b_1)$ such that $b_1 - a_1 < \delta$ then $|f(b_1) - f(a_2)| <  \epsilon$. So for any point $x \in [a,b]$, if $|x - y| < \delta $ then we cany take the one subinterval to be $(x, y)$ if $x < y$ or $(y, x)$ if $y < x$. 

	Now we prove that $f$ is of bounded variation. Let $P \in \mathcal{P}[a,b]$ such that $\| P\| < \delta$ and $P = \{x_0, \dots x_m\}$. 
	Then on each interval 
	$[x_{k-1}, x_k]$ consider a partition $P_k$, by grouping the points of the parition as disjoint subintervals we 
	get $\sum_{P_k} |\Delta f| < \epsilon$ since the sum of the lengths of disjoint subintervals 
	cannont exceed $x_k - x_{k-1} < \delta$ so that $f$ is clearly of bounded variation on each subinterval. 
	Then let $P^\prime = P \cup \bigcup_{k=1}^m P_k$ and we have 
	\[\sum_{p^\prime} |\Delta f_k| < m \epsilon \] 

	Since $P_k \in \mathcal{P}[x_{k-1}, x_k]$ was arbitrary and since $\bigcup_{k=1}^m P_k \in \mathcal{P}[a,b]$, we see that $f$ is of bounded variation.
\end{proof}

\question 
12 

\begin{proof} 

Fix $M$ such that $|f(x) - f(y)| \leq M |x - y|$ then given $\epsilon > 0$, let $\delta = \frac{\epsilon}{M}$. Then 
Given $(a_k, b_k)$ satisfing $\sum b_k - a_k < \delta$ we have 
\[\sum |f(b_k) - f(a_k)| \leq M \sum |b_k - a_k| < \epsilon\] 
where the first inequality followed by our assumption on $f$.
\end{proof}
\end{document}
