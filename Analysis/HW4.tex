\documentclass[11pt,largemargins]{homework}
\usepackage{commath, amsmath, amsfonts, hyperref}
\newcommand{\R}{\mathbb{R}}
\newcommand{\N}{\mathbb{N}}
\newcommand{\Q}{\mathbb{Q}}
\newcommand{\Z}{\mathbb{Z}}
\newcommand{\ran}{\operatorname{ran}}
\newcommand{\dom}{\operatorname{dom}}
\newcommand{\eps}{\varepsilon}
\newcommand{\ssd}{\bigtriangleup}
\newcommand{\pow}{\mathcal{P}}

% TODO: replace these with your information
% TODO: replace these with your information
\newcommand{\hwname}{Evan Fox}
\newcommand{\hwemail}{efox20@uri.edu}
\newcommand{\hwtype}{HW}
\newcommand{\hwnum}{1}
\newcommand{\hwclass}{MTH 435: Analysis}
\newcommand{\hwlecture}{}
\newcommand{\hwsection}{}

\begin{document} 

\question 
Show that if $f$ is lipschitz with $\alpha > 1$ then 
$f$ is constant and show that if $\alpha = 1$ then 
$f$ is of bounded variation. 

\begin{proof} 
Let $\alpha >1$. Then  we have that for all $x, y \in [a, b]$, 
\begin{equation} 
\label{eq:1}
\frac{|f(x) - f(y)|}{|x-y|} \leq M|x - y|^{\alpha -1}
\end{equation}

where $\alpha - 1 > 0$ by our assumption on $\alpha$. Now let $c \in (a, b)$, 
$f$ is differentiable at $c$ if $\lim_{x \to c} \frac{f(x) - f(c)}{x- c} $ is finite for all $c$. 
Since the absoulute value of this limit is bounted above by $M|x - y|^{\alpha -1}$, a term which goes to 0 as $x \to y$, 
$f^\prime (c) = 0$ for all $c \in (a, b)$. Then it is a consequence of the mean value theorem that a function whose deriviative is zero must 
be constant. 

Now if $\alpha = 1$, equation \ref{eq:1} becomes 
\[ \frac{|f(x) - f(y)|}{|x-y|} \leq M \]
and a similliar argument to above will show that $f^\prime$ is bounded and $f$ is clearly continuous. Thus, by a theorem proved in class will be of bounded variation. 

\end{proof}

\question 

\end{document}
