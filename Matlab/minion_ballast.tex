\documentclass[11pt]{article}

%\usepackage{ntheorem}
\usepackage{amssymb, graphicx, amsmath, amsthm}
\RequirePackage{graphicx}
\RequirePackage{hyperref}
\usepackage{commath}
\newcommand{\R}{\mathbb{R}}
\newcommand{\N}{\mathbb{N}}
\newcommand{\Q}{\mathbb{Q}}
\newcommand{\Z}{\mathbb{Z}}
\newcommand{\ran}{\operatorname{ran}}
\newcommand{\dom}{\operatorname{dom}}
\newcommand{\eps}{\varepsilon}
\newcommand{\ssd}{\bigtriangleup}
\newcommand{\pow}{\mathcal{P}}
\newcommand{\Pk}{P_\text{k}}
\newcommand{\Ck}{C_\text{k}}
\newcommand{\fit}{\operatorname{fit}}

\newtheoremstyle{break}%
{}{}%
{\itshape}{}%
{\bfseries}{}% % Note that final punctuation is omitted. 
{\newline}{}

\theoremstyle{break}
\newtheorem{thm}{Theorem}[section]
\newtheorem{defn}[thm]{Definition}
\newtheorem{lem}[thm]{Lemma}
\newtheorem{cor}[thm]{Corollary}
\newtheorem{prop}[thm]{Proposition}
\newtheorem{rem}[thm]{Remark}
\newtheorem{ex}{Exercise}[section]


% TODO: replace these with your information
% TODO: replace these with your information
\author{Evan Fox}
\title{Minion Ballast Scripts}

\begin{document}
\maketitle
\tableofcontents
\begin{abstract}
    This short report gives a detailed discription of the Minion Ballasting matlab scripts. 
    
\end{abstract}



\begin{section}{Introduction}
    The Minion is a subsurface float designed to sit at specific depths without needing to use any active form of propuslion. 
    In order to achive this the Minions need to be ballasted for the correct depth; initial ballast experiments are run in 
    a custom pressure tank and the height of the minion off the bottom is recored for each pressure. This data is necessary in order 
    to use the minion scripts because we need to estimate the compressability of the Minion. In general we use two kinds of metal to ballast, 
    lead and 316 stainless steel. Since it is easier to get very low mass 316, we get as close to the correct ballast weight with lead as we can, 
    then use 316 to make up the rest. There are 3 main scripts 
    \begin{enumerate}
        \item Minion Ballast Master
        \item Minion Ballast inLab
        \item Minion Ballast atSea
    \end{enumerate} 
    The inLab and atSea scripts just contain functions called in the master script. 

\end{section}

\begin{section}{Minion Ballast Master}
    \subsection{Minion Parameters}
        There are several Minion parameters that need to be set before the main ballast script can be run. 
        The user needs to input 
        \begin{enumerate}
            \item 'm nolead' - air weight of Minion, g No lead 
            \item 'lead air' - air weight of lead added in tank, g
            \item 'T tank' - tank temperature, C 
            \item 'S tank' - tank salinity, psu 
            \item 'pvec' - vector of pressures 
            \item 'cvec' - vector of links raised   
        \end{enumerate}

        Futher the user may ajust the bounds for $Pkeek$ and $ckeep$, which 
        are bounds for the data to be used for curve fitting. After all of the 
        approprate info has been entered, the ballast master script should be run 
        for the first time. There will be three variables output but at this point 
        only 'b lead' is of concern. Lead should be massed out as close to this value as 
        possible, then enter the amount of lead being added to 'lead sea' and run the program 
        again; this time the 'stainless sea' output is of interest. this is the amount of 
        316 stainless needed to complete in lab ballasting. The actuall amount of 316 added should 
        be as close to this value as possible without going over. However much is added should be 
        recorded in 'stainless sea'. This completes in lab ballasting. When at sea the script will 
        have to be run one last time with the appropiate condidtions set for 'P targ', 'T targ',  and 'S targ'.
        Then when the script is run the 'nuts' output is how much the 316 ballast needs to change 
        for the current condidtions. 
        
    \subsection{Screw-up Factor}
        The screw-up factor should be commented out by defualt. It exists so that if the mass 
        of the minion changes slightly after ballasting, the need to re ballast can be avoided. 
        For example, if some of the sensors on the minion are changed after ballasting, screw-up factor 
        can be uncommented and used to calculate the correct ballast. To use this part of the code, 
        simply uncomment it and set 'm nolead ini' to the inital mass without lead of the minion. 
        then simply use the code as before. 
    
\end{section}

\begin{section}{Minion Ballast inLab}
    
    $P$ and $C$ are vectors containg pressure and height data such that $(P_i, C_i)$ is the
    $i^{\text{th}}$ data point.  
    We let $\Pk$ be the vector whos entries are an appropraite restriction of $P$ we wish to fit and 
    define $\Ck$ simillarly.
    First we fit a linear polynomial $\fit(x)$ such that $\fit({\Pk}_i) \approx {\Ck}_i$.
    (more precisly the matlab function $\operatorname{polyfit}(\Pk,\Ck, n)$ returns a list of coefficients of a polynomial of degree $n$, so in the matlab script 
    $\fit(P_i)$ would be written $P_i \cdot \fit(1) + \fit(2)$ where $\fit(i)$ is the coefficient of $x^{n+1-i}$.)

    We let $p_1 = 150$ and $p_2 = 300$ then $c_1 = (ml) \cdot \fit(p_1)$ 
    and $c_2 = (ml) \cdot \fit(p_2)$ gives the water of the suspened chain. 

    Let $\rho_{t1}$, $\rho_{t2}$ be the tank water density at $p_1$ and $p_2$ respectivily (in matlab $\rho_{ti} = \operatorname{sw dens}(S_t, T_t, p_i)$).

    Now we wish to calculate the water weight of the temporary ballast given its weight in air $b_{temp}$,
    \[b_t = b_{temp} \cdot (\frac{\rho_{316} - \rho_{t1}}{\rho_{316}}) + c_{\text{extra}} \]
\end{section}

\begin{section}{Minion Ballast atSea}
    
\end{section}

\end{document}