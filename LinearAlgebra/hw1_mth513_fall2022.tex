%\documentclass[10pt,table]{article}
\documentclass[11pt,table]{article}
%
%		
%		Last Modified: Sept 9, 2022

%\documentclass[10pt]{article}
\usepackage{latexsym, amssymb, amsmath, amscd, amsthm}
\usepackage{verbatim, lscape, xcolor, mathtools}
\usepackage{enumerate, collcell, array}


\setlength{\textwidth}{6.5in}
\setlength{\textheight}{9.2in}
\setlength{\topmargin}{-.6in}
\hoffset -1.1in

\usepackage{hyperref}
\hypersetup{
	colorlinks   = true, %Colour links instead of ugly boxes
	urlcolor     = blue, %Colour for external hyperlinks
	linkcolor    = blue, %Colour of internal links
	citecolor   = red %Colour of citations
} 


%%%%%%%%%%%%%%%%%%%%%%%%%%%
%%
%%		THEOREMS AND PROOFS            
%%
%%%%%%%%%%%%%%%%%%%%%%%%%%%
%\numberwithin{equation}{section}
%\numberwithin{table}{section}
%\numberwithin{figure}{section}
\newtheorem{theorem}{Theorem}[section]
\newtheorem{proposition}[theorem]{Proposition}
\newtheorem{lemma}[theorem]{Lemma}
\newtheorem{corollary}[theorem]{Corollary}
\newtheorem{algorithm}[theorem]{Algorithm}

%%%%%%%%%%%%%%%%%%%%%%%%%%%%%%%%%%%%%%%%%%%%%%%%%%%%%%%%%%%%%%%
%\theoremstyle{definition}
\newtheorem{definition}[theorem]{Definition}
%%%%%%%%%%%%%%%%%%%%%%%%%%%%%%%%%%%%%%%%%%%%%%%%%%%%%%%%%%%%%%%
%\theoremstyle{definition}
\newtheorem{example}[theorem]{Example}
%%%%%%%%%%%%%%%%%%%%%%%%%%%%%%%%%%%%%%%%%%%%%%%%%%%%%%%%%%%%%%%
%\theoremstyle{remark}
\newtheorem{remark}[theorem]{Remark}
%%%%%%%%%%%%%%%%%%%%%%%%%%%%%%%%%%%%%%%%%%%%%%%%%%%%%%%%%%%%%%%


%%%%%%%%%%%%%%%%%%%%%%%%%%%
%%
%%		MATRIX CONSTRUCTIONS            
%%
%%%%%%%%%%%%%%%%%%%%%%%%%%%
\newcommand {\mat}  [1] {\left[\begin{array}{#1}}
\newcommand {\rix}      {\end{array}\right]}


\def\bmatrix#1		{\left[\matrix{#1}\right]}
\def\kbyk			{k \times k}
\def\mbym			{m \times m}
\def\mbyn			{m \times n}
\def\nbyn			{n \times n}
\def\pbyp			{p \times p}
\def\pbyn			{p \times n}


%%%%%%%%%%%%%%%%%%%%%%%%%%%
%%
%%		OCCASIONAL USE            
%%
%%%%%%%%%%%%%%%%%%%%%%%%%%%%%%%%%%%%%%%%%
\newcommand{\<}			{\langle}
\renewcommand{\>}      		{\rangle}
\newcommand{\nin}			{\noindent}
\newcommand{\tensor}        	{\otimes}
\newcommand{\dsum}          	{\oplus}
\newcommand{\adj}			{\textstyle \star}
	
\def\wt					{\widetilde}
\def\wh					{\widehat}
\def\ov					{\overline}
	
\newcommand{\vareps}   		{\varepsilon}
\renewcommand{\l}       		{\ensuremath{\lambda}}
\newcommand{\la}        		{\ensuremath{\lambda}}
	
\renewcommand{\a}       		{\ensuremath{\alpha}}
\renewcommand{\b}			{\ensuremath{\beta}}
\newcommand{\g}       		{\ensuremath{\gamma}}
\renewcommand{\d}			{\ensuremath{\delta}}
\newcommand{\s}       		{\sigma}  

	
%%%%%%%%%%%%%%%%%%%%%%%%%%%
%%
%%		NORM SHORTCUTS            
%%
%%%%%%%%%%%%%%%%%%%%%%%%%%%
\def\normt#1				{\|#1\|_2}
\def\normp#1				{\|#1\|_p}
\def\normtq#1				{\|\,#1\,\|_2}
\def\normo#1				{\|#1\|_\infty}
\def\norm#1				{\|#1\|}
\def\normF#1				{\|#1\|_F}


%%%%%%%%%%%%%%%%%%%%%%%%%%%
%%
%%		MATH OPERATORS            
%%
%%%%%%%%%%%%%%%%%%%%%%%%%%%
\def\max				{\mathop{\rm max}}
\def\rank				{\mathop{\rm rank}}
\def\min				{\mathop{\rm min}}
\def\rev				{\mathop{\rm rev}}
\def\diag				{\mathop{\rm diag}}
\def\det				{\mathop{\rm det}}
\def\grade				{\mathop{\rm grade}}
\def\ord				{\mathop{\rm ord}}
\def\deg				{\mathop{\rm deg}}
\def\trace				{\mathop{\rm trace}}
\def\sign				{\mathop{\rm sign}}
%\def\span{\mathop{\sf span}}
\def\colsp				{\mathop{\sf ColSp}}
\def\rowsp				{\mathop{\sf RowSp}}


%%%%%%%%%%%%%%%%%%%%%%%%%%%
%%
%%		MACROS         
%%
%%%%%%%%%%%%%%%%%%%%%%%%%%%
\def\mystrut#1{\rule{0cm}{#1}}  % E.g. 0.4cm % adds vspace after rule
	
% For fine-tuning spacing in \sqrt etc=.  From \cite[p.~155]{knut99}.
% In math mode, @ will act as a macro that adds 1 unit of space.
	\mathcode`@="8000 % Make @ behave as per catcode 13 (active).  TeXbook p. 155.
	{\catcode`\@=\active\gdef@{\mkern1mu}}
%%%%%%%%%%%%%%%%%%%%%%%%%%%%%%%%%%%%%%%%%%%%%%%%%%%%%%%%%%%%%%%


%%%%%%%%%%%%%%%%%%%%%%%%%%%
%%
%%		Uppercase GREEK Characters Italic           
%%
%%%%%%%%%%%%%%%%%%%%%%%%%%%
% Make uppercase Greek characters italic.
% Copied from latex.ltx and changed second digit from 0 (roman font)
% to 1 (math italic).
\mathchardef\Gamma="7100
\mathchardef\Delta="7101
\mathchardef\Theta="7102
\mathchardef\Lambda="7103
\mathchardef\Xi="7104
\mathchardef\Pi="7105
\mathchardef\Sigma="7106
\mathchardef\Upsilon="7107
\mathchardef\Phi="7108
\mathchardef\Psi="7109
\mathchardef\Omega="710A


%%%%%%%%%%%%%%%%%%%%%%%%%%%
%%
%%		SCRIPT LETTERS            
%%
%%%%%%%%%%%%%%%%%%%%%%%%%%%
\newcommand{\cA}		{{\cal A}}
\newcommand{\cB}		{{\cal B}}
\newcommand{\cC}		{{\cal C}}
\newcommand{\cD}		{{\cal D}}
\newcommand{\cE}		{{\cal E}}
\newcommand{\cF}		{{\cal F}}
\newcommand{\cG}		{{\cal G}}
\newcommand{\cH}		{{\cal H}}
\newcommand{\cI}		{{\cal I}}
\newcommand{\cJ}		{{\cal J}}
\newcommand{\cK}		{{\cal K}}
\newcommand{\cL}		{{\cal L}}
\newcommand{\cM}		{{\cal M}}
\newcommand{\cN}		{{\cal N}}
\newcommand{\cO}		{{\cal O}}
\newcommand{\cP}		{{\cal P}}
\newcommand{\cQ}		{{\cal Q}}
\newcommand{\cR}		{{\cal R}}
\newcommand{\cS}		{{\cal S}}
\newcommand{\cT}		{{\cal T}}
\newcommand{\cU}		{{\cal U}}
\newcommand{\cV}		{{\cal V}}
\newcommand{\cW}		{{\cal W}}
\newcommand{\cX}		{{\cal X}}
\newcommand{\cY}		{{\cal Y}}
\newcommand{\cZ}		{{\cal Z}}

%%%%%%%%%%%%%%%%%%%%%%%%%%%
%%
%%		CONSTANT Matrices            
%%
%%%%%%%%%%%%%%%%%%%%%%%%%%%
%%%%%%%%%%%%%%%%%%%%%%%%%%%%%%%%%%%%%
%%%%%%%           BLACKBOARD LETTERS             %%%%%%%%%
%%%%%%%%%%%%%%%%%%%%%%%%%%%%%%%%%%%%%
\newcommand{\bA}		{\mathbb{A}}
\newcommand{\bB}		{\mathbb{B}}
\newcommand{\bC}		{\mathbb{C}}
\newcommand{\bD}		{\mathbb{D}}
\newcommand{\bE}		{\mathbb{E}}
\newcommand{\bF}		{\mathbb{F}}
\newcommand{\bG}		{\mathbb{G}}
\newcommand{\bH}		{\mathbb{H}}
\newcommand{\bI}		{\mathbb{I}}
\newcommand{\bJ}		{\mathbb{J}}
\newcommand{\bK}		{\mathbb{K}}
\newcommand{\bL}		{\mathbb{L}}
\newcommand{\bM}		{\mathbb{M}}
\newcommand{\bN}		{\mathbb{N}}
\newcommand{\bO}		{\mathbb{O}}
\newcommand{\bP}		{\mathbb{P}}
\newcommand{\bQ}		{\mathbb{Q}}
\newcommand{\bR}		{\mathbb{R}}
\newcommand{\bS}		{\mathbb{S}}
\newcommand{\bT}		{\mathbb{T}}
\newcommand{\bU}		{\mathbb{U}}
\newcommand{\bV}		{\mathbb{V}}
\newcommand{\bW}		{\mathbb{W}}
\newcommand{\bX}		{\mathbb{X}}
\newcommand{\bY}		{\mathbb{Y}}
\newcommand{\bZ}		{\mathbb{Z}}

%%%%%%%%%%%%%%%%%%%%%%%%%%%
%%
%%		CONSTANT Matrices            
%%
%%%%%%%%%%%%%%%%%%%%%%%%%%%
%%%%%%%%%%%%%%%%%%%%%%%%%%%%%%%%%%%%%
%%%%%%%%           SANS SERIF LETTERS             %%%%%%%%%
%%%%%%%%%%%%%%%%%%%%%%%%%%%%%%%%%%%%%
\newcommand{\sA}	{\mathsf{A}}
\newcommand{\sB}	{\mathsf{B}}
\newcommand{\sC}	{\mathsf{C}}
\newcommand{\sD}	{\mathsf{D}}
\newcommand{\sE}	{\mathsf{E}}
\newcommand{\sF}	{\mathsf{F}}
\newcommand{\sG}	{\mathsf{G}}
\newcommand{\sH}	{\mathsf{H}}
\newcommand{\sI}	{\mathsf{I}}
\newcommand{\sJ}	{\mathsf{J}}
\newcommand{\sK}	{\mathsf{K}}
\newcommand{\sL}	{\mathsf{L}}
\newcommand{\sM}	{\mathsf{M}}
\newcommand{\sN}	{\mathsf{N}}
\newcommand{\sO}	{\mathsf{O}}
\newcommand{\sP}	{\mathsf{P}}
\newcommand{\sQ}	{\mathsf{Q}}
\newcommand{\sR}	{\mathsf{R}}
\newcommand{\sS}	{\mathsf{S}}
\newcommand{\sT}	{\mathsf{T}}
\newcommand{\sU}	{\mathsf{U}}
\newcommand{\sV}	{\mathsf{V}}
\newcommand{\sW}	{\mathsf{W}}
\newcommand{\sX}	{\mathsf{X}}
\newcommand{\sY}	{\mathsf{Y}}
\newcommand{\sZ}	{\mathsf{Z}}

%%%%%%%%%%%%%%%%%%%%%%%%%%%
%%
%%		CONSTANT Matrices            
%%
%%%%%%%%%%%%%%%%%%%%%%%%%%%
%%%%%%%%%%%%%%%%%%%%%%%%%%%%%%%%%%%%%
%%%%%%%%%%%%           COLORS             %%%%%%%%%%%%%
%%%%%%%%%%%%%%%%%%%%%%%%%%%%%%%%%%%%%
\usepackage{colortbl}
% \usepackage{xcolor}
%
\definecolor{lightgrey}{rgb}{.9,.9,.9}
\definecolor{mediumgrey}{rgb}{.6,.6,.6}
\definecolor{darkgrey}{rgb}{.3,.3,.3}
%
\definecolor{lightgreen}{rgb}{0.7,1.0,0.7}
\definecolor{mediumgreen}{rgb}{0.3,1.0,0.3}
\definecolor{darkgreen}{rgb}{0.0,0.7,0.0}
%
\definecolor{lightred}{rgb}{1.0,0.6,0.6}
\definecolor{mediumred}{rgb}{1.0,0.3,0.3}
\definecolor{darkred}{rgb}{0.8,0.0,0.0}
%
\definecolor{lightblue}{rgb}{0.8,0.8,1.0}
\definecolor{mediumblue}{rgb}{0.5,0.5,1.0}
\definecolor{darkblue}{rgb}{0.1,0.1,0.9}
%
\definecolor{green}{rgb}{0.0,0.7,0.0}
\definecolor{red}{rgb}{1.0,0.0,0.0}
%
\definecolor{magenta}{rgb}{.75,0,.75}
\definecolor{hotpink}{rgb}{0.9,0,0.5}
\definecolor{cyan}{rgb}{0,0.8, .8}
%
\def\tcr#1{\textcolor{red}{1}}
\def\tcb#1{\textcolor{blue}{1}}
\def\tcg#1{\textcolor{darkgreen}{1}}


%%%%%%%%%%%%%%%%%%%%%%%%%%%
%%
%%		CONSTANT Matrices            
%%
%%%%%%%%%%%%%%%%%%%%%%%%%%%
%%%%% Constant matrices 
\def\Fn{\mathbb{F}^{n}}
\def\Fm{\mathbb{F}^{m}}
\def\Fp{\mathbb{F}^{p}}

\def\Fnn{\mathbb{F}^{n\times n}}
\def\Fmm{\mathbb{F}^{m\times m}}
\def\Fpp{\mathbb{F}^{p\times p}}

\def\Fmn{\mathbb{F}^{m\times n}}
\def\Fnm{\mathbb{F}^{n\times m}}

\def\Fmp{\mathbb{F}^{m\times p}}
\def\Fpm{\mathbb{F}^{p\times m}}

\def\Fnp{\mathbb{F}^{n\times p}}
\def\Fpn{\mathbb{F}^{p\times n}}

%%%%% Constant matrices over R
\def\Rn{\mathbb{R}^{n}}
\def\Rm{\mathbb{R}^{m}}
\def\Rp{\mathbb{R}^{p}}

\def\Rnn{\mathbb{R}^{n\times n}}
\def\Rmm{\mathbb{R}^{m\times m}}
\def\Rpp{\mathbb{R}^{p\times p}}

\def\Rmn{\mathbb{R}^{m\times n}}
\def\Rnm{\mathbb{R}^{n\times m}}

\def\Rmp{\mathbb{R}^{m\times p}}
\def\Rpm{\mathbb{R}^{p\times m}}

\def\Rnp{\mathbb{R}^{n\times p}}
\def\Rpn{\mathbb{R}^{p\times n}}



%%%%% Constant matrices over C
\def\Cn{\mathbb{C}^{n}}
\def\Cm{\mathbb{C}^{m}}
\def\Cp{\mathbb{C}^{p}}

\def\Cnn{\mathbb{C}^{n\times n}}
\def\Cmm{\mathbb{C}^{m\times m}}
\def\Cpp{\mathbb{C}^{p\times p}}

\def\Cmn{\mathbb{C}^{m\times n}}
\def\Cnm{\mathbb{C}^{n\times m}}

\def\Cmp{\mathbb{C}^{m\times p}}
\def\Cpm{\mathbb{C}^{p\times m}}

\def\Cnp{\mathbb{C}^{n\times p}}
\def\Cpn{\mathbb{C}^{p\times n}}


%%%%%%%%%%%%%%%%%%%%%%%%%%%
%%
%%		BOLDFACE VECTORS            
%%
%%%%%%%%%%%%%%%%%%%%%%%%%%%
\def\bfa	{\mathbf{a}}
\def\bfb	{\mathbf{b}}
\def\bfc	{\mathbf{c}}
\def\bfd	{\mathbf{d}}
\def\bfe	{\mathbf{e}}
\def\bfo	{\mathbf{0}}
\def\bfp	{\mathbf{p}}
\def\bfq	{\mathbf{q}}
\def\bfr 	{\mathbf{r}}
\def\bfs 	{\mathbf{s}}
\def\bft	{\mathbf{t}}
\def\bfu	{\mathbf{u}}
\def\bfv	{\mathbf{v}}
\def\bfw	{\mathbf{w}}
\def\bfx	{\mathbf{x}}
\def\bfy	{\mathbf{y}}
\def\bfz	{\mathbf{z}}	

\newcommand{\zvec}    {\ensuremath{\mathbf{0}}}



%%%%%%%%%%%%%%%%%%%%%%%
%%
%%         Shortcuts to Basic Colored Boxes
%%
%%%%%%%%%%%%%%%%%%%%%%%
%%%% This command makes a basic gray box 
\usepackage{tcolorbox}
\newcommand {\basicbox} 		{\begin{tcolorbox}[width=\textwidth,colback={black!1}]}

%%% This command ends the {tcolorbox} (ALL colors)
\newcommand {\overbox}      {\end{tcolorbox}}

%%%%    Basic Boxes with Other Colors
\newcommand {\rbasicbox} 		
{\begin{tcolorbox}[width=\textwidth,colback={red!5}]}

\newcommand {\bbasicbox} 		
{\begin{tcolorbox}[width=\textwidth,colback={blue!5}]}

\newcommand {\gbasicbox} 		
{\begin{tcolorbox}[width=\textwidth,colback={green!10}]}

\newcommand {\ybasicbox} 		
{\begin{tcolorbox}[width=\textwidth,colback={yellow!10}]}


   

\setlength{\parindent}{2em}
\setlength{\parskip}{.5em}
%\renewcommand{\baselinestretch}{1.2}   
     
 


%%%%%%%%%%%%%%%%%%%%%%%%%%%%%
%%%%								%%%%%
%%%%		DOCUMENT BEGINS              	%%%%%
%%%%								%%%%%
%%%%%%%%%%%%%%%%%%%%%%%%%%%%%    
                     
\begin{document}

\thispagestyle{empty}

\begin{center}
	{\bf MTH 513  $\cdot$ LINEAR ALGEBRA}
	
	\medskip
	
	Problem Set 1
	
\end{center}

\basicbox
	
	\noindent
	{\bf POSTED} on Friday, 9 September, 2022. 
	
	\bigskip
	
	\noindent
	{\bf DUE} by 11:59pm on Sunday, 18 September 2022, via Brightspace.
	
	\bigskip
	

\noindent {\bf SUBMISSION GUIDELINES / INSTRUCTIONS:} 

%\smallskip

\begin{itemize}
	\item Review {\em general submission guidelines} before submitting your assignment, in particular how to create a single pdf document from multiple handwritten pages, page numbering, problem statements, etc. 
	
	\item Make this ``cover page'' the first page in your submitted pdf file. 
		
	\item When you are done with your work, rename the document as specified below and submit it via Brightspace. 
	
	\begin{center} 
		{\em YOURLASTNAME--hw1--mth--513} 
	\end{center} 
\end{itemize} 





\overbox


	\bigskip
	\bigskip
	
	\noindent 
	By \textcolor{red}{{\em signing your name below} (or {\em submitting this page as your cover page})}
	you are confirming that 
	you are aware of and understand
	the policies and procedures in the University Manual that pertain to Academic Honesty.
	These policies include cheating, fabrication, falsification and forgery, multiple submission,
	plagiarism, complicity and computer misuse. 
	Further information can be found in the UNIVERSITY MANUAL sections on Plagiarism and Cheating~at
	\begin{center}
		\url{http://web.uri.edu/manual/chapter-8/chapter-8-2/}
	\end{center}


	\vspace*{9mm}
	
	\noindent
	{\bf Name:} \line(1,0){250}  \\
	
	\vspace*{8mm}


\vspace{5mm}


\newpage


\begin{enumerate}

%%%%%%%%%%%%%%%%%%%%%%%%
%%%%%              Problem 1
%%%%%%%%%%%%%%%%%%%%%%%%

\item Let $X, Y \in \Rnn$ be arbitrary and assume that a good-hearted oracle 
	proved for you  the fact that  
	$ (X \cdot Y ) ^T = Y^T \cdot X^T$. 
	Use this fact to {\bf prove} that if $A_i \in \mathbb{R}^{n \times n} $ for $i=1,\ldots, k$, 
	then for $k \geq 2$ the following equality holds 
	\begin{equation} \label{eqn-prob1}
	\Big(A_1A_2\cdots A_{k-1} A_k \Big)^T \, = \, A_k^TA_{k-1}^T \cdots A_2^T A_1^T \, .
	\end{equation} 

	
{\bf Remark:} 
This is essentially just asking you 
to set up {\em proof by induction} correctly. 
The ``harder'' part of this problem would be proving the base case, 
which you may assume is true for the purpose of this problem right now. 

\medskip
%%%%%%%%%%%%%%%%%%%%%%%%
%%%%%              Problem 2
%%%%%%%%%%%%%%%%%%%%%%%%
\item \quad {\, } 

\basicbox
Recall that the set of complex numbers is defined as 
\begin{equation} \label{eq:complex-num-set}
\mathbb{C} 
\, := \, 
\big\{ 
a + b {\tt i} \, \big| \,\, a, b \in \bR, \, {\tt i}^2 = -1 
\big\} 
\, .
\end{equation} 
Moreover, the {\em addition} and the {\em multiplication}, 
$+: \bC \times \bC \rightarrow \bC$ 
and
$\cdot : \bC \times \bC \rightarrow \bC$, 
are defined~as 
\begin{eqnarray} \label{eq:add-cmplx-num-def}
\big( a + b {\tt i} \big) 
+ 
\big( c + d {\tt i} \big) 
& := & 
\big(a + c \big) + \big(b + d \big) {\tt i} \, ,
\\[2mm]  \label{eq:mult-cmplx-num-def}
%%
\big( a + b {\tt i} \big) 
\cdot
\big( c + d {\tt i} \big) 
& := & 
\big(ac - bd \big) + \big(bc + ad \big) {\tt i} \, ,
\end{eqnarray} 
for all complex numbers $a+b{\tt i}$ and $c + d {\tt i}$. 
Clearly, the set of complex numbers whose imaginary part is 
zero represents the set of real numbers, 
that is, $\bR$ is a proper subset of $\bC$. 
\overbox

\medskip

Let $\bR^{2 \times 2}$ be the set of all $2 \times 2$ 
real matrices and consider function 
$\phi: \bC \rightarrow \bR^{2 \times 2}$ 
given by 
\begin{equation} \label{eq:iso-cmplx-2x2}
\phi(a + b{\tt i}) 
\, := \, 
\mat{rr}
a & -b \\
b & a 
\rix \, , 
\qquad 
\textup{for all } \, a + b {\tt i} \in \bC \, .
\end{equation}
\begin{enumerate}[\rm (a)]
\item 
Show that $\phi$ is an injective (or one-to-one) function. 

\item 
Describe the range/image of $\phi$, that is, 
describe the set $\phi(\bC) = \left\{ \phi(z) \, | \, z \in \bC \right\}$. 

\item 
Prove or disprove: $\phi (z_1 + z_2) \, = \, \phi(z_1) + \phi(z_2)$ 
for all $z_1, z_2 \in \bC$. 

\item 
Prove or disprove: $\phi (z_1 \cdot z_2) \, = \, \phi(z_1) \cdot \phi(z_2)$ 
for all $z_1, z_2 \in \bC$. 


\end{enumerate} 


\medskip
%%%%%%%%%%%%%%%%%%%%%%%%
%%%%%              Problem 3
%%%%%%%%%%%%%%%%%%%%%%%%
\item 
The ``same'' way the set of real numbers 
was extended into the set of complex numbers, 
one can also continue this process and look 
for an extension of complex numbers. 
To that end, let us consider the set 
\begin{equation} \label{eq:quat-def}
\bH \, := \, 
\big\{ 
a + b {\tt i} + c {\tt j} + d {\tt k}  \, \big| \,\, 
a, b, c, d \in \bR 
\big\} \, ,
\end{equation} 
where the following multiplication conditions are imposed: 
\begin{enumerate}[\rm (i)] 
\item 
$ {\tt i}^2 = {\tt j}^2 ={\tt k}^2 = -1$,

\item 
${\tt i}{\tt j}={\tt k}, \,\,\, 
{\tt j}{\tt i}=-{\tt k}, \, \,\,
{\tt j}{\tt k}={\tt i}, \, \,\,
{\tt k}{\tt j}=-{\tt i}, \, \,\,
{\tt k}{\tt i}={\tt j}, \, \,\,
{\tt i}{\tt k}=-{\tt j},$

\item 
every $a\in \bR$ commutes with ${\tt i}, {\tt j}, {\tt k}$.
\end{enumerate} 

Now the addition of two elements in $\bH$ can be defined 
analogous to complex numbers where we simply 
add the corresponding parts. 
The multiplication is slightly more difficult, 
but it can be completely defined given the conditions 
(i)-(iii) from above along with the distributive law. 

\medskip

{\bf Observation:} 
$\bH$ is {\em NOT} a field since multiplication is not commutative, for example, 
condition (ii) gives us $
{\tt i}{\tt j} \, = \, {\tt k} \, \ne \,  -{\tt k} \,=\, {\tt j}{\tt i}$. 
However, except the commutativity of multiplication $\bH$ does satisfy 
all other conditions to be a field. 

\medskip

{\bf Example:} 
Multiply $({\tt i}+{\tt j})({\tt i}-{\tt j})$ assuming the distribute law 
and the conditions (i)-(iii). 
\[
({\tt i}+{\tt j})({\tt i}-{\tt j}) 
\, = \, 
{\tt i}^2 - {\tt i}{\tt j} + {\tt j}{\tt i}-{\tt j}^2 
\, = \, 
-1-{\tt k}-{\tt k}-(-1) 
\, = \, 
-2{\tt k} \, , 
\]
while ${\tt i}^2 - {\tt j}^2 = -1 - (-1) = 0$.

\medskip

\begin{enumerate}[\rm (a)]
\item 
Let $q = a + b{\tt i} + c{\tt j} + d{\tt k}$ 
and $w = e + f{\tt i} + g{\tt j} + h{\tt k}$ 
be two arbitrary elements of $\bH$. 
Write the product $q\, w$ in the 
form $z_0 + z_1 {\tt i} + z_2 {\tt j} + z_3 {\tt k}$, 
where $z_1, z_2, z_3, z_4 \in \bR$.  

\item 
Find $A \in \bR^{4 \times 4}$ and $\bfb \in \bR^4$ 
(both obviously related to $q$ and/or $w$) such that 
\[
A \, \bfb \, = \, 
\mat{c} 
z_1 \\ z_2 \\ z_3 \\ z_4
\rix \,.
\]

{\bf Remark:} Part (b) may have to wait until the second homework, 
but we can discuss it in class on Tuesday. 
\end{enumerate} 


\medskip
%%%%%%%%%%%%%%%%%%%%%%%%
%%%%%              Problem 4
%%%%%%%%%%%%%%%%%%%%%%%%
\item 
Consider the following system of equations over 
the finite field  $\bZ_3$, that is, all coefficients and operations are done over $\bZ_3$,
\begin{equation} \label{eqn-prob2}
\begin{array}{rcrcrcr}
x & + & 2y & + & z & = & 1 \\
x & + &  &  & z & = & 1 \\
x & + & y & + & z & = & 1 
\end{array}  
\end{equation}






\begin{enumerate}
	\item  What is the reduced row echelon form of the associated augmented matrix? Write down the sequence of operations you performed to obtain the reduced row echelon form. 
	
	\item Clearly describe the solution set and state how many different solutions are there. 
\end{enumerate}

\end{enumerate}
\end{document}

%%%%%%%%%%%%%%%%%%%%%%%%%

