\documentclass[11pt,largemargins]{homework}
\newcommand{\R}{\mathbb{R}}
\newcommand{\N}{\mathbb{N}}
\newcommand{\Q}{\mathbb{Q}}
\newcommand{\Z}{\mathbb{Z}}
\newcommand{\ran}{\operatorname{ran}}
\newcommand{\dom}{\operatorname{dom}}

\newcommand{\eps}{\varepsilon}
\newcommand{\ssd}{\bigtriangleup}
\newcommand{\pow}{\mathcal{P}}

% TODO: replace these with your information
% TODO: replace these with your information
\newcommand{\hwname}{Evan Fox}
\newcommand{\hwemail}{efox20@uri.edu}
\newcommand{\hwtype}{Chapter}
\newcommand{\hwnum}{1.2}
\newcommand{\hwclass}{Understanding Analysis}
\newcommand{\hwlecture}{}
\newcommand{\hwsection}{}

\begin{document}
\maketitle

\question
Give a definition for greatest lower bound and prove a lemma analagous to 1.3.8 

\newtheorem{definition}{Definition}
\begin{definition}
    
 Let $A \subseteq \R$ and let $l \in \R$. We say that $l = \inf(A)$ if and only if 
\begin{enumerate}
    \item $l$ is a lower bound for $A$ (i.e. $l \leq a$ for all $a \in A$).
    \item For an arbitrary lower bound $L$, we have that $L \leq l$. 
\end{enumerate}
\end{definition}
\newtheorem{lemma}{Lemma}[definition]
\begin{lemma}
    Assume that $l \in \R$ is a lower bound for a set $A \subseteq \R$. Then, $l = \inf(A)$ if and only if, for all choices $\eps > 0$, we have that 
    $l + \eps > a$ for some $a \in A$. 
\end{lemma}

\begin{proof}
    Assume $l = \inf(A)$. Then note for all $\eps \geq 0$ that $l < l + \eps$. Then since $l$ is the greatest lower bound for $A$
    by definiton, we have that $l + \eps$ is not a lower bound. But then there must exist $a \in A$ such that $l + \eps > a$. 

    To prove the Other direction assume we have $l \in \R$ such that $l$ is a lower bound for $A$ with the property that 
    for all $\eps > 0$ we have $l + \eps > a$ for some $a \in A$. For the sake of contradiction assume that we have $L \in \R$ such that $L > l$ and that $L = \inf(A)$. Then we note that by choosing $\eps = -l + L$
    we get that $L > a$ for some $a \in A$. Hence $L$ is not a lower bound for $A$ contradicting our assumtion. Hence $l = \inf(A)$.  
\end{proof}

\question
For each part either given an example or state that the request is impossible. 

\begin{alphaparts}
    \questionpart
        A set B with $\inf(B) \geq \sup(B)$. 

        Consider the set $B = \{0\}$. It is clear that $B \subset \R$ and that $\inf(B) = \sup(B) = 0$. 

    \questionpart
        A set that contains its infinium but not it's supprenum. 

        Consider the set $[0, 1)$. 

    \questionpart
        A set $B \subseteq \Q$ that contains its supprenum but not its infinium. 

        Consider the set $B = \{x \in \Q | 0 < x \leq 1 \} $. 

\end{alphaparts}

\question 

\begin{alphaparts}
    \questionpart 
    Let $A$ be non-empty and bounded below. Then
    Define $B = \{ b \in \R | \text{ b is a lower bound for A} \}$. Prove that 
    $\sup(B) = \inf(A)$. 

    \begin{proof}
        We fix $s \in \R$ such that $s = \inf(A)$. Then we know the $s \in B$ since $s$ is a lower bound for $A$. We also have that 
        for an arbitrary element $l \in B$, $s \geq l$; then $s$ is an upper bound for $B$ and must be the least upper bound since 
        $s \in B$ and any arbitrary upper bound $S$ must then satisfy $s \leq S$. 
    \end{proof}

    \questionpart 
    Use the result from (a) to argue why there is no need to assert that greatest lower bounds exist in the axiom of completness. 

    The result from part (a) shows shows us that we can define the greatest lower bound as the supprenum of the set of lower bounds. 
    Hence the assertion that all sets bounded above have a least upper bound implys that all sets bounded below have a greatest lower bound. 
\end{alphaparts}

\end{document}
