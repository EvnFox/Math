\documentclass[11pt,largemargins]{homework}
\newcommand{\R}{\mathbb{R}}
\newcommand{\N}{\mathbb{N}}
\newcommand{\Q}{\mathbb{Q}}
\newcommand{\Z}{\mathbb{Z}}
\newcommand{\ran}{\operatorname{ran}}
\newcommand{\dom}{\operatorname{dom}}

\newcommand{\eps}{\varepsilon}
\newcommand{\ssd}{\bigtriangleup}
\newcommand{\pow}{\mathcal{P}}

% TODO: replace these with your information
% TODO: replace these with your information
\newcommand{\hwname}{Evan Fox}
\newcommand{\hwemail}{efox20@uri.edu}
\newcommand{\hwtype}{Chapter}
\newcommand{\hwnum}{1.2}
\newcommand{\hwclass}{Understanding Analysis}
\newcommand{\hwlecture}{}
\newcommand{\hwsection}{}

\begin{document}
\maketitle

\question
Give a definition for greatest lower bound and prove a lemma analagous to 1.3.8 

\newtheorem{definition}{Definition}
\begin{definition}
    
 Let $A \subseteq \R$ and let $l \in \R$. We say that $l = \inf(A)$ if and only if 
\begin{enumerate}
    \item $l$ is a lower bound for $A$ (i.e. $l \leq a$ for all $a \in A$).
    \item For an arbitrary lower bound $L$, we have that $L \leq l$. 
\end{enumerate}
\end{definition}
\newtheorem{lemma}{Lemma}[definition]
\begin{lemma}
    Assume that $l \in \R$ is a lower bound for a set $A \subseteq \R$. Then, $l = \inf(A)$ if and only if, for all choices $\eps > 0$, we have that 
    $l + \eps > a$ for some $a \in A$. 
\end{lemma}

\begin{proof}
    Assume $l = \inf(A)$. Then note for all $\eps \geq 0$ that $l < l + \eps$. Then since $l$ is the greatest lower bound for $A$
    by definiton, we have that $l + \eps$ is not a lower bound. But then there must exist $a \in A$ such that $l + \eps > a$. 

    To prove the Other direction assume we have $l \in \R$ such that $l$ is a lower bound for $A$ with the property that 
    for all $\eps > 0$ we have $l + \eps > a$ for some $a \in A$. For the sake of contradiction assume that we have $L \in \R$ such that $L > l$ and that $L = \inf(A)$. Then we note that by choosing $\eps = -l + L$
    we get that $L > a$ for some $a \in A$. Hence $L$ is not a lower bound for $A$ contradicting our assumtion. Hence $l = \inf(A)$.  
\end{proof}

\question
For each part either given an example or state that the request is impossible. 

\begin{alphaparts}
    \questionpart
        A set B with $\inf(B) \geq \sup(B)$. 

        Consider the set $B = \{0\}$. It is clear that $B \subset \R$ and that $\inf(B) = \sup(B) = 0$. 

    \questionpart
        A set that contains its infinium but not it's supprenum. 

        Consider the set $[0, 1)$. 

    \questionpart
        A set $B \subseteq \Q$ that contains its supprenum but not its infinium. 

        Consider the set $B = \{x \in \Q | 0 < x \leq 1 \} $. 

\end{alphaparts}

\question 

\begin{alphaparts}
    \questionpart 
    Let $A$ be non-empty and bounded below. Then
    Define $B = \{ b \in \R | \text{ b is a lower bound for A} \}$. Prove that 
    $\sup(B) = \inf(A)$. 

    \begin{proof}
        We fix $s \in \R$ such that $s = \inf(A)$. Then we know the $s \in B$ since $s$ is a lower bound for $A$. We also have that 
        for an arbitrary element $l \in B$, $s \geq l$; then $s$ is an upper bound for $B$ and must be the least upper bound since 
        $s \in B$ and any arbitrary upper bound $S$ must then satisfy $s \leq S$. 
    \end{proof}

    \questionpart 
    Use the result from (a) to argue why there is no need to assert that greatest lower bounds exist in the axiom of completness. 

    The result from part (a) shows shows us that we can define the greatest lower bound as the supprenum of the set of lower bounds. 
    Hence the assertion that all sets bounded above have a least upper bound implys that all sets bounded below have a greatest lower bound. 
\end{alphaparts}

\question 
Let $A_1, A_2, A_2, ... $ be a collection of non empty sets that are bounded above. 

\begin{alphaparts}
    \questionpart
    Find a formula for $\sup(A_1 \cup A_2)$ and extend it to finite unions 
\end{alphaparts}


\question
Let $A \in \R$ be bounded above and let $c \in \R$. Then define $cA = \{ ca | a \in A \} $. 

\begin{alphaparts}
    \questionpart
    For $c \geq 0$, prove $\sup(cA) = c\sup(A)$. 

    \begin{proof}
        Let $c  \geq 0$ and fix $s \in \R$ such that $s = \sup(A)$. Then $cs \geq ca $ for all $a \in A$. Then if $b$ is an arbitrary upper bound of $A$ then 
        $s \leq b$, hence $cs \leq cb$ where $cb$ is an arbitrary upper bound of $cA$. So  $cs = c\sup(A) = \sup(cA)$.
    \end{proof}

    \questionpart
    For $c < 0$, prove $\sup(cA) = c\inf(A)$. 

    \begin{proof}
        Assume $c < 0 $ and fix $s, i \in \R$ such that $s = \sup(cA)$ and $i = \inf(A)$. Then since $i \leq a$ for all $a \in A$ we have 
        $ci \geq ca$ for all $a \in A$. Hence $i$ is a upper bound for $cA$. To prove $s = ci$ first suppose that $s < ci$; then
        for all $a \in A$ we have $ci > s \geq ca$ which is equivalent to $i < s \leq a$, contradicting $i = \inf(A)$. Now suppose that 
        $s > ci$; since we already know that $ci \geq ca$ for all $a \in A$ we see this contradicts $s = \sup(cA)$. Hence we must have $s = ci$. 

    \end{proof}

    Below is my attempt to shorten the argument and do a direct proof. In general I feel like using contradiction is easier for me. 
    \begin{proof}
        Assume $c < 0$ and let $L \in \R$ be an arbitrary lower bound for $A$, then we note that since $L \leq a$ for all $a \in A$, we have that 
        $cL \geq ca$ for all $a \in A$; therefore $cL$ is an arbitrary upper bound for $cA$. Now we fix $i \in \R$ such that $i = \inf(A)$, then for our arbitrary lower bound L we have $i \geq L$ and $ci \leq cL$. Hence, $ci = sup(cA)$. 
    \end{proof}
\end{alphaparts}

\question
Prove that if $a \in A$ and $a$ is an upper bound for $A$, then $a = \sup(A)$. 

\begin{proof}
    Fix $a \in \R$ such that $a \in A$ and $a$ is an upper bound for $a$. Then to prove that $a = \sup(A)$, note that 
    if $b$ is an arbitrary upper bound for $A$ then $x \leq b$ for all $x \in A$. Then since $a \in A$ it is clear that $a \leq b$. 
    Hence, $a$ satisfys the definition of a least upper bound. 
\end{proof}

\question
    if $\sup(A) < \sup(B)$ then there is some element $b \in B$ that is an upperbound for $A$.

    \begin{proof}
        We use the contrapostive. Assume that there is no $b \in B$ that bounds $A$ above. Then $\sup(A) > b$ for all $b \in B$. Hence $\sup(A)$ is an upper bound for $B$ and so 
        $\sup(B) \leq \sup(A)$. 
    \end{proof}


\question

\begin{definition}[The Cut Property]
    If $A, B \subseteq \R$ such that $A \cap B = \varnothing$ and $A \cup B = \R$ and for all $a \in A $ and $b \in B$ we have $a < b$ then then there exists $c \in \R$ such that 
    $c \geq a$ for all $a \in A$ and $c \leq b$ for all $b \in B$. 
\end{definition}

Prove that the cut property is equivalent to the axiom of completness (i.e. prove that they imply each other).

\begin{proof}
    We begin by assuming the cut property so suppose that we have $A \subseteq \R$ then define $B = \{x \in \R | (\forall a \in A)(x > a) \}$. Then we may fix $c \in \R$ such that $a \leq c \leq b$ for all $a \in A$ and all $b \in B$. 
    Then it is clear that $c$ is an upper bound for $A$. Then to prove that $c$ is the least upper bound we first condsider the case where $c \in A$ then by previous result it follows that $c = \sup(A)$. If $c \notin A$ then $B$ must be the set of all upper bounds; then since
    we have $c \leq b$ for all $b \in B$ we see that $c = \sup(A)$ as desired. 

    To prove the other direction we assume the axiom of completness, that is, assume every bounded subset of $\R$ has a least upper bound. Then let $A \subseteq \R$ that is bounded above. Then we may fix $\alpha \in \R$ such that $a \leq \alpha$ for all $a \in A$. Now we define  $B = \{x \in \R | (\forall a \in A)(x > a) \}$. 
    then since every element of $B$ is an upper bound for $A$ we must have $a \leq \alpha \leq b$ for all $a \in A$ and all $b \in B$. 
\end{proof}


\question
Prove or disprove the following 

\begin{alphaparts}
    \questionpart
    Let $A$ and $B$ be nonempty subsets of $\R$ such that $A \subseteq B$. Then $\sup(A) \leq \sup(B)$. 

    \begin{proof}
        Note that since $A \subseteq B$ we must have $\sup(B) \geq a$ for all $a \in A$; so $\sup(B)$ is an upper bound for $A$. Then $\sup(A) \leq \sup(B)$ by definition. 
    \end{proof}

    \questionpart
    if $\sup(A) < \inf(B)$ then there exists $c \in \R$ satisfying $a < c < b$ for all $a \in A$ and all $b \in B$. 

    \begin{proof}
        Since $\sup(A) < \sup(B)$ there exists $\eps \in \R$ such that $\sup(A) +  \eps = \sup(B)$. Then consider $\sup(A) + \frac{\epsilon}{2}$; it is clear that $\sup(A) < \sup(A) + \frac{\epsilon}{2} < \sup(B)$
    \end{proof}

    \questionpart
    if there exists $c \in \R$ satisfying $a < c < b$ for all $a \in A$ and all $b \in B$ then  $\sup(A) < \inf(B)$. 

    This is false
    \begin{proof}
        Consider the counter example $A = (0, 2)$ and $B = (2, 4)$. 
    \end{proof}
\end{alphaparts}

\end{document}
