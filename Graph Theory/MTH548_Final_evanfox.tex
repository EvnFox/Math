\documentclass{article}

\usepackage{import}
\usepackage{pdfpages}
\usepackage{transparent}
\usepackage{xcolor}
\usepackage{amsmath, amsthm, amsfonts, amsthm, mathtools, mathrsfs}
\usepackage{graphicx} 
\usepackage{tikz, tikz-cd} 
\usepackage{hyperref} 
\usepackage[margin=1in]{geometry} 


\newcommand{\R}{\mathbb{R}}
\newcommand{\Q}{\mathbb{Q}}
\newcommand{\Z}{\mathbb{Z}}
\newcommand{\N}{\mathbb{N}}
       %	\newcommand{\incfig}[2][1]{%
%	    \def\svgwidth{#1\columnwidth}
%	    \import{./figures/}{#2.pdf_tex}
%}
%\pdfsuppresswarningpagegroup=1

\newtheorem{prb}{Problem}

\title{Final Exam}
\author{Evan Fox} 
\date{}


\begin{document}
\maketitle
	\begin{prb}
	    	Let $G$ be a triangle free graph, prove 
		\begin{enumerate}
			\item $|E(G)| \leq \alpha(G) \beta(G)$
				\item $E(G) \leq \frac{n^2}{4} $
		\end{enumerate}
	    \end{prb}
	    \begin{proof}
		    (i) First note that since $G$ is triangle free, $\alpha(G) \geq \Delta(G)$, since the neighborhood of any point must 
		    be an independant set. Then if $M$ is a minimal vertex cover so that $|M| = \beta(G)$, ever edge in 
		    $G$ is incident to some vertex in $v \in M$, further, since each vertex has at most $\Delta(G)$ edges incident to it, 
		    the sum $ \sum_{v \in M} deg(v) \geq |E(G)|$. Hence, 
		    \[
		    	|E(G)| \leq \Delta(G) \beta(G) \leq \alpha(G) \beta(G)
		    \]
		
		    (ii) recall $\alpha(G) + \beta(G) = n$, for simplicity, let $a = \alpha(G)$ and $ b= \beta(G)$. So $a + b = n$. Then it follows from basic 
			calculus that $f(a, b) = ab$ is maximised for $a = \frac{n}{2} $ and $b = \frac{n}{2} $. Hence, $ab \leq \frac{n^2}{4} $ will follow. 
			To see this, note $a = n -b$ so $ab = (n - b)b$. Taking the first derivitive and setting equal to zero we obtain 
			\[
				n - 2b = 0 \implies b = \frac{n}{2} 
			\]
			which imlies $a = \frac{n}{2} $. 
	    \end{proof}


	    \begin{prb}
	    	\begin{enumerate}
	    		\item Show that every graph has a vertex ordering according to which the greedy coloring algotithm uses $\chi (G)$ colors. 
			\item  For every integer $n \geq 2$, find a bipartite graph with $2n$ vertices such that there exists a vertex ordering relative to which the greedy coloring alforithm uses $n$ colors. 
	    	\end{enumerate}
	    \end{prb}

	    \begin{proof}
		    (i) Given a coloring, let $C_i$ denote the set of all vertices with color $i$. 
		    Let $\phi$ be a coloring with $\chi (G)$ colors. Then we may assume that for any $v \in C_i$, for all $j < i$ there exists 
		    $x_j \in C_j$ such that $v x_j \in E(G)$; if this was not the case, then there exists some $j < i$ such that $v \in C_i$ is 
		    not adjacent to any vertex in $C_j$, hence we may color $v$ with color $j$ instead. 
		    Now we give an ordering by associating the vertices to natural numbers. We let the vertices in $C_1$ have $\{1, \dots , |C_1| \}$ 
		    and for $i > 1$, the vertices in $C_i$ have $\{|C_{i-1}| + 1, \dots, |C_{i}| \}$ (The order within a color class does not matter). 
		    When this is colored by the greedy algorithm, all vertices in $C_1$ get color one, there is no edge between them since $\phi$ was 
		    a coloring. Similliarly, all vertices in $C_2$, must get color $2$, since each $v \in C_2$ is adjacent to a vertex in $C_1$. 
		    In general, it is clear that the vertices in $C_i$ must get color $i$ and so we obtain our coloring 
		    $\phi$, a coloring with $\chi(G)$ colors. 

		    (ii) The case where $n=2$ is clear, since the graph is bipartite, it is two colorable. I have drawn a picture for $n = 3$ 
		    which is attacthed to this document. The vertices are ordered $v_1, \dots, v_6$, and the colors are the circled numbers. 

		    The key important property of this example, is that in both partite sets, all three colors appear.  
		    This means that to construct an example for $n = 4$, one can add two vertices $v_7, v_8$
		  below (one to each partite set), and connect each* to the 
		    first 3 vertices in the opposite partite set. Since this graph contains 
		    the $n=3$ example, it is clear that vertices $v_1, \dots v_6$ will use 3 colors, and that all three colors will appear in each set. 
		    
		    Then since $v_7$ and $v_8$ are incident to colors 
		    1, 2, 3, they both must be given color $4$. Now inductivly for $n \geq 4$ Condiser the counterexample for $n-1$, and add two vertices, 
		    putting them last in the ordering and let them have edges to the first $n-1$ vertices in the opposiate partite set, then just as above 
		    this give a counter example for $n$. \newline 
		    * the explicit edges being added are 
		    $\{v_7, v_6\}, \{v_7, v_2\}, \{v_7, v_4\} $ and 
		    $\{v_8, v_1\}, \{v_8, v_3\}, \{v_8, v_5\}$
	    \end{proof}

	   \begin{prb}
		   Show that every $3$-regular graph on $n$ vertices has a matching which covers at least $ \frac{3}{4}n $ vertices. 
	   \end{prb} 

	   \begin{proof}
	   	By the handshaking lemma, $|E| = \frac{3n}{2} $, now from results in class, the chromatic index of $G$ is at most $\Delta(G) + 1 = 4$. 
		Hence, given a (proper) edge coloring, at least one color class has $ \frac{1}{4} |E|$ edges, (if not then all color class contain less, 
		and we would get a contradiction on the size of $|E|$ by summing over all color classes). Hence there is a color class with 
		$ \frac{1}{4} \cdot \frac{3n}{2} $ edges, by the definition of a proper coloring, a color class forms a matching, and the number 
		of vertices will be 2 times the number of edges, so we get $ \frac{3n}{4} $. 
	   \end{proof}

	   \begin{prb}
	   	Ler $G$ be a graph. Set $l(G)$ to be the length of a longest path in $G$. Show that $\chi(G) \leq 1 + l(G)$
	   \end{prb}

	   \begin{proof}
		if $|G| = 1$ there is nothing to prove, $\chi(G) = 1$ and $l(g) + 1 = 1$. 
	   	Let $|G| = 2$, then if $G$ has an edge $\chi(G) = 2$ and $l(G) + 1 = 1 + 1 = 2$. and if $G$ doesnt have an edge $\chi(G) = 1$ and 
		$l(G) + 1 = 0 + 1 = 1$. So the base case holds. Now, assume that the statement holds for $n-1 > 2$. Let $G$ be a graph on $n$ vertieces and let $P$ be the longest path in $G$, 
		then for $x \in P$ a endpoint, condider $G \setminus x$, since this graph is on $n - 1$ vertices we have $\chi(G \setminus x) 
		\leq l(G \setminus x) + 1$. Then the following two equations hold, 
		\begin{equation}
			\chi(G) -1 \leq \chi(G\setminus x) \leq \chi(G) \, \text{ and } \, l(G) - 1 \leq l(G \setminus x) \leq l(G)
		\end{equation}
Then in all cases we will get  the desired result. For example if $\chi(G \setminus x) = \chi(G) $ and $l(G \setminus x) = l(G)-1$, we have
		\[
		  \chi(G \setminus x) \leq l(G \setminus x) + 1	
		\]

		\[
			\chi(G) \leq l(G \setminus x) + 1
		\]

		\[
			\chi(G) \leq l(G) \leq l(G) + 1
		\] 
		The other cases are similiar 
	   \end{proof}
\begin{prb}
	Skipping 5
\end{prb}

\begin{prb}
	Ler $T$ be a tree on $k \geq 2$ vertices. Prove that the Turan number $ex(n, T) < n(k -1)$. 
\end{prb}

\begin{proof}
	Suppose not, then let $G$ be a graph with $n$ vertices and $n(k-1)$ edges. Then, we know from lecture 2 that there exsits 
	$H \subset G$ with $\delta(H) > \epsilon(H) \geq \epsilon(G)$, then 
	$\delta(H) > e(G) = \frac{n(k-1)}{n} = k -1$.  Then in lecture 7 we proved via induction that if $T$ is a tree with $k$ vertices 
	and $H$ a graph with $\delta(H) \geq k-1$, we have $T \subset H$. Hence $T \subset H \subset G$, thus $ex(n, T) < n(k-1)$. 
\end{proof}
\end{document}
