\documentclass[11pt]{report}

%\usepackage{ntheorem}
\usepackage{amssymb, graphicx, amsmath, amsthm}
\RequirePackage{graphicx}
\RequirePackage{hyperref}
\usepackage{commath}
\newcommand{\R}{\mathbb{R}}
\newcommand{\N}{\mathbb{N}}
\newcommand{\Q}{\mathbb{Q}}
\newcommand{\Z}{\mathbb{Z}}
\newcommand{\ran}{\operatorname{ran}}
\newcommand{\dom}{\operatorname{dom}}
\newcommand{\eps}{\varepsilon}
\newcommand{\ssd}{\bigtriangleup}
\newcommand{\pow}{\mathcal{P}}

\newtheoremstyle{break}%
{}{}%
{\itshape}{}%
{\bfseries}{}% % Note that final punctuation is omitted. 
{\newline}{}

\theoremstyle{break}
\newtheorem{thm}{Theorem}[section]
\newtheorem{defn}[thm]{Definition}
\newtheorem{lem}[thm]{Lemma}
\newtheorem{cor}[thm]{Corollary}
\newtheorem{prop}[thm]{Proposition}
\newtheorem{rem}[thm]{Remark}


% TODO: replace these with your information
% TODO: replace these with your information
\author{Evan Fox}
\title{MTH 316 Notes}

\begin{document}
\maketitle

\chapter{Cosets and Lagrange's theorem}

\section{Cosets}
\begin{defn}[Cosets] 
    Let $H$ be a subgroup of $G$ then for each $a \in G$ we define the left coset of $H$ as 
    \[aH = \{ah \in G | h \in H\} \]
\end{defn}
We will find this definition usefull in our study of groups. First we prove basic propertys of cosets. 

\begin{thm}[Properties of Cosets ] 

    \begin{enumerate}
        \setcounter{enumiv}{1}
        \item []
        \item $aH = H$ iff $a \in H$.
        \item $aH = bH$ iff $a \in bH$
        \item $o(aH) = o(H)$.
        \item $aH = bH$ or $aH \cap bH = \varnothing$.
        
    \end{enumerate}
\end{thm}

\begin{proof}
   (1) Assume $aH = H$, then since $e \in H$, $ae = a \in aH$ thus $a \in H$. Conversly assume that $a \in H$, then for $ah \in aH$ we 
   have by closure $ah \in H$. Simillarly if $h \in H$ then $ha^{-1} \in H$ and so $h \in aH$. For (2) The proof is similar. For (3) we 
   want to find a bijection from $aH$ to $H$. Define $\phi : H \to aH$ as $\phi(h) \mapsto ah$. Then one may easily show this is a bijection. 
   (4) Assume $aH \neq bH$, then for the sake of contradiction assume that $aH \cap bH \neq \varnothing$. We fix $x$ in the intersection 
   and note that $x = ah_1$ and $x = bh_2$. Then $b = ah_1 h_2^{-1} \in aH$, but then by (2) $aH = bH$; a contradiction.     
\end{proof}

Note that the last two parts of the theorem show that the cosets of a subgroup $H$ partition the group $G$, later we will find out 
when it makes sense to define an operation on these equivalence classes. 


\section{Lagrange's Theorem}
This is often called the A, B, Cs in finite group theory and one of the biggest results in MTH 316. 
\begin{thm}[Lagrange]
    Let $H \leq G$ then $o(H) | o(G)$. 
\end{thm}

\begin{proof}
    Let $H \leq G$. Using the result from the last section we already know that the cosets of $H$ partition $G$. Now consider 
    the set of left cosets of $H$ in $G$.
    \[\{g_1H, g_2H, \dots\, g_rH\} \]
    Since $a \in aH$ we know that each element of $G$ is in atleast one coset, then since the cosets are disjoint, 
     \[ro(H) = o(G)\]
    
\end{proof}


\end{document}