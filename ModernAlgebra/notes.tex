\documentclass[11pt,largemargins]{homework}
\usepackage{commath}
\newcommand{\R}{\mathbb{R}}
\newcommand{\N}{\mathbb{N}}
\newcommand{\Q}{\mathbb{Q}}
\newcommand{\Z}{\mathbb{Z}}
\newcommand{\ran}{\operatorname{ran}}
\newcommand{\dom}{\operatorname{dom}}
\newcommand{\eps}{\varepsilon}
\newcommand{\ssd}{\bigtriangleup}
\newcommand{\pow}{\mathcal{P}}

% TODO: replace these with your information
% TODO: replace these with your information
\newcommand{\hwname}{Evan Fox}
\newcommand{\hwemail}{efox20@uri.edu}
\newcommand{\hwtype}{Homework}
\newcommand{\hwnum}{2}
\newcommand{\hwclass}{MTH 316}
\newcommand{\hwlecture}{}
\newcommand{\hwsection}{}

\begin{document}
\maketitle

\question
Prove all subgroups of a cyclic group are cyclic 

\begin{proof}
Let $G$ be a group such that $<g> = G$ and let $H \leq G$. If $g \in H$ then it is clear $H = G$ and $H$ is cyclic. So 
Assume $g \notin H$. Then we write the elements of $H$ 

\[H = \{e, g^{\alpha_1}, g^{\alpha_2}, g^{\alpha_3}, ... \} \] 

To prove $H$ is cyclic we show there exists $i \in \N$ such that for all $j \in \N$ 
$\alpha_i | \alpha_j$. Let $S$ be the set of powers of $g$ in $H$ that is, $S = \{\alpha \in \N | g^\alpha \in H \}$. Then 
$S$ is obviously non-empty so fix $\alpha$ as the least element of $S$. Then assume there exists $\beta \in S$ such that $\alpha$ does not 
devide $\beta$, then using the divison algorithm we may write 
\[ \beta = \alpha q + r  \] 
For $0 < r < \alpha $ But this implies $g^{\beta} = g^{\alpha q}g^{r}$ so 

\[g^{-\alpha q}g^{\beta} = g^r \] 
so $r \in S$ but $r < \alpha $, a contradiction. Thus all positive exponents of $g$ in $H$ are multiples of $\alpha_i$, then since $g^{-x} \in H$ 
iff $g^x \in H$ it follows that $<\alpha> = H$. 

a $\neq$ b
    
\end{proof}

% Your content

\end{document}