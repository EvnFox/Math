\documentclass[11pt,largemargins]{homework}
\newcommand{\R}{\mathbb{R}}
\newcommand{\N}{\mathbb{N}}
\newcommand{\Q}{\mathbb{Q}}
\newcommand{\Z}{\mathbb{Z}}
\newcommand{\ran}{\operatorname{ran}}
\newcommand{\dom}{\operatorname{dom}}
\newcommand{\E}{\operatorname{E}}
\newcommand{\eps}{\varepsilon}
\newcommand{\ssd}{\bigtriangleup}
\newcommand{\pow}{\mathcal{P}}

% TODO: replace these with your information
\newcommand{\hwname}{Evan Fox}
\newcommand{\hwemail}{efox20@uri.edu}
\newcommand{\hwtype}{Quiz}
\newcommand{\hwnum}{4}
\newcommand{\hwclass}{MTH 451}
\newcommand{\hwlecture}{}
\newcommand{\hwsection}{}

\begin{document}
\maketitle
\question
\begin{alphaparts}
\questionpart
Applying the definition gives
\[\E(X) = \displaystyle \sum_{x \in X} xf(X) = 0 * 7/17 + 1 * 5/17 + 2*3/17 + 3*2/17 = 1 \]

\questionpart
Since $\E$ is linear
$\E(2x + 3) = 2\E(X) + 3 = 2 + 3 = 5$

\questionpart
We use the LOTUS. 
\[\E(X^2) = \displaystyle \sum_{x \in X} x^2 f(X) \]
\[= 0^2 * 7/17 + 1^2 * 5/17 + 2^2 *3/17 + 3^2 *2/17= 35/17 \]

\questionpart
By definiton $V(X) = E[x - E[X]^2]$, Then we have $V(X) = E(X^2) - E(X)^2 = 35/17 - 1 = 18/17$

\questionpart 
Since $V(aX + b) = a^2V(X)$ we have 
\[V(2X + 3) = 4V(X) = 4(18/17) = 72/12 \]

\end{alphaparts}


% Your content

\end{document}