\documentclass[11pt,largemargins]{homework}
\newcommand{\R}{\mathbb{R}}
\newcommand{\N}{\mathbb{N}}
\newcommand{\Q}{\mathbb{Q}}
\newcommand{\Z}{\mathbb{Z}}
\newcommand{\ran}{\operatorname{ran}}
\newcommand{\dom}{\operatorname{dom}}
%\newcommand{\d}{\operatorname{dx}}
\newcommand{\eps}{\varepsilon}
\newcommand{\ssd}{\bigtriangleup}
\newcommand{\pow}{\mathcal{P}}
\usepackage{qtree}
\usepackage{commath}
% TODO: replace these with your information
% TODO: replace these with your information
\newcommand{\hwname}{Evan Fox}
\newcommand{\hwemail}{efox20@uri.edu}
\newcommand{\hwtype}{Worksheet}
\newcommand{\hwnum}{3}
\newcommand{\hwclass}{MTH 451}
\newcommand{\hwlecture}{}
\newcommand{\hwsection}{}

\begin{document}
\maketitle
\question 
First I calculate $C(7,x) = \binom{7}{x}$ for $x \in \{1,2,3,4,5,6,7\}$.

\[C(7,0) = 1 \]

\[C(7,1) = 7\]

\[C(7,2) = 21\]

\[C(7,3) = 35\]

\[C(7, 4) = 35\]

\[C(7,5) = 21\]

\[C(7,6) = 7\]

\[C(7,7) =1\]

In order for $f$ to be a PMF each value of the function must be between 0 and 1 and the sum over 
all values of $f$ must equal 1. We note that $\sum_{i = 0}^7 C(7,i) = 128$, so setting $c = 1/128$ will have the desired effect.



\newpage 

\question
For $f$ to be a PMF it must have all outputs between 0 and 1 and it must sum to 1. Let $c = 3$, then it is clear that 
$f(x) < 1$ for all $x \in \N$. Let $S = \sum_{i = 1}^\infty 3(1/4)^i$. We must show that $S = 1$, note that 
$S$ is a geometric series with common ratio $1/4$ then since $1/4 < 1$ the series will converge to 3r/1 - r = (3/4)/(1 - 1/4) = 1. 
Thus for $C = 3$, $f$ is a probability mass function as desired.

\newpage

\question
\begin{alphaparts}
    \questionpart 
    We need $\int_{-\infty}^{\infty} f(x) \dif x= 1$. Note that there is only a contribution on the interval from $0$ to $4$. Then 

    \[ \int_{0}^{4} \frac{c}{\sqrt{x}} \dif x = c \int_{0}^{4} x^{-1/2} \dif x \]
    \[ = \tfrac{c}{2} x^{1/2} \Big|_0^4 = \frac{c}{2}2 = c \]

    Thus we select $c = 1$ so that the intergral equals $1$. 

    \questionpart 
    We want to find $P(X > 1) = \int_{1}^4 f(x) \dif x = \int_{1}^4 \frac{1}{\sqrt{x}} \dif x$ which is 
    $ \tfrac{1}{2} x^{1/2} \Big|_1^4$ so we get $ 2 - \tfrac{1}{2} = \frac{3}{2}$. 
% Your content
\end{alphaparts}

\newpage 

\question
\begin{alphaparts}
    \questionpart
    We have $E(x) = \int_{-\infty}^{\infty} xf(X) \dif x$, then plugging in $f(x)$ and noting there is only a non zero contribution 
    on the interval $(0, 2)$ we get 
    \[ \tfrac{1}{2} \int_{0}^{2} x^2 \dif x = \tfrac{1}{2} (\tfrac{1}{3}x^3 \big|_0^2) = \frac{4}{3} \]
    
    \questionpart
    To find $E(x^2)$ we use the law of the unconscious statistican, and get 
    \[ \tfrac{1}{2} \int_{0}^{2} x^3 \dif x = \tfrac{1}{2} (\tfrac{1}{4}x^4 \big|_0^2) = 2 \]

    \questionpart 
    We use the identity $V(X) = E(x^2) - E(x)^2$ to find the variance so 

    \[V(X) = 2 -(\frac{4}{3})^2 = \frac{2}{9} \] 

\end{alphaparts}

\newpage
\question

\begin{alphaparts}
    \questionpart
    $P(X \leq 0)$ is by definition equivalent to $F(X)$ evaluated at zero, so $P(X \leq 0) = 0.61$

    \questionpart
    We want $P(X = 1)$, since for $x \in [0, 3)$ we have $F(x) = 0.61$ we can see that $P(X = 1) = 0$, since $1$ cannot be in the support 
    of $X$. For instance if $P(X = 1) = c$ and $c \neq 0$ then we should have $P(X \leq 0) < P(\leq 1)$ but this is not the case. 

    \questionpart 
    We note  $P(-1 < X < 4) = P(-2 \leq X \leq 3)  = F(-2) - F(3) = 0.52$ 

    \questionpart 
    To find $P(X > 2)$ we observe that $P(X \leq 2) + P(X > 2) = 1$ so $P(X > 2) = 1 - F(2) = 0.39$.  
\end{alphaparts}


\end{document}