\documentclass[11pt,largemargins]{homework}
\newcommand{\R}{\mathbb{R}}
\newcommand{\N}{\mathbb{N}}
\newcommand{\Q}{\mathbb{Q}}
\newcommand{\Z}{\mathbb{Z}}
\newcommand{\ran}{\operatorname{ran}}
\newcommand{\dom}{\operatorname{dom}}
\newcommand{\eps}{\varepsilon}
\newcommand{\ssd}{\bigtriangleup}
\newcommand{\pow}{\mathcal{P}}
\usepackage{qtree}
\usepackage{commath}

% TODO: replace these with your information
% TODO: replace these with your information
\newcommand{\hwname}{Evan Fox}
\newcommand{\hwemail}{efox20@uri.edu}
\newcommand{\hwtype}{Quiz}
\newcommand{\hwnum}{8}
\newcommand{\hwclass}{MTH 451}
\newcommand{\hwlecture}{}
\newcommand{\hwsection}{}

\begin{document}
\maketitle

\question
In order to be independant we must have that for all $x, y$ in the support, the value of the joint density function must equal the 
product of the marginal distributions. Observe $f(2,1) = 0$ but $f_X(2)f_Y(1) = \frac{1}{4} * \frac{1}{4} \neq 0$, hence we have 
a counter example. 


\question 
\begin{alphaparts}
    \questionpart 
    We calculate the marginal distributions and check to see if their product is equalivalent to the density function. 

    \[f_X(x) = \frac{1}{96} \int_o^\infty e^{-(\tfrac{x}{8} + \tfrac{y}{12})} \dif y = \frac{1}{96}[-12e^{-(\tfrac{x}{8} + \tfrac{y}{12})}\big|_0^\infty] \] 
    \[ \frac{1}{96} 12e^{-\frac{x}{6}} \] 

    Now we do the same for $Y$ 

    \[f_Y(y) = \frac{1}{96}\int_0^\infty e^{-(\tfrac{x}{8} + \tfrac{y}{12})} \dif x =  \frac{1}{96}[-8e^{-(\tfrac{x}{8} + \tfrac{y}{12})}\big|_0^\infty] \] 
    \[\frac{1}{96}8e^{-\frac{y}{12}} \] 


    Then when we multiply these together we get 

    \[ f_X(x) \times f_Y(y) = \frac{1}{96}e^{-(\tfrac{x}{8} + \tfrac{y}{12})} \] 

    So yes they are independant. 

    \questionpart 
    we have that $\sigma_{XY} = E(XY) - E(X)E(Y)$ But since $X, Y$ are independant we can split $E(XY) = E(X)E(Y)$. Hence 
    $\sigma_{XY} = 0$. 



    
\end{alphaparts}

\question
Since we are assume $X$ and $Y$ to be independant we can get the joint probablity function by multiplying the marginal distributions. 
We find $P(X + Y = 3) = f(2, 1) + f(1, 2) = f_X(2)f_Y(1) + f_X(1)f_Y(2) = \frac{1}{6} + \frac{2}{6} = \frac{1}{2}$. 


\end{document}