\documentclass[11pt,largemargins]{homework}
\newcommand{\R}{\mathbb{R}}
\newcommand{\N}{\mathbb{N}}
\newcommand{\Q}{\mathbb{Q}}
\newcommand{\Z}{\mathbb{Z}}
\newcommand{\ran}{\operatorname{ran}}
\newcommand{\dom}{\operatorname{dom}}
\newcommand{\G}{\operatorname{Geom}}
\newcommand{\eps}{\varepsilon}
\newcommand{\ssd}{\bigtriangleup}
\newcommand{\pow}{\mathcal{P}}
\usepackage{qtree}
\usepackage{commath}

% TODO: replace these with your information
% TODO: replace these with your information
\newcommand{\hwname}{Evan Fox}
\newcommand{\hwemail}{efox20@uri.edu}
\newcommand{\hwtype}{Quiz}
\newcommand{\hwnum}{14}
\newcommand{\hwclass}{MTH 451}
\newcommand{\hwlecture}{}
\newcommand{\hwsection}{}

\begin{document}
\maketitle

\question
\begin{alphaparts}
    \questionpart
    Find $P(X \geq k)$. 

    Let $X \sim \G(p)$. Then the PMF is $f(x) = p(1 - p)^x$ for $x = 0, 1, 2, 3, ...$. Then we have the CMF given by 
    $F(X) = \sum_0^x p(1 - p)^i$. It is clear that 

    \[P(X \geq k) = 1 - F(x) = 1 - \sum_{i = 0}^{k - 1} p(1 - p)^i = \sum_{i = k}^\infty p(1 - p)^i =   \]
    Now since our index starts at $k$ each summand will contain a copy of $(1 - p)^k$ which we can factor out giving,
    \[ = (1 - p)^k \sum_{i = k}^\infty p(1 - p)^{i - k} = (1 - p)^k \sum_{i = 0}^\infty p(1 - p)^{i} = (1 - p)^k\]
    Where the last equality follows since $\sum_{i = 0}^\infty p(1 - p)^{i}$ is just a sum of the PDF over the support. So we have 

    \[P(X \geq k) = (1 - p)^k\]


    \questionpart 
    \begin{proof}
    We can prove this with a string of equalitys(recall def of conditional probability), Observe
    \[P(X \geq m + n| X \geq n) = \frac{P(X \geq m + n \& X \geq n)}{P(X \geq n)} = \frac{P(X \geq m + n)}{P(X \geq n)}\]
    where the third equality follows since if the random variable $X$ is greater than $m + n$ it is necessarly greater than $n$. 
    Now using the formula derived in part (a) we get 
    \[\frac{P(X \geq m + n)}{P(X \geq n)} = \frac{(1 - p)^{m + n}}{(1 - p)^n} = (1 - p)^m = P(X \geq m)\]
    and this completes the proof. 
    \end{proof}
    
\end{alphaparts}

\question
The probility of 3 coins being heads is $.5^3 = 1/8$. Now define a random variable $X \sim \G(1/8)$ as the number of attempts until all 
three are heads. We then define the random variable $Y = 70 - 10(X - 1)$ as the amount of winnings. We want to find the expectation of 
$Y$. 
\[E(Y) = E(70 - 10(X - 1)) = 60 - 10E(X) \]
by linearity of expectation. Then from class we have $E(X) = (1 - p)/p = \frac{7/8 }{ 1/8 } = 7$. so 
\[E(Y) = 60 - 10(7) = -10\]
Since the expectation is negitive you should not take the bet. 


\end{document}