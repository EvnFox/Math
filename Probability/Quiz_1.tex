\documentclass[11pt,largemargins]{homework}
\newcommand{\R}{\mathbb{R}}
\newcommand{\N}{\mathbb{N}}
\newcommand{\Q}{\mathbb{Q}}
\newcommand{\Z}{\mathbb{Z}}
\newcommand{\ran}{\operatorname{ran}}
\newcommand{\dom}{\operatorname{dom}}
\newcommand{\eps}{\varepsilon}
\newcommand{\ssd}{\bigtriangleup}
\newcommand{\pow}{\mathcal{P}}

% TODO: replace these with your information
% TODO: replace these with your information
\newcommand{\hwname}{Evan Fox}
\newcommand{\hwemail}{efox20@uri.edu}
\newcommand{\hwtype}{Quiz}
\newcommand{\hwnum}{1}
\newcommand{\hwclass}{MTH 451}
\newcommand{\hwlecture}{}
\newcommand{\hwsection}{}

\begin{document}
\maketitle

\question
\begin{alphaparts}
    \questionpart
    These values are not permissible because $P(D) = -0.2$ violates axiom 1 which requires that all probabliltys are non-negative. 

    \questionpart 
    We first see that since all values are non-negative, axiom 1 is not violated. Similarly axiom 2 is not violated since the sum 
    of all events equals 1. There is no contradiction with the axioms and so this assignment is allowable.
    \questionpart
    Note by axiom two we must have $P(S) = 1$, then since A, B, C, and D are disjoint, we may apply axiom 3 to get 
    \[ P(S) = P(A \cup B \cup C \cup D) = + P(A) + P(B) + P(C) + P(D) = \frac{18}{19} \neq 1 \] 
    Hence by axioms 2 and 3, this probability distribution is impossible. 
\end{alphaparts}

\question 
\begin{alphaparts}
    \questionpart
    We use $P(A \cup B) = P(A) + P(B) - P(A \cap B)$. 
    
    \begin{proof}
    Let $A, B \subseteq S$ and let $C = A \cap B$. Then $P(A \cup B) = P(A - C \cup B) = P(A  - C) + P(B)$.
    Now consider 
    \[P(A^c \cup C) + P(B^c) = 1 - P(A - C)  + 1 - P(B)\]
    \[P(A^c) + P(C) + P(B^c) = 1 - P(A - C) + 1 - P(B)\]
    \[P(A - C) + P(B) = 1 - P(A^c) + 1 - P(B^c) - P(C)\]
    \[P(A - C) + P(B) = P(A) + P(B) - P(C) \] 
    As desired. 
    \end{proof}
    
    
    Since all of these values are given we simply plug in to get 

    \[ P(A \cup B) = 0.3 + 0.5 - 0.25 = .55 \] 

    \questionpart
    We use $P(B/A) = P(B) - P(A \cap B)$

    \begin{proof}

    By definition of set difference 
    $P(B/A) = P(B/( A \cap B))$. Now let $C = A \cap B$, then observe $B^c \cap C = \varnothing$ 
    since $C \subseteq B$, so our use of axiom 3 in the next step is permissable. We compute 
    \[ P(B/C) = 1 - P(B^c \cup C) = 1 - (P(B^c) + P(C)) \] 
    \[ 1 - ((1 - P(B)) + P(C)) = P(B) - P(C)\]
    as desired.
    \end{proof}
    Thus $P(B/C) = P(B/A) = P(B) - P(A \cap B) = 0.5 - 0.25 = 0.25$.

\end{alphaparts}
% Your content

\end{document}