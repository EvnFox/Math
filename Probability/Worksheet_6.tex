\documentclass[11pt,largemargins]{homework}
\newcommand{\R}{\mathbb{R}}
\newcommand{\N}{\mathbb{N}}
\newcommand{\Q}{\mathbb{Q}}
\newcommand{\Z}{\mathbb{Z}}
\newcommand{\ran}{\operatorname{ran}}
\newcommand{\dom}{\operatorname{dom}}
\newcommand{\eps}{\varepsilon}
\newcommand{\ssd}{\bigtriangleup}
\newcommand{\pow}{\mathcal{P}}
\usepackage{qtree}
\usepackage{commath}

% TODO: replace these with your information
% TODO: replace these with your information
\newcommand{\hwname}{Evan Fox}
\newcommand{\hwemail}{efox20@uri.edu}
\newcommand{\hwtype}{Worksheet}
\newcommand{\hwnum}{6}
\newcommand{\hwclass}{MTH 451}
\newcommand{\hwlecture}{}
\newcommand{\hwsection}{}

\begin{document}
\maketitle

\question
We have 
\[M_X(t) = \sum_{x=1}^{\infty}2(\frac{1}{3}^xe^{tx}) = 2\sum_{x = 1}^{\infty} \left(\frac{e^t}{3}^x \right) = \frac{-2e^t}{e^t-3} \] 

This is vaild as long as $t \neq ln(3)$. 

To find the mean we compute 

\[E(X = M_X^\prime(0) = \left(\frac{-2e^t}{e^t - 3} \right)^\prime = \frac{6e^t}{(e^t - 3)^2}\big|_{t=0} = \frac{6}{4}\]

Then to find the second moment we take the second derivitive of the MGF

\[E(X^2) = M_X^{(2)}(0) = \left( \frac{6e^t}{(e^t - 3)^2} \right)^\prime\big|_{t=0} = 3  \]

\newpage

\question
\begin{alphaparts}
    \questionpart 
    To find $c$ we sum over the natural numbers union $0$ and choose $c$ such that the summation is equal to $1$. 

    \[\sum_{x = 0}^\infty \frac{c}{x!} = c\sum_{x = 0}^\infty \frac{1}{x!} = c(1 + 1 + \frac{1}{2!}+...) = ce = 1\]
    hence $c = e^{-1}$. 

    \questionpart 
    \[M_X(t) = e^{-1}\sum e^{xt}\frac{1}{x!} = e^{-1} \times (1 + e^t + \frac{e^{2t}}{2!} + \frac{e^{3t}}{3!}+...) \]

    which gives

    \[e^{-1} e^{e^t} = e^{e^t - 1}\]
    
\end{alphaparts}

\newpage 

\question 
We use the given formulas. 

\[E(X) = [ln(e^{4(e^t - 1)})]^\prime \big|_{t=0} = 4e^0 = 4\]

Then 

\[E(X^2) = [ln(e^{4(e^t - 1)})]^{\prime \prime}\big|_{t=0} = 4e^0 = 4\]

Then using the fact that varience is given by second moment subtract the square of the first moment we get 

\[V(X) = 4 - 4^2 = -12 \]

\newpage
We have that 

\[M_Z(t) =E(e^{tZ}) = E(e^{t(\tfrac{x - 3}{4})}) = e^{-3/4}E(e^{\tfrac{1}{4}tX}) =e^{-3/4}M_X(\tfrac{1}{4}t) \]

so then 

\[M_Z(t) = e^{-3/4}((e^{3/4t+2t^2})) = (e^{3/4t+2t^2 - \tfrac{3}{4}})\]

Then 

\[E(Z) = \left( e^{3/4(t-1) + 2t^2 })\right)^\prime \big|_{t = 0} = (3/4 + 4t)e^{3/4(t-1) + 2t^2} = 3/4e^{-3/4}\] 

and 

\[E(Z^2) = \left( (3/4 + 4t)e^{3/4(t-1) + 2t^2} ) \right)^\prime \big|_{t = 0} = \] 
\[=(3/4)e^{3/4(t-1) + 2t^2} + (3/4 + 4t)^2e^{3/4(t-1) + 2t^2} = (3/4)e^{-3/4} + (3/4)^2e^{-3/4}\]

Hence 

\[v(Z) = (3/4)e^{-3/4} + (3/4)^2e^{-3/4} - (3/4e^{-3/4})^2\]

\end{document}