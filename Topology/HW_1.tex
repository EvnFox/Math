\documentclass[11pt,largemargins]{homework}
\usepackage{commath}
\newcommand{\R}{\mathbb{R}}
\newcommand{\N}{\mathbb{N}}
\newcommand{\Q}{\mathbb{Q}}
\newcommand{\Z}{\mathbb{Z}}
\newcommand{\T}{\mathfrak{T}}
\newcommand{\B}{\mathfrak{B}}
\newcommand{\C}{\mathfrak{C}}
\newcommand{\ran}{\operatorname{ran}}
\newcommand{\dom}{\operatorname{dom}}
\newcommand{\eps}{\varepsilon}
\newcommand{\ssd}{\bigtriangleup}
\newcommand{\pow}{\mathcal{P}}

% TODO: replace these with your information
% TODO: replace these with your information
\newcommand{\hwname}{Evan Fox}
\newcommand{\hwemail}{efox20@uri.edu}
\newcommand{\hwtype}{}
\newcommand{\hwnum}{}
\newcommand{\hwclass}{MTH 525: Topology}
\newcommand{\hwlecture}{}
\newcommand{\hwsection}{}



\begin{document}
\maketitle

\question 
TYhis is somthing Which of the 9 topologies in the example given in the chapter are comparable. Numbering from top left to right, 

\begin{center}
    \begin{tabular}{ |c|c|c|c| } 
     \hline
     Number & Finer than & Corser than& Not comparable with \\ 
     \hline
     1 & none & 1-9 & none \\ 
     \hline
     2 & 8,9 & 1,7 & 3,4,5,6 \\ 
     \hline
     3 &  1,4,7 & 6,9 & 2,5,8 \\ 
     \hline
     4 & 1 & 3,8,9 & 2,5,6,7 \\ 
     \hline
     5 & 1 & 9  & 2,3,4,6,7,8 \\ 
     \hline
     6 & 1,3,4,7 & 9  & 2,5 ,8\\ 
     \hline
     7 & 1 & 2,3,6,8,9  & 4,5 \\ 
     \hline
     8 & 1,2,4,7 & 9  & 3,5,6 \\ 
     \hline
     9 & 1-9 & none  & none \\
     \hline

    \end{tabular}
    \end{center}

\question
Show that $\R_K$ and $\R_l$ are not comparable. 

\begin{proof}
    Let $x = 0$, and condisder $X = (-1, 1) - K \in \R_K$. For any $[a ,b) \subset \R_l$, with $0 \in [a, b)$, then $0 < b$ and we may use the archimedian 
    property of $\R$ to fix $N \in \N$ such that $n \geq N$ implies $\frac{1}{n} < b$, so $\frac{1}{n} \in [a, b)$ since $\frac{1}{n} \notin X$, there is no 
    basis element of $\R_l$ containg zero that is a subset of $X$. Now consider the basis element of $\R_l$, $[0, b)$. We must show there is no 
    basis element of $\R_K$ containg $0$ that is a subset of $[0, b)$. In order for $0$ to be in a basis element of $\R_K$.
    we must have $a < 0$, but then $a \notin [0, b)$. 
\end{proof}

\question 
Show that $\B = \{(a, b)| a, b \in \Q\}$ generates the standard topology on $\R$. Show that the this statement if false with the lower limit topology.

\begin{proof}
    We have by earlier result that $\B$ is a basis if for every open set $U$ and for all $x \in U$ there exists an element of $\B$ 
    containg $x$ that is a subset of $U$. So let $U \subset R$ be open, for $x \in U$, $x$ must appear in some basis element, say $x \in (a, b)$ for $a, b \in \R$. Now by the density of the rationals in $\R$, there exists 
    $s, t \in \Q$ such that 
    \[a < s < x\]
    and 
    \[x < t < b\]
    Thus $x \in (s, t) \subset (a, b) \subset U$; it follows that $\B$ generates the topology on $\R$. 

    Now consider $\R_l$. We have that $[e, 3)$ is an open subset of $\R_l$, but there is no element of $\{[a, b)| a, b \in \Q\}$ containg $e$ which is a subset 
    of $[e, 3)$ since we must choose $a$ as a rational if $e < a$ then we have a contradiction and if $e > a$ then $[a, b)$ is not a subset of $[e, 3)$. Hence, 
    $\{[a, b)| a, b \in \Q\}$ does not form a basis for the lower limit topology on $\R$.  
\end{proof}


\newpage
\question
If $A$ is a basis for the topology on $X$ then the topology generated by $A$ equals the intersection of all topolgies containg $A$ 


\begin{proof}
    Let $A$ be a basis for the topolgy on $X$, $\T$. Let $\{ \mathfrak{D}_{\alpha} \}$ be the collection of all topologies containg $A$. 
    We want to show that $\T = \bigcap \mathfrak{D}_\alpha$. If $U \in \T$ then $U$ is equal to a union of elements of $A$, since
    $A$ is contained in each $\mathfrak{D}_\alpha$ so is $U$, since each $\mathfrak{D}_\alpha$ is a topology. Now if $U \in \bigcap \mathfrak{D}_\alpha$, 
    then it is in every topology containing $A$, since $\T$ is one such topology we have $U \in \T$.  In the case that $A$ is a subbasis generating 
    the topology $\T_A$. If $U \in \T_A$, then $U$ is a union of finite intersections of elements of $A$, then since $A \in \mathfrak{D}_\alpha$ it follows 
    from the definiton of a topology that $U$ will be in each $\mathfrak{D}_\alpha$ hence it is in the intersection. The second part of the argument is a repeat of the above. 

\end{proof}



\question 
is the finite complement topology true if we replace finite with infinite? 

Ans: No, under $\Z$ we have a collection of open sets 
\[ \mathfrak{U} = \{ ... , [-4, -3], [-2, -1], [1, 2], [3, 4], ... \}\] 
whose union equals $(-\infty, -1] \cup [1, \infty)$. Then it is clear the complement is not infinite. 

\question 
Let $\{\T_\alpha\}$ be a collection of topologies on $X$ prove that the union and intersection of this set are topologies on $X$. 


\begin{proof}
    It is clear that $X, \varnothing \in \bigcap \{\T_\alpha\}$. Now suppose that $\{U_\beta\}_{\beta \in J} \subset \bigcap \{\T_\alpha\}$. 
    Since each $U_\beta$ is in the intersection, $\{U_\beta\} \subset \T_\alpha$ for all $\alpha$. Then since $\T_\alpha$ is a topology, 
    the union is in each $\T_\alpha$ and thus is contained in the intersection. Now let $U_1, ..., U_n$ be a finite collection of elements in 
    $\bigcap \{\T_\alpha\}$. Then each $U_i$ is in all $\T_\alpha$ and again by the definition of topology the finite intersection over $U_i$ is in each 
    $\T_\alpha$ and thus contained in the intersection. 

    Now in general the union of two topologies need not be a topology. How ever, the union of topologies on $X$ will form a subbasis for a topology on $X$. 
    and this topology is the smallest one containg each of the topologies in the union. To see that the union is not always a topology just consider a case where 
    the topologys are not comparable. The fact the the union forms a subasis is clear since the elements of any one topology for a subbais ($X \in \T$ makes $T$ a subbasis). 
    It is also clear that the topology generated by the subbasis contains all topologys in the union. Now suppose there exists a topology $\T$ such that $\T_\alpha \subset \T$ for all $\alpha$. 
    Then if $U$ is open is the topology genererated by the subbasis $\cup \{\T_\alpha\}$, $U$ is a union of finite intersections of elements of $\cup \{\T_\alpha\}$, hence 
    by definiton of topology $U$ is open in $\T$. 
     
\end{proof}
    
\end{document}
