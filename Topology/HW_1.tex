\documentclass[11pt,largemargins]{homework}
\usepackage{commath}
\newcommand{\R}{\mathbb{R}}
\newcommand{\N}{\mathbb{N}}
\newcommand{\Q}{\mathbb{Q}}
\newcommand{\Z}{\mathbb{Z}}
\newcommand{\T}{\mathfrak{T}}
\newcommand{\B}{\mathfrak{B}}
\newcommand{\C}{\mathfrak{C}}
\newcommand{\ran}{\operatorname{ran}}
\newcommand{\dom}{\operatorname{dom}}
\newcommand{\eps}{\varepsilon}
\newcommand{\ssd}{\bigtriangleup}
\newcommand{\pow}{\mathcal{P}}

% TODO: replace these with your information
% TODO: replace these with your information
\newcommand{\hwname}{Evan Fox}
\newcommand{\hwemail}{efox20@uri.edu}
\newcommand{\hwtype}{}
\newcommand{\hwnum}{}
\newcommand{\hwclass}{MTH 525: Topology}
\newcommand{\hwlecture}{}
\newcommand{\hwsection}{}



\begin{document}
\maketitle

\question 
TYhis is somthing Which of the 9 topologies in the example given in the chapter are comparable. Numbering from top left to right, 

\begin{center}
    \begin{tabular}{ |c|c|c|c| } 
     \hline
     Number & Finer than & Corser than& Not comparable with \\ 
     \hline
     1 & none & 1-9 & none \\ 
     \hline
     2 & 8,9 & 1,7 & 3,4,5,6 \\ 
     \hline
     3 &  1,4,7 & 6,9 & 2,5,8 \\ 
     \hline
     4 & 1 & 3,8,9 & 2,5,6,7 \\ 
     \hline
     5 & 1 & 9  & 2,3,4,6,7,8 \\ 
     \hline
     6 & 1,3,4,7 & 9  & 2,5 ,8\\ 
     \hline
     7 & 1 & 2,3,6,8,9  & 4,5 \\ 
     \hline
     8 & 1,2,4,7 & 9  & 3,5,6 \\ 
     \hline
     9 & 1-9 & none  & none \\
     \hline

    \end{tabular}
    \end{center}

\question
Show that $\R_K$ and $\R_l$ are not comparable. 

\begin{proof}
    Let $x = 0$, and condisder $X = (-1, 1) - K \in \R_K$. For any $[a ,b) \subset \R_l$, with $0 \in [a, b)$, then $0 < b$ and we may use the archimedian 
    property of $\R$ to fix $N \in \N$ such that $n \geq N$ implies $\frac{1}{n} < b$, so $\frac{1}{n} \in [a, b)$ since $\frac{1}{n} \notin X$, there is no 
    basis element of $\R_l$ containg zero that is a subset of $X$. Now consider the basis element of $\R_l$, $[0, b)$. We must show there is no 
    basis element of $\R_K$ containg $0$ that is a subset of $[0, b)$. In order for $0$ to be in a basis element of $\R_K$.
    we must have $a < 0$, but then $a \notin [0, b)$. 
\end{proof}

\question 
Show that $\B = \{(a, b)| a, b \in \Q\}$ generates the standard topology on $\R$. Show that the this statement if false with the lower limit topology.

\begin{proof}
    We have by earlier result that $\B$ is a basis if for every open set $U$ and for all $x \in U$ there exists an element of $\B$ 
    containg $x$ that is a subset of $U$. So let $U \subset R$ be open, for $x \in U$, $x$ must appear in some basis element, say $x \in (a, b)$ for $a, b \in \R$. Now by the density of the rationals in $\R$, there exists 
    $s, t \in \Q$ such that 
    \[a < s < x\]
    and 
    \[x < t < b\]
    Thus $x \in (s, t) \subset (a, b) \subset U$; it follows that $\B$ generates the topology on $\R$. 

    Now consider $\R_l$. We have that $[e, 3)$ is an open subset of $\R_l$, but there is no element of $\{[a, b)| a, b \in \Q\}$ containg $e$ which is a subset 
    of $[e, 3)$ since we must choose $a$ as a rational if $e < a$ then we have a contradiction and if $e > a$ then $[a, b)$ is not a subset of $[e, 3)$. Hence, 
    $\{[a, b)| a, b \in \Q\}$ does not form a basis for the lower limit topology on $\R$.  
\end{proof}


\newpage
\question
If $A$ is a basis for the topology on $X$ then the topology generated by $A$ equals the intersection of all topolgies containg $A$ 


\begin{proof}
    Let $A$ be a basis for the topolgy on $X$, $\T$. Let $\{ \mathfrak{D}_{\alpha} \}$ be the collection of all topologies containg $A$. 
    We want to show that $\T = \bigcap \mathfrak{D}_\alpha$. If $U \in \T$ then $U$ is equal to a union of elements of $A$, since
    $A$ is contained in each $\mathfrak{D}_\alpha$ so is $U$, since each $\mathfrak{D}_\alpha$ is a topology. Now if $U \in \bigcap \mathfrak{D}_\alpha$, 
    then it is in every topology containing $A$, since $\T$ is one such topology we have $U \in \T$.  In the case that $A$ is a subbasis generating 
    the topology $\T_A$. If $U \in \T_A$, then $U$ is a union of finite intersections of elements of $A$, then since $A \in \mathfrak{D}_\alpha$ it follows 
    from the definiton of a topology that $U$ will be in each $\mathfrak{D}_\alpha$ hence it is in the intersection. The second part of the argument is a repeat of the above. 

\end{proof}



\question 
is the $\T_\infty = \{U |\, X \setminus U \text{ is infinite or empty or all of } X\}$ a topology?

Ans: No, under $\Z$ we have a collection of open sets 
\[ \mathfrak{U} = \{ ... , [-4, -3], [-2, -1], [1, 2], [3, 4], ... \}\] 
whose union equals $(-\infty, -1] \cup [1, \infty)$. Then it is clear the complement is not infinite. 

\question 

\begin{alphaparts}
\questionpart
Let $\{\T_\alpha\}_{\alpha \in J}$ be a collection of topologies on $X$ prove that the intersection is a topology. Is the union of this set a topology? 


\begin{proof}
    It is clear that $X, \varnothing \in \bigcap \{\T_\alpha\}$. Now suppose that $\{U_\beta\}_{\beta \in J} \subset \bigcap \{\T_\alpha\}$. 
    Since each $U_\beta$ is in the intersection, $\{U_\beta\} \subset \T_\alpha$ for all $\alpha$. Then since $\T_\alpha$ is a topology, 
    the union is in each $\T_\alpha$ and thus is contained in the intersection. Now let $U_1, ..., U_n$ be a finite collection of elements in 
    $\bigcap \{\T_\alpha\}$. Then each $U_i$ is in all $\T_\alpha$ and again by the definition of topology the finite intersection over $U_i$ is in each 
    $\T_\alpha$ and thus contained in the intersection. 

    Now in general the union need not be a topology, if the set $\{T_\alpha\}_{\alpha \in J}$ ordered under set inclusion contains a maximal element, then it is clear that the union will be a topology. 
    But if this is not the case, then it is not to hard to create an example where $U_1 \in \T_1 \setminus \T_2$ and $U_2 \in \T_2 \setminus \T_1$ where $U_1 \cap U_2$ is not in the union. However, 
    the union does form a subbasis for a topology. This follows since any $\T_\alpha$ already meets the condiditions for a subbasis.
    
\end{proof}

\questionpart 
Let $\{\T_\alpha\}_{\alpha \in J} $ Show that there is a unique largest topology on $X$ contianed in each $\T_\alpha$ and unique smallest topology containg all the collections $\T_\alpha$. 

\begin{proof}
    The largest topology which is contained in each $\T_\alpha$ is $\T_l = \cap_{\alpha \in J } \T_\alpha$. We have already established that this is a topology and it is clear that it must be contained in each $\T_\alpha$. Now let $\C$ be a topology 
    contained in each $\T_\alpha$, then it must be contained in the intersection so $\C \subset \T_l$. Now given $\T^\prime $ and $\T$ both with the property that they are the largest topology contained in all $\T_\alpha$, we have that 
    $T^\prime \subset \T$ since $\T^\prime$ is a topology contained in all $\T_\alpha$ and $\T$ is an upperbound, but this argument can be reveresed to show $\T^\prime = \T$. 

    While the union over a familiy of topologys need not be a topology, it does form a subbasis, then the topology that this subbasis generates will the smallest topology which contains all $\T_\alpha$. The fact that it forms a subbasis was answered in part (a) and 
    it is clear that the topology it generates will contain each $\T_\alpha$. Now suppose $\C$ is a topology which contains all $\T_\alpha$, and let $U$ be in the topology generated by the subbasis $\bigcup_{\alpha \in J} \T_\alpha$, then $U$ can be written as 
    arbitrary unions of finite intersections of elements from (several different) $\T_\alpha$, then since all $\T_\alpha$ are contained in $\C$, it follows that $U \in \C$. Thus the topology generated by the subbasis is minimal. As above the uniquness of this topology is 
    clear since if there was another smallest topology which contained all $\T_\alpha$ It must be contained in the topology generated by the subbasis given the above argument, but then since we are assuming this topology is the smallest it also must be contain the topology generated by 
    the subbasis, hence they are equal. 
\end{proof}

\questionpart
The largest will be the intersection $\T_1 \cap \T_2 = \{\varnothing, X, \{a\}\}$

Now the smallest will be given by the basis formed with finite intersections of elements of $\T_1 \cup \T_2 = \{\varnothing, X, \{a\}, \{a, b\}, \{b, c \}\}$. 
So the topology is $ \{\varnothing, X, \{a\}, \{b\}, \{a, b\}, \{b, c\}\}$


\question 
We start with $\T_1$, by definition it is contained in $\T_2$, then given a basis element of $\T_1$, $(a, b)$ we may fix $z \in (a, b)$ and take $b^\prime = z + \frac{b-z}{2}$ so that $z \in (a, b^\prime]$, hence $T_1$ is corser than $\T_4$. Then noting that $\T_5$ forms a subbasis for the 
standard topology on $\R$ we have that $\T_1$ is also contained in $\T_5$ and that $\T_5 $ is contained in $\T_1$. Now any element of $\T_3$ will be in $\T_1$ since any element of $U \in \T_3$ has a finite complement, and since finite sets are closed in $\T_1$, we must have $U \in \T_1$. 

Now we already know $\T_2$ contains $\T_1$ and so it also will contain $\T_3$ and $\T_5$. We prove that it is not comparable with $\T_4$, Now $\T_2 \subsetneq \T_4$ is clear, and considering $(0, 1]$ we see $1 \in (0, 1]$ and that there is no open set in $\T_2$ contained in $(0, 1]$ which contains $1$. 
hence the topologies are not comparable. 

We have already seen $\T_3 \subset \T_1 = \T_5 \subset \T_4$ and $\T_3 \subset \T_1 \subset \T_2$ also holds. 

Now $\T_4 \supset \T_1 = \T_5 \supset \T_3$ has been established, as well as $\T_4 \supset \T_1 \supset T_2$. 


Then again $\T_5 = \T_1$ and we have already completed $\T_1$. To see that this is indeed as subbasis as I claim, note $(a, b) = (-\infty, b) \cap (-\infty, a)
\end{alphaparts}
\end{document}
