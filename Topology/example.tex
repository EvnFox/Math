\documentclass{article}


\begin{document}

\begin{abstract} 
	This is me writing my abstract for an article. 
\end{abstract}
This is a document.
\begin{equation}
	f(x) = x^2
\end{equation}

In this document I can write as much as I want. It may be of use yet to use this nvim to write my SOP for possible REU's if I choose to apply. So far I am finding this to be very user friendly, after of course navigating the initial 
hurdles. One must see that this can be a very efficent way to write Latex and produce nice looking mathematical documents. I also am finding the new keyboard placement to be quite usable. Wired that I am getting some slowdowns. 
I will have to figure out how I can enable/diable this add-ons and disable this ltex linter. I do not enjoy the constant distraction it is creating on screeen. 

Let's have a look at how the paragraph indentation works. I like the continuous complier which will compile when ever you save the file. I wonder if there is a way for me to navigate from spelling error to spelling error. I also wonder if there iany sort of an auto-correct feature that can easily replace my mispellings. I wish that hiting the end of the screen would automatically cause a newline character. It would also seem that the only way to get ride of the synctex is to do a 
complete and total clean, which will also delete the associated PDF file, which we obviously want to keep. Future things I wish to add to my current NVIM setup include a more customized VimTex expirence as well as better git intergration. I would also like to figure out what 'telescope' is and 'fuzzy-finding', two things I currently don't understand. TMR is Christmas, so I gues I should get some rest. 
\end{document} 
