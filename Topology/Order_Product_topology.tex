
\begin{chapter}{Order and Product Topology}
    \section{Order Topology}
    
    \begin{defn}
        Let $X$ be a set, we define the order topology on $X$ as the topology generated by the basis $\B$ such that 
        $\B$ contains all elements of the form 
        
        \begin{enumerate}
            \item $(a, b)$ for $a, b \in X$. 
            \item $[a_0, b)$ where $a_0 $ is the minimal element of $X$. 
            \item $(a, b_0]$ where $b_0$ is the maximal element of $X$.
        \end{enumerate}
        If $X$ has no maximal or minimal elements, then $\B$ consists only of elements of the first type. 
    \end{defn}

    Now we must of course prove that this choice of $\B$ does indeed form the basis of a topology, but it is clear since 
    the intersection of any of these sets yeilds another set of the same type. There are several cases to check. 
    
    \begin{ex}
        Consider the order topology on $\R$, since $\R$ has no maximal or minimal elements the order topology is generated by 
        $\B = \{(a, b) | a, b \in \R\}$. Then it is clear that this coencides with the standard topology on $\R$
    \end{ex}

    
    \begin{ex}
        Consider the order topology on $\Z_{+}$ given the usual order. Then the order topology is equivalent to the discrete topology. 
    \end{ex}

    
    \begin{ex}
        Consider $\R \times \R$ equiped with the dictonary order, then the basis elements are of the form 
        $(a, b) \times (c, d)$ where $a < c$ or $(a = c) \wedge (b < d)$. 
    \end{ex}

    \section{Product Topology}
        Given two sets $A$ and $B$ we may want to define a topology on the cartisian product $A \times B$. 
    The cartisian product comes equiped with two functions $\pi_A(x): A \times B \to A$ and $\pi_B(x): A \times B \to B$ called 
    the projections of $A \times B$. 


    
\end{chapter}