
\begin{chapter}{Topology and Basis}
    \section{Topology}

    
    \begin{defn}
        Let $X$ be a set. A $\emph{topology}$ on $X$ is a collection of subsets 
        $T$ such that 
        
        \begin{enumerate}
            \item $X$, $\varnothing \in \T$. 
            \item Closed under arbitrary unions.
            \item Closed under finite intersections. 
        \end{enumerate} 
    \end{defn}

    On a given set there may be many diffrerent topologies. That can be defined on that set. 
    Let $X = \{1, 2, 3, 4, 5\}$, we will write down several examples of topologies on $X$. 
    
    \begin{enumerate}
        \item $\T = \{\varnothing, X\}$
        \item $\T = \mathfrak{p}(X)$
        \item $\T = \{\varnothing, X, \{1\}\}$
        \item $\T = \{\varnothing, X, \{1\}, \{2\}, \{1,2\}\}$
    \end{enumerate}

    are all topologies on $X$. The first is called the trivial topology or the indescrete topology and the second is the descrete topology. We 
    will now give an example of another topology, the finite complement topology. 

    
    \begin{defn}{Finite complement Topology}
        Let $X$ be a set and let $\T$ consist of all subsets $U \subset X$ such that the complemnet of $U$ in $X$ is finite 
        or $X$. 
    \end{defn}

    
    \begin{prop}
        The finite complement topology is a topology
    \end{prop}

    
    \begin{proof}
        We must show all three conditions are true, first we show that $X$ and $\varnothing$ are in $\T$. Note, $X-\varnothing = X$ and $X - X = \varnothing$ and both of 
        these satisfy are conditions on $\T$. Now we show $\T$ is closed under arbitrary unions. Let $\{U_\alpha\}_{\alpha \in J} \subset \T$ and we want to show that 
        $(\bigcup_{\alpha} U_\alpha)^c \in \T$. We have by DeMorgans laws 
        \[(\bigcup_{\alpha \in J} U_\alpha)^c = \bigcap_{\alpha \in J} U_\alpha^c \]
        and since each $U_i \in \T$ we now that each $U_i^c$ is finite, thus since the intersection of finite sets are finite we are done. Now we must show closure under finite 
        intersections, so let $\{U_1, \dotsb, U_n \}$ be a subset of $\T$. Then we have 
        \[(\bigcap U_i)^c = \bigcup U_i^c \]
        and since each $U_i^c$ is finite, and we have a finite number of sets to union, the result is finite. Hence we have showed the finite complement topology is indeed a 
        topology. 
    \end{proof}

    We could replace the finite condidtion with countable and we would still have a topology since the union of countable sets is again 
    countable. 

    Given two topologies on a set we can also compare them. 
    
    \begin{defn}
        Let $X$ be a set and let $\T$ and $\T^\prime$ be topologies on $X$. We say $\T$ is finer than $\T^\prime$ if 
        $\T^\prime \subset \T$. We say $\T$ is corser in the reverse situation. 
    \end{defn}

    \section{Basis for a Toplogy}
    
    \begin{defn}{Basis}
        Let $X$ be a set, we say $\B$ is a $\emph{Basis}$ if 
        
        \begin{enumerate}
            \item For all $x \in X$ there exists $B \in \B$ such that $x \in B $.
            \item If $B_1$, $B_2 \in \B$ and $x \in B_1 \cap B_2$ then there exists $B_3 \in \B$ such that $x \in B_3 \subset B_1 \cap B_2$. 
        \end{enumerate}

        We define the topology generated by $\B$ as the collection $\T$ such that for any $U \subset X$, if 
        for all $x \in U$ there exists $B \in \B$ such that $x \in B \subset U$ then $U \in \T$. 
        For any $U \in \T$ we say that $U$ is open. 
    \end{defn}

    
    \begin{prop}
        The topology generated by a basis is a topology. 
    \end{prop}

    
    \begin{proof}
        The first condidtion for a basis gives us $X$ as an open set and the empty set satisfys our condition vacuosly. 
        Now we must prove closure under arbitrary unions. Let $\{U_\alpha\}_{\alpha \in J} \subset \T$ be a collection of 
        open sets and consider their union. Then for $x \in \bigcup U_\alpha$ we must have $x$ apearing in some $U_\alpha$ since it 
        is in the union, but $U_\alpha $ is by assumtion open so there exists $B \in \B$ such that 
        \[x \in B \subset U_\alpha \subset \bigcup_{\alpha \in J} U_\alpha\]
        as desired. Now we must show that the finite intersection of open sets is again open, for that let 
        $\{U_1, \dotsb, U_n\}$ be a collection of open sets. Then if $x$ lies in their intersection it must lie in each 
        $U_i$. Thus there exists a familiy of basis elements $\{B_1, \dotsb, B_n\}$ such that $x \in B_1 \subset U_i$. 
        It follows then that $x \in \bigcap_{i < n} B_i$. Now to see that this intersection must be a basis element, we use induction 
        on the second part of the definition of a basis. 
    \end{proof}

    We now look at another way to define the topology generated by a basis. 

    
    \begin{lem}
        Let $X$ be a topological space and let $\B$ be a basis for the topology on $X$. Then $\T$ is equal 
        to set containg all unions of elements of $\B$
    \end{lem}

    
    \begin{proof}
        Let $U \in \T$, we want to write $U$ as a union of basis elements. By definition we know that for each $x \in U$ there exists $B_x \in \B$ satisfying 
        $x \in B_x \subset U$. Taking the union over all $B_x$ gives us the desired result. Now since Basis elements are open, any union of them must be contained 
        in $\T$, by definition. 
    \end{proof}

    It may be helpful to be able to check whether or not a given set of subsets forms a basis for the topology. 
    
    \begin{lem}
        Let $X$ be a topological space. Let $\C$ be a collection of open sets such that 
        for all $U \in \T$ and all $x \in U$ there exists $C \in \C$ such that 
        \[x \in C \subset U\]
        Then $\C$ is a basis for the topology on $X$
    \end{lem}

    
    \begin{proof}
        The first condition of a basis is satisfied by assumption. Now suppose $x \in C_1 \cap C_2$ we must show there exists 
        $C_3 \in \C$ such that $x \in C_3 \subset C_1 \cap C_2$. We may use the fact that $\C$ is a collection of open sets together with our assumption 
        to produce such an element. 

        Now we must show that $\C$ generates the correct topology. Let $\C$ generate $\T^\prime$. If $U$ is open in $\T$ then by assumption it is open in $\T^\prime$. 
        If $U$ is open in $\T^\prime$ then it is a union of elements of $\C$, since $\C$ is a collection of open sets of $\T$, $U$ must be open in $\T$.
    \end{proof}

    Now we may wish to tell whether one topology is finer than another, we can use the following lemma. 

    
    \begin{lem}
        Let $X$ be a set and let $\T, \T^\prime$ be topologies on $X$ with basis's $\B$ and $\B^\prime $ respectively. Then the following are equivalent. 
        
        \begin{enumerate}
            \item $\T^\prime$ is finer than $\T$
            \item For all $B \in \B$ and $x \in B$ there exists $B^\prime \in \B^\prime$ such that $x \in B^\prime \subset B$.  
        \end{enumerate}
    \end{lem}

    
    \begin{proof}
        $(2) \implies (1)$: Let $U \in \T$, then for every $x \in U$ there exists $B \in \B$ with $x \in B \subset U$. Then by assumption we have there exists 
        $B^\prime \in \B^\prime$ with $x \in B^\prime \subset U$. Hence $U$ is open in $\T^\prime$. 

        $(1) \implies (2)$: Assume that $\T^\prime$ is finer than $\T$. Then since $\T$ is a topology and $\T \subset \T^\prime$, for all $x \in B$ 
        there must exist $B^\prime \in \B^\prime$ such that $x \in B^\prime \subset B$. 
    \end{proof}
    
    To end this section we discuss subbasises. 

    
    \begin{defn}
        Let $X$ be a set. A subbasis is a collection of subsets $A$ such that th union over $A$ is $X$. We define the topology generated 
        by the subbasis as the collection of all unions of all intersections of elements of $A$
    \end{defn}

    Of course we must prove that this is indeed a topology, but first notice that the definition of a subbasis is just the first axiom of a basis. Thus, 
    every basis is a subbais, and if $\B$ is a basis considering it as a subbasis will generate the same topology. So Subasises are a generalization of 
    of a basis. Can you give an example of a subbasis which is not also a basis? 

    
    \begin{prop}
        The topology generated by a subbasis is a topology
    \end{prop}
    
    \begin{proof}
        It is sufficient to show that the collection of all finite intersections of elements of $A$ is a basis. Since the union of all elements 
        of $A$ is $X$, the first condition of a basis is clearly satisfied. Now suppose $x \in a_1 \cap a_2$, where $a_1, a_2$ are finite intersections of elements of $A$.
        Then $a_1 \cap a_2$ is a finite intersection with length equal to the sum of the lengths of $a_1$ and $a_2$. 
    \end{proof}

    
    \section{Order Topology}
    
    \begin{defn}
        Let $X$ be a set, we define the order topology on $X$ as the topology generated by the basis $\B$ such that 
        $\B$ contains all elements of the form 
        
        \begin{enumerate}
            \item $(a, b)$ for $a, b \in X$. 
            \item $[a_0, b)$ where $a_0 $ is the minimal element of $X$. 
            \item $(a, b_0]$ where $b_0$ is the maximal element of $X$.
        \end{enumerate}
        If $X$ has no maximal or minimal elements, then $\B$ consists only of elements of the first type. 
    \end{defn}

    Now we must of course prove that this choice of $\B$ does indeed form the basis of a topology, but it is clear since 
    the intersection of any of these sets yields another set of the same type. There are several cases to check. 
    
    \begin{ex}
        Consider the order topology on $\R$, since $\R$ has no maximal or minimal elements the order topology is generated by 
        $\B = \{(a, b) | a, b \in \R\}$. Then it is clear that this coincides with the standard topology on $\R$
    \end{ex}

    
    \begin{ex}
        Consider the order topology on $\Z_{+}$ given the usual order. Then the order topology is equivalent to the discrete topology. 
    \end{ex}

    
    \begin{ex}
        Consider $\R \times \R$ equipped with the dictionary order, then the basis elements are of the form 
        $(a, b) \times (c, d)$ where $a < c$ or $(a = c) \wedge (b < d)$. 
    \end{ex}

    \begin{thm}
        Open rays for a subbasis for the order topology on $X$. 
    \end{thm}

    \begin{proof}
        We may simply prove that finite intersections form a basis. Let $x \in U$ for some open set $U$, then $x$ is in some basis element $B$ contained in $U$. 
        Now $B$ can be one of three forms, if $X$ has no minimal or maximal element, $B = (a, b)$ for some $a < b$. Then $x \in (-\infty, b) \cap (a , \infty) \subset U$. 
        Now suppose that $X$ has a minimal element $a_0$, then $B$ can be of the form above or $[a_0, b)$, but if it is the former we are done and the latter is already an open ray. 
        Now since open rays are open in the order topology, the order topology must contain the topology generated by the open rays, so we are done. 
    \end{proof}

    \section{Product Topology}
        Given two sets $A$ and $B$ we may want to define a topology on the cartesian product $A \times B$. 
    The cartesian product comes equipped with two functions $\pi_A(x): A \times B \to A$ and $\pi_B(x): A \times B \to B$ called 
    the projections of $A \times B$. 

    
    \begin{defn}
        Given topological spaces $A, B$ we define the basis for the topology on $A \times B$ as 
        \[\B = \{U \times V \}\]
        Where $U$ is open in $A$ and $V$ is open in $B$. 
    \end{defn}

    First we must prove that this is indeed a topology on $A \times B$. In fact not only does this define a topology, but it will satisfy certain universal properties that 
    we would like it too, given an arbitrary topological space $X$ with continuous maps to $A$ and to $B$ there is a unique map into $A \times B$ that makes the relevant diagrams commute.

    
    \begin{proof}
        place
    \end{proof}
    The next theorem is obvious and the proof is trivial

    
    \begin{thm}
        If $\B$ is a basis for $A$ and $\C$ is a basis for $B$ then $\B \times \C$ is a basis for the product topology. 
    \end{thm}

    There are plenty of good examples in the book. 


   
   \begin{thm}
    The projections $\pi_A$ and $\pi_B$ are both continuous maps 
   \end{thm}

   \begin{thm}
    Sets of the form 
    \[S = \{\pi^{-1}_1(A) \cup \pi^{-1}_2(B) \}\]
    For $A \subset X$ and $B \subset Y$ opensets, form a subbasis for the product topology. 
   \end{thm}

   \begin{proof}
    Elements in $S$ are clearly open sets of the product topology, so then so are unions and finite intersections. 
    We have $\pi^{-1}_1(A) = A \times Y$ and $\pi^{-1}_2(B) = X \times B$, then the intersection is clearly a basis element of the product topology. So finite intersectons of elements of $S$ give us all basis elements of the product topolgy in the 
    topology generated by $S$. 
   \end{proof}

   \section{Subspace Topology}
   Given a topological space $X$ and $Y \subset X$ we may often wish to place some topology on $Y$ and consider it as a topological space. There is a very clear way of doing that, 
   and in this case we just define the open sets of $Y$ without bothering to define a basis. 

   
   \begin{defn}
    $U \subset Y$ is open in the topology of $Y$ if there exists an open set $V \subset X$ such that $U = V \cap Y$
   \end{defn}

   That is to say that the open sets of $Y$ are just the open sets of $X$ intersected with $Y$. Once again one must show that this defines a topology on $Y$. 
   \begin{proof}
    Since $X \cap Y = Y $ and $\varnothing \cap Y = \varnothing$ it is clear that $\varnothing, Y$ are open in the topology on $Y$. 
    Now let $\{U_\alpha\}_{\alpha \in J}$ be an arbitrary collection of opensets in $Y$. Then for each $U_\alpha$ we may fix an open set $V_\alpha \subset X$ such that $U_\alpha = V_\alpha \cap Y$. 
    \[\bigcup_{\alpha \in J} U_\alpha = \bigcup_{\alpha \in J} (V_\alpha \cap Y) = \left(\bigcup_{\alpha \in J} V_\alpha \right) \cap Y\]
    Then since each $V_\alpha$ is open in $X$ and $X$ is a topological space we have that their union must also be an open set. Then it follows that the intersection is open in $Y$. 
    Lastly, we must show that finite intersections of open sets are open. To this end, let $\{U_1, \dots , U_n \}$ be a finite collection of open sets. 
    Then just as before, we may associate with each $U_k$ a $V_k$ such that $V_k \cap Y = U_k$. Then taking the intersection gives 
    \[\bigcap_{i \leq n} (V_k \cap Y) = \left( \bigcap_{i \leq n} V_k \right) \cap Y.\]
    We know what this must be open in $Y$ since finite intersections of opensets are again open by the definition of a topology. 
   \end{proof}
   
   \begin{thm}
    If $A \subset Y$ is open in $Y$ and $Y$ is open in $X$ then $A$ is open in $X$
   \end{thm}

   
   \begin{proof}
    Since $A$ is open in $Y$ there exists an open set $B$ such that $A = Y \cap B$, then since $Y$ is open in $X$ it follows that $A$ is open in $X$. 
   \end{proof}


   \begin{thm}
    If $\B$ is a basis for $X$ then $B \cap Y$ for $B \in \B$ is a basis for the subspace topology on $Y$. 
   \end{thm}

   \begin{proof}
    Let $\B$ be a basis for $X$ and let $\C = \{B \cap Y | B \in \B \}$. We want to show that $\C$ is a basis for the topology on $Y$. Let $U$ be an open set of $Y$ and let $x \in U$, then $x \in V \cap Y = U$ for an open $V \subset Y$ and there exists a basis element $B \in \B$ such that 
    $x \in B \subset V$. Then we have $x \in B \cap Y \subset U = V \cap Y$, hence $\C$ is a basis. 
   \end{proof}
   
   So far we have disscussed three different ways of putting a topology on a set. We may wonder, when do they give us the same topology. 


   \begin{thm}
    Let $X, Y$ be spaces with $A \subset X$ and $B \subset Y$. Then the product topology on $A \times B$ is the same as 
    considering $A \times B$ as a subspace of $X \times Y$. 
    \end{thm}

    \begin{proof}
        We want to show that two topologys are the same so we must show that they have the same opensets. It will follow from 

        \[(U \times V ) \cap (A \times B) = (U \cap A) \times (V \cap B)\]
        So I can rewrite basis elements of one topolgy as that of the other. 
    \end{proof}

    So products and subspaces play nicely with each other, we will see that this is not the case in general. 

   \begin{ex}
    Let $I = [0,1]$ and $X  = \R$. Now by the order topolgy on $I$ we mean the order of  $\R$ restricted to $I$, we will compare this with the subspace topology on $I$. 
    The basis elements of the order topology on $I$ will be of the form $[0, b), (a, b), (a, 1]$, Now if we consider a basis element of $\R$ intersected with $I$ we will get one 
    of these intervals, so the order and supspace topology are the same.  
   \end{ex}

   Now lets see some examples that are not so nice. 


   \begin{ex}
    Let $I = [0, 1] \cup \{\frac{1}{2}\}$, then the subspace topology on $I$ will give $\{\frac{1}{2}\}$ as an open set, but in the order topology it will be closed. 
   \end{ex}

   One may observe that in order to get an example where the order and subspace topology did not give us the same thing we had to consider a strange set. This may hint that there is a nice 
   way of formulating when the subspace and order topology agree. This will lead us to the idea of convexity, which we will discuss later, but first another example. For $a, b \in X$ we will 
   use $a \times b$ to mean the $\emph{point} \, (a, b) \in X^2$ to avoid problems with our notation.  

   \begin{ex}
    Let $I = [0, 1] \times [0, 1]$, and let $X = \R^2$ in the dictionary order topology. Then considering $I$ as a subspace of $\R^2$ gives a different topolgy then restricting the dictionary order to $I$ and considering the order topology. 
    To see this one can consider the set $\{\frac{1}{2} \times (\frac{1}{2}, 1]\}$, we can easily find an open set of $\R^2$ to intersect $I$ which gives this set, so it is open in the subspace topology. However it is a half open interval which ends 
    at the point $\frac{1}{2} \times 1$, which cannot be open in the order topolgy on $I$.

    We will call $I$ the ordered square denoted $I^2_o$. 

   \end{ex}

   Now we will answer the quesion, when do the order topology and subspace topolgy agree? 
   \begin{defn}
    Let $X$ be an ordered set, then a subset $A \subset X$ is called $\emph{convex}$ If for every $a, b \in A$ such that $a < b$ then $(a, b) \subset A$. 
   \end{defn}

   \begin{thm}
    If $X$ is an ordered set and $A \subset X$ then the subspace topology on $A$ equals the order topolgy on $A$ if $A$ is convex. 
   \end{thm}

   \begin{proof}
    Consider an open ray of $X$, $(a, \infty)$, Then we will look at its intersection with $Y$. If $a \in Y$ 
    \[Y \cap (a, \infty) = \{x | x \in Y \, x > a\}\]
    is an open ray in $Y$. If $a \notin Y$ then it is either a lower bound or an upperbound since $Y$ is convex. 
    If it is an upper bound the intersection is empty, if it is a lower bound then $Y \subset (a, \infty)$ so that the 
    intersetion just gives $Y$. Thus the subspace topology is contained in the order topology on $Y$. Conversly if we have 
    some interval in $Y$, then that interval can easily be written as an open set of $X$ intersected with $Y$.
   \end{proof}


   \section{Closed sets and Limit points}
   So far we have discussed only open sets, now we will move on to closed sets and limit points. 

   \begin{defn}
    Let $X$ be a topological space, then $A \subset X$ is $\emph{closed}$ if its complement $A^c$ is open. 
   \end{defn}

   It is important to understand that we consider open and closed sets to be dual under complementation, not under logical negation; that is, a set can be both open and closed or it can even be neither! 
   For example, since $\varnothing, X$ are both open in any topology and they are complements of each other, it follows they are both open and closed at the same time. 


   \begin{thm}
    Let $X$ be a space, then 
    \begin{enumerate}
        \item $\varnothing, X$ are closed. 
        \item Finite unions of closed sets are closed. 
        \item Arbitrary unions of closed sets are closed. 
    \end{enumerate}
   \end{thm}

   \begin{proof}
    $(1)$ is clear. For $(2)$, let $\C = \{C_1, \dots, C_2\}$ be a finite set of closed sets. Then 
    \[\left(\bigcup C_n \right)^c = \bigcap (C_n^c)\]
    which is a finite intersection of open sets and thus must be open. Since the complement of $\C$ is open, by definition, $\C$ must be closed. 
    The proof of $(3)$ is similliar. 
   \end{proof}

   One can easily see the connection of the above with the definiton of a topological space, in fact, we could have definied a topology in terms of closed sets and then defined open sets to be the complements of closed sets. We would obtain the 
   exact same theory. 

   \begin{thm}
    Let $Y$ be a subspace of $X$, then if $A \subset Y$ is closed in $Y$, then there is a closed set $C \subset X$ such that 
    \[A = C \cap Y\]
   \end{thm}

   \begin{proof}
    Since $A$ is closed in $Y$, $Y \setminus A$ is open in $Y$, hence there exists an open set of $X$, $V$ such that $Y \setminus A = V \cap X$. 
    Then taking the complement of $V$ in $X$ gives us a closed set of $X$ whose intersection with $Y$ will equal $A$. 
   \end{proof}

   \begin{thm}
    If $Y$ is a subspace of $X$, and $Y$ is closed in $X$, then any closed subset of $Y$ is also closed in $X$. 
   \end{thm}

   \begin{proof}
    This is the same as for opensets in a previous section. Let $A$ be closed in $Y$, then there exists $C$ so that 
    $A = C \cap Y$, then since intersectons of closed sets are again closed, $A$ must be closed in $X$.
   \end{proof}

   \begin{thm}
    Let $X$ be a space and $Y$ a subspace with $A \subset Y$, then the closure of $A$ in $Y$, $\bar{A_y}$ is equal to the intersection of $Y$ with the closure of $A$ in $X$, $\bar{A_x }$. 
   \end{thm}

   \begin{proof}
    It is clear that $A \subset \bar{A_y}$ and since $\cA_y$ is closed there exists an closed set in $Y$, $C$ wich also must contain $A$, such that $C \cap Y = \cA_y$. 
    Then $\cA_x \cap Y \subset C \cap Y = \cA_y$. Conversly, $\cA_x \cap Y$ is a closed set of $Y$ which contains $A$, so $\cA_y \subset \cA_x \cap Y$.
   \end{proof}

   Now we move on to two important concepts, that of the interior and the closure of a set. In particular, we will find the closure of a set very useful. 

   \begin{defn}
    The $\emph{closure}$ of $A$ is the intersection of all closed sets which contain $A$. We denote the closure by $\bar{A}$. 
    Now the interior of $A$ is the union of all opensets contained in $A$. We will denote the interior of a set by $\interior{A}$. 
   \end{defn}

   Since the intersections of closed sets are always again closed and the union of open sets is always open, we can see that the clousre of $A$ is a closed set and the interior of $A$ is an open set. 
   We can think of these two sets as approximating $A$ with open and closed sets. For the following theorem we will introduce a new term, we say that a set $X$ intersects $Y$ if the intersection of $X$ and $Y$ is non-empty.

   \begin{thm}
    Let $X$ be a space and $A \subset X$. The following two statements are true
    \begin{enumerate}
        \item $x \in \bar{A} $ if and only if for every nbhd of $x$ intersects $A$ 
        \item $x \in \bar{A}$ if and only if every basis element containg $x$ intersects $A$. 
    \end{enumerate}
   \end{thm}

   \begin{proof}
    To prove the first statement we will use the contrapositive, so assume that there exists a nbhd of $x$, $U$ that doesn't intersect $A$. Then 
    $X \setminus U$ is a closed set containg $A$ but not containg $x$, thus $x$ cannot belong to the closure of $A$. Conversy assume that $x$ is not in 
    the closure, then there is some closed set $C$ which contains $A$ but does not conatain $x$, then taking $X \setminus C$ gives a nbhd of $x$ which does not intersect 
    $A$. 

    The second statement follows easily from the first, if $x \in \bar{A}$ then every nbhd of $x$ intersects $A$, and the basis elements containg $x$ are certainly nbhds of $x$. 
    Then if every basis element contain $x$ intersects with $A$, it follows that every open set must aswell, but then by $(1)$, we have that $x$ is in the closure of $A$.
   \end{proof}


   Now it is clear that $X$ is closed if and only if it is equal to its closure and $X$ is open if and only if it is equal to its interior. 


   \begin{defn}
    Let $X$ be a space and $A \subset X$, we say that $x \in X$ is a limit point or an accumluation point of $A$ if every nbhd of $x$ intersects 
    $A$ in a point $\emph{other}$ than $x$. We denote the set of limit points of $A$ as $A^\prime$. 
   \end{defn}

   \begin{thm}
    Let $X$ be a space and $A \subset X$, then $\bar{A} = A \cup A^\prime$. 
   \end{thm}

   \begin{proof}
    We procced with a double containment argument, first suppose that $x \in \bar{A}$, then if $x \in A$ we are done so we may add the assumption that $x \notin A$. Then 
    since $x$ is in the closure, we know that every nbhd of $x$ intersects $A$, but since $x \notin A$ this intersection must be in a point other than $x$, thus $x \in A^\prime$. 
    Now suppose that $x \in A \cup A^\prime$ then again, if $x \in A$ we are done, so suppose that $x \in A^\prime$ then by definition, every nbhd of $x$ intersects $A$ in a point other than $x$, 
    thus every nbhd of $x$ intersects $A$ so we must have that $x$ belongs to the closure. 
   \end{proof}

   \begin{cor}
    $A$ is closed if and only if it contains all of its limit points. 
   \end{cor}

   We will find that the above characterization of a closed set is very usefull. 

   \begin{defn}
    Let $X$ be a topological space, we say that $X$ is $\emph{Hausdorff}$ if for any two $x, y \in X$, there exits disjoint
    open sets $U_1, U_2$ such that $x \in U_1 $ and $y \in U_2$. 
   \end{defn}


   \begin{thm}
    Let $X$ be Hausdorff, then finite point sets are closed. 
   \end{thm}

   \begin{proof}
    It is sufficient to prove that any singleton $\{x\}$ is closed, since finite unions of closed sets are closed. 
    Now we will deterimine the closure of $\{x\}$, note for any $y \neq x$, there must exists a nhbd of $y$ which does not contain $x$, 
    but then this nbhd has an empty intersection with $\{x\}$, so $y$ cannot be in the closure. It follows that $\{x\}$ is its own closure and as such, 
    it must be closed. 
   \end{proof}

   Note that the contidion that finte point sets be closed is acutally a weaker assumption that the Hausdorff axiom, the condition that fintie sets be closed is refered to as the $T_1$ axiom. 
   Generally most interesting theorems will require the full strenght of the Hausdorff axiom so we have little interest in $T_1$, asside from the following theorem. 


   \begin{thm}
    The $X$ be a space with the $T_1$ axiom, let $A \subset X$. Then $x \in X$ is a limit point of $A$ if and only if every nbhd of $x$ intersects $A$ in 
    infinitly many points. 
   \end{thm}

   \begin{proof}
    If every nbhd of $x$ intersects $A$ in infinitly many points, then it intersects $A$ so that $x$ is a limit point. 

    conversly, assume for the sake of contradiction that there exists a nbhd $U$ containing $x$ whose intersection with $A$ is finite. 
    Let $\Xi$ be the intersection of $U$ and $A$ execept for possibly $x$. Then $\Xi = \{x_1, \dots, x_2\}$. Now since finite point sets are closed we have that $\Xi$ is closed. 
    We will now construct an open set containg $x$ which doesn't intersect $A$. Consider $U \cap X \setminus \Xi$. Since finite intersections of open sets are open, this is open, 
    it contains $x$, but intersects $A$ nowhere, contradicting the fact that $x$ is a limit point. 
   \end{proof}

   A corollary of the last result is that in finite spaces every Hausdorff topology in the discrete topology, and further, there are no limit points. 


   \section{Continuous Functions}

   Continuous functions are of extreme importance in analysis and at a basic level they can be defined as functions which preserve limits of sequences. More generally 
   we must realize that continuity is as much a property of functions as it is of general spaces. That is, whether a function is continuous or not depends on the space it 
   is in. 

   
   \begin{defn}
    A function $f: X \to Y$ is said to be $\emph{continuous}$ if the preimage of every open set in $Y$ is open in $X$. 
   \end{defn}


   From analysis we recall the epsilon delta definition of continuity, we will find that in a metric space, that definition is equivalent to ours. 
   However, the definition just given has the advantage of being much more general. 

   We proved a different order in class. 
   \begin{thm}
    Let $f$ be a continuous function, then the following are all equivalent. 
    \begin{enumerate}
        \item $B \subset Y$ closed, $f^{-1}(B)$ closed. 
        \item For $A \subset X$, we have $f(\bar{A}) \subset \bar{f(A)}$. 
        \item (local formulation of continuity) For every $x \in X$ then for all nbhds of $f(x)$, $V \subset Y$, there exists $U \subset X$ such that $f(U) \subset V$. 
    \end{enumerate}
   \end{thm}

   \begin{proof}
    $(1)$ is clear, then since $\bar{f(A)}$ is closed its pre-image is a closed set containing $A$, so $\bar{A} \subset f^{-1}(\bar{f(a)})$; from this 
    $(2)$ follows. 
    
    \medskip
    then let $A \subset X$, Then for $a \in \bar{A}$ we want to show $f(a) \in \bar{f(A)}$. Then consider a nbhd of $f(a)$, $V$, its preimage 
    must both be open and contain $a$, thus $f^{-1}(V) \cap A$ is non-empty. But from this if follows 
    \[f(f^{-1}(V) \cap A) \subset V \cap f(A)\]
    so $V$ intersects $f(A)$, thus $f(a)$ belongs to the closure. 
   \end{proof}


   equivalence with $\epsilon-\delta$. 

   
   \begin{defn}
    Let $X, Y$ be spaces, and let $f:X \to Y$ be a continuous injection. Then we call $f$ an $\emph{embedding}$ 
    of $X$ into $Y$. 
   \end{defn}

   \begin{defn}
    A function $f:X \to Y$ is a homeomorphism, if it is a continuous bijection with a continuous inverse. 
   \end{defn}

   
   \begin{ex}
    A function can we a continuous bijection without having a continuous inverse. for example 
    consider the function ${\tt id}:\R_l \to \R$. The mapping is continuous since the lower limit topology 
    is finer than the standard topology on $\R$ but its inverse will not be continuous since there exists 
    an open set of $\R_l$ whos pri image is not open in $\R$. 
   \end{ex}


    Homeomorphisms are important because they preserve the topology of a set, that is, a homeomorphism induces a bijection 
    between the open sets of two spaces. 

   state and prove thm abt this. 

   In general given continuous functions there are many  ways we may go about constructing more. 




   
   \begin{thm}[Pasting Lemma]
    let $X, Y$ be spaces and let $A,B$ be closed subsets of $X$ with $X = A \cup B$
    \[f:A \to Y \, \, \, g:B \to Y\]
    such that $f$ and $g$ agree on the intersection of $A$ and $B$, 
    then we can construct a continuous function $h:X \to Y$. 
   \end{thm}


   \begin{thm}[Maps into products]
    $f:X \to A \times B$ where $x \mapsto (f_1(x), f_2(x))$ 
    is continuous if and only if $f_i$ is continuous. 
   \end{thm}

   \section{Product Topology}
   In this section we will generalize our results on the product topology to arbitrary cartesian products. 


    
\end{chapter}
