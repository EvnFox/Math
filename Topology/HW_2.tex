\documentclass[11pt,largemargins]{homework}
\usepackage{commath}
\newcommand{\R}{\mathbb{R}}
\newcommand{\N}{\mathbb{N}}
\newcommand{\Q}{\mathbb{Q}}
\newcommand{\Z}{\mathbb{Z}}
\newcommand{\T}{\mathfrak{T}}
\newcommand{\B}{\mathfrak{B}}
\newcommand{\C}{\mathfrak{C}}
\newcommand{\ran}{\operatorname{ran}}
\newcommand{\dom}{\operatorname{dom}}
\newcommand{\eps}{\varepsilon}
\newcommand{\ssd}{\bigtriangleup}
\newcommand{\pow}{\mathcal{P}}

% TODO: replace these with your information
% TODO: replace these with your information
\newcommand{\hwname}{Evan Fox}
\newcommand{\hwemail}{efox20@uri.edu}
\newcommand{\hwtype}{}
\newcommand{\hwnum}{}
\newcommand{\hwclass}{MTH 525: Topology}
\newcommand{\hwlecture}{}
\newcommand{\hwsection}{}

 


\begin{document}
    \maketitle

    \question 
    Show that if $A$ is closed in $Y$ and $Y$ is closed in $X$ then $A$ is closed in $X$. 

    \begin{proof}
        Assume that $A$ is closed in $Y$, then there exists a closed set of $X$, $C$ such that $A = Y \cap C$. 
        Then since this is the intersection of two closed sets in $X$, $A$ is closed. 
    \end{proof}

    \question 
    Let $A$, $B$, and $A_\alpha$ denote subsets of a space $X$. Prove the following: 
    \begin{enumerate}
        \item If $A \subset B$, then $\bar{A} \subset \bar{B}$ 
        \item $\overline{A \cup B} = \overline{A} \cup \overline{B}$ 
        \item $\overline{\bigcup A_\alpha } \supset \bigcup \overline{A_\alpha}$
        
    \end{enumerate}

    \begin{proof}
        (1) Let $A \subset B$. Let $x$ be a limit point of $A$, then every nbhd of $x$ intersects $A$ in a point other than $x$, since $A \subset B$, every nbhd of $x$ also must contain a point of $B$ other than $x$, 
        hence $x$ is a limit point of $B$, it now follows 
        \[\cA = (A \cup A^\prime) \subset (B \cup B^\prime ) = \cB\]
        as desired. 

        (2) Let $x \in \overline{A \cup B}$ and suppose $x \notin \overline{B}$, then (since $x$ is not a limit point) there exists a neigborhood of $x$, $U$ which does not intersect $B$, now if there also exists a neighborhood $V$ which doesn't intersect $A$ in a point 
        other than $x$, taking the intersection $U \cap V$ furnishes an open set containg $x$ which doesn't intersect $A \cup B$ (in a point other than $x$), which is a contradiction. Hence every neighborhood or $x$ must intersect $A$ so that $x$ belongs to the 
        closure of $A$. Conversly,  let $x \in \overline{A} \cup \overline{B}$, then suppose $x \in \overline{A} = A \cup A^prime$, if $x \in A$, then we are done so assume that $x$ is a limit point of $A$. Then every neighborhood will intersect $A$ in a point other than $x$, so 
        every neighborhood intersects $A \cup B$ in a point other than $x$ so that $x$ belongs to the closure of $A \cup B$. If $x \in \overline{B}$, the argument is similliar. 

        (3)Let $x \in \bigcup \overline{A_\alpha}$, then $x \in \overline{A_\alpha}$ for some $\alpha$, hence every neighborhood intersects $A_\alpha$ and thus intersects $\bigcup A_\alpha$, hence $x \in overline{\bigcup A_\alpha}$. 
        To see that the converse is false, let $A_n = (0, \frac{n}{n+1})$ for $n \in \N$. Then $\bigcup_{n \in \N}A_n = (0, 1)$ so $1 \in \overline{\bigcup_{n \in \N} A_n}$. But for any $A_k$, the $\epsilon-$ball of radius $\frac{1}{2} |\frac{k}{k+1} -1|$, 
        is a neighborhood around $1$ which doesn't intersect $A_k$, hence $1 \notin \overline{A_k}$, since $k$ was arbitrary, $1 \notin \bigcup \overline{A_k}$.  

    \end{proof}


    \question 
    Let $A, B, $ and $A_\alpha$ be as in the previous question. Determine if the following are true. 

    \begin{enumerate}
        \item $\overline{A \cap B} = \overline{A} \cap \overline{B}$. 
        \item $\overline{\bigcap A_\alpha} = \bigcap \overline{A_\alpha}$. 
        \item $\overline{A \setminus B} = \overline{A} \setminus \overline{B}$. 
        
    \end{enumerate}

    \begin{proof}
        (1) Let $x \in \overline{A \cap B}$, if $x \in A \cap B$ the result is clear so suppose that $x$ is a limit point; then every neighborhood of $x$ intersects $A \cap B$. 
        Hence every neighborhood will intersect $A$ and $B$, thus $x$ is a limit point of $A$ and $B$ so $x \in \overline{A} \cap \overline{B}$. Now suppose 
        $x \in \overline{A} \cap \overline{B}$, then $x$ belongs to the closure of $A$ and $B$ so that every neighborhood of $x$ intersects $A$ and every neighborhood must also intersect $B$. 
        Hence, every neighborhood intersects $A \cap B$, so $x \in \overline{A \cap B}$. 
        
        (2) Suppose that $x \in \bigcap \overline{A_\alpha}$, then $x$ is in the closure of $A_\alpha$ for all $\alpha$. Then every neighborhood of $x$ intersects $A_\alpha$ for all $\alpha$. 
        Thus every neighborhood must intersect $\bigcap A_\alpha$, so $x \in \overline{\bigcap A_\alpha}$. Now conversly let $x \in \overline{\bigcap A_\alpha}$, then an arbitrary neighborhood $U$
        of $x$ intersects $\bigcap A_\alpha$, and so $U$ intersects each $A_\alpha$, thus $x \in \overline{A_\alpha}$ and so it belongs to $\bigcap \overline{A_\alpha}$. 

        (3) We prove $\overline{A} \setminus \overline{B} \subset \overline{A \setminus B}$. Let $x \in \overline{A} \setminus \overline{B}$. Then $x \notin B$. If $x \in A$, then 
        $x \in A \setminus B \subset \overline{A \setminus B}$, so assume that $x$ is a limit point of $A$. It follows that every neighborhood of $x$ intersects $A$ in a point not in $B$, since otherwise
        if such a neighborhood existed it would contradict our assumption on $x$. Then every neighborhood intersects $A \setminus B$ so $x \in \overline{A \setminus B}$. 
    \end{proof}



    \question 
    Let $X$ and $X^\prime$ denote a single set in the teo topologies $\T$ and $\T^\prime$, respectiviely. 
    Let $i: X^\prime \to X$ be the identity function. 
    \begin{enumerate}
        \item Show that $i$ is continuous iff $\T^\prime$ is finer than $\T$
        \item Show that $i$ is a homeomorphism iff $\T^\prime = \T$
    \end{enumerate}


    \begin{proof}
        (1) Assume that $i$ is continous, then let $U$ be open in $X$ (i.e, $U \in \T$). It follows from continuity that $i^{-1}(U) = U \subset X^\prime$ is open, thus 
        $\T \subset \T^\prime$. Conversly, assume that $\T^\prime $ is finer than $\T$. Then let $U$ be open in $X$, since $\T^\prime$ is finer than $\T$, we know that the primage of $U$, 
        is open in $X^\prime$, hence $i$ is continous. 

        (2) If $i$ is a homeomorphism, we know that it and its inverse are continous, 
        so we simply apply (1) in both directions to get $\T^\prime \subset \T$ and $\T \subset \T^\prime$ and hence $\T^\prime = \T$. Now conversly, assume $\T^\prime = \T$. Again by applying (1) in both directions 
        we will get that $i:X^prime \to X$ is continous and $i^{-1}:X \to X^\prime$ is continous, it is a homeomorphism. The fact that $i$ is a bijection follows since the identity map from a space to its self is always a bijection.  
    \end{proof}



    \question 
    Let $Y$ be an ordered set in the order topology. Let $f,g:X \to Y$ be continous. 
    \begin{enumerate}
        \item Show that the set $\{x|f(x) \leq g(x) \} $ is closed in $X$
        \item Show that $h(x) = min\{f(x), g(x) \}$ is continuous.
    \end{enumerate}

    \begin{proof}
        (1) We prove $X -S$ is open. If it is empty we are done, so suppose there exists $x_0 \in X - S$, i.e. assume 
        $f(x_0) > g(x_0)$. Since $Y$ is in the order topology, it is Hausdorff, thus there exists disjoint nbhd's $V_1, V_2$ with 
        $f(x_0) \in V_1$ and $g(x_0) \in V_2$. Since $f$ and $g$ are continuous functions, there exists $U_1, U_2 \subset X$ around $x_0$ such that 
        \[f(U_1) \subset V_1 \, \text{ and } \, g(U_2) \subset V_2\]
        Now take $U = U_1 \cap U_2$. Then for $x \in U$, we have $f(x) \in V_1 $ and $g(x) \in V_2$, since $f(x_0)> g(x_0)$ and $V_1 \cap V_2 = \varnothing$, 
        it follows $f(x) > g(x)$ hence there is a nbhd around $x_0$ contained in $X-S$, so $x_0$ is an interior point. Since it was chosen arbitrarily, it follows that 
        $X - S$ is open. 


        (2) 
        Define $A = \{x|f(x) \leq g(x) \}$ and $B = \{x|g(x) \leq f(x) \}$. By the above argument both of these sets are closed and it is clear $A \cup B = X$. Further,  $x \in A \cap B$ implies $f(x) = g(x)$.
        Now define $h(x) = f(x)$ when $x \in A$ and $h(x) = g(x)$ for $x \in B$. Then we see $h(x) = {\tt min}\{f(x), g(x)\}$ and by the pasting lemma $h$ is continous. 
       \end{proof}


    \question 
    Let $F: \R \times \R \to \R$ be defined by the equation ... 
    \begin{enumerate}
        \item Show that $F$ is continous in each variable separately
        \item Compute $g(x) = F(x \times x)$ 
        \item show that $F$ is not continous
    \end{enumerate}


    \begin{proof}
        (1) Without loss of generality fix $y \in \R$, if $y = 0$ the function just becomes the zero function which is continous. If $y \neq 0$, then the function will never have a denominator of $0$ since $x^2 + y^2 > 0$ for all $x$ given $y \neq 0$. Then $F$ 
        just becomes a quotient of two continous functions (polynomials are continuous) with a nonzero denominator on its domain and therefore is continous. We can ingore that it was defined peicewise since it will be zero iff $x = 0$. The situation for a fixed $X$ is the same since there is clearly some symmetry with the variables, 
        the proof would just be a relabeling of the above. 

        (2) if $y = x$ then $\frac{xy}{x^2 + y^2} = \frac{x^2}{2x^2} = \frac{1}{2}$ so
        \begin{equation}
                g(x) \, = \, 
                \left\{\begin{array}{lr}
                    \frac{1}{2}, & \text{ if } x \neq 0 \\
                    0, & \text{if } x = 0
                \end{array}\right\}
        \end{equation}

        Note that $g$ is not continous at $0$.

        (3) Define $h: \R \to \R^2$ such that $h(x) = (x, x)$. Then since maps into products are continous iff the coordinate functions are continuous, we see that $h$ is continuous. 
        Now assume that $F$ is conintinous, then $F \circ h:\R \to \R$ is continuous (composistion of continuous functions), but $F \circ h = F(x \times x) = g(x)$ is discontinuous at $0$; a contradiction. Hence, $F$ is not continuous. 

    \end{proof}


    \question 
    Let $x_1, x_2, \dots$ be a sequence of the points of the product space $\Pi X_\alpha$. Show that this sequence converges to the point $x$ if and only in the sequence $\pi_\alpha(x_1), \dots$ converges to $\pi_\alpha(x)$ for each $\alpha$. Is this fact true if one uses the 
    box topology instead of the product topology? 

    \begin{proof}
        Suppose $(x_n) \rightarrow x$, let $V_\alpha \subset X_\alpha$ be a nbhd around $\pi_\alpha(x)$. Then the preimage $\pi_\alpha^{-1}(x) \subset \Pi X_\alpha$ is an open set containing $x$, and thus contains all but finitely many points 
        of the sequence $(x_n)$. But then $V_\alpha$ must contain all but finitely many points of the sequence $(\pi_\alpha(x_n))$. Hence $(\pi_\alpha(x_n)) \rightarrow \pi_\alpha(x)$. Since I only used the fact that projections are continous 
        this direction is true in the product or box topology. The converse is only true in the product topology. Assume that $(\pi_\alpha(x_n)) \rightarrow \pi_\alpha(x)$ for all $\alpha$. Then let $U = {U_\alpha}_1 \times \dots {U_\alpha}_m \times X \times \dots$. 
        be a neighborhood around $x$. Then for each $\alpha_i$ there exists a $k_i$ such that for all $k > k_i$, ${\pi_\alpha}_i(x_k) \in {U_\alpha}_i$. Now take $K = {\tt max}\{k_1, \dots k_m\}$, then for $k > K$, 
        $x_k \in U$. Hence $(x_n) \rightarrow x$. To see that this is false in the box topology, let $X = \R^\omega$ and consider the neighborhood $A = (-1, 1) \times (\frac{-1}{2}, \frac{1}{2}) \times \dots$ around $0$, and define the sequence 
        $x_n = (\frac{1}{n}, \frac{1}{n}, \frac{1}{n}, \dots)$. Then each projection converges to $0$ in $\R$, but for each index $k$, $x_k \notin A$ since the $(k+1)^{th}$ index of $x_k$ is not in $(\frac{-1}{k+1}, \frac{-1}{k+1})$. Hence the sequence cannot 
        converge to zero. 


    \end{proof}


    \question 
    Let $\R^\infty$ be the subset of $R^\omega$ consisting of all sequences that are eventually zero. What is the closure in the product and box topologies

    \begin{proof}
        First consider the product topology. Let $x \in \R^\omega$ and let $U = {U_\alpha}_1 \times \dots \times {U_\alpha}_n \times \R \times \dots$ be an open set 
        of $x$. Then it is clear that $U$ contains an element of $\R^\infty$, we can pick any element from each ${U_\alpha}_i$ for $i = 1, \dots n$ and then just pick zeros for the rest. 
        Hence every $x$ is a limit point and thus $\R^\infty $ is dense in $\R^omega$ so its closure is the whole space. 


        Now we consider the box topology. Let $x$ be a limit point of $\R^\infty$ and assume that $x \notin \R^\infty$. We write $x = (x_\alpha)_{\alpha \in J}$. Since this sequence is never eventually zero, 
        for each term not equal to zero we can select an $\epsilon$ nbhd $U_\alpha = (x_\alpha - \epsilon_\alpha, x_\alpha + \epsilon_\alpha)$ that does not contain zero. 
        Then it is clear that this nbhd cannot contain an element of $\R^\infty$, hence $x$ is not a limit point; a contradiction. Thus, $\R^\infty$ must contain all its limit points so, $\R^\infty$ is a closed subset in the box topology. 
    \end{proof}


    \question 

    \begin{proof}
        To show that $h$ is a bijection, Let $(x_1, x_2, \dots) \in \R^\omega$, then let $x = (\frac{x_1 - b_1}{a_1}, \frac{x_2-b_2}{a_2})$. Since $a_i > 0$ each term is well defined. Then it is clear that $h(x) = (x_1, \dots)$, hence 
        $h$ is a surjection. Now suppose that $h(x_1, x_2, \dots) = h(x_1^\prime, x_2^\prime, \dots)$. Then 
        \[(a_1x_1 + b_1, a_2x_2 + b_2, \dots) = (a_1x_1^\prime + b_1, a_2x_2^\prime + b_2)\]
        so $a_ix_i + b_i = a_ix_i^\prime + b_i$ with $a_i \neq 0$, so $x_i = x_i^\prime$. Thus $h$ is a bijection. Now we must show that $h$ is continuous with a continuous inverse. Let $f_i(x) = a_ix + b_i$, then we can write $h$ 
        as $h(x) = ((f_1 \circ \pi_1)(x), (f_2 \circ \pi_2)(x), \dots )$, then each coordinate function is continuous since it is the composistion of two continuous functions. Then since maps into products under the product topology are continuous 
        if and only if the coordinate functions are continuous, we see that $h$ is a continuous function. To see that the inverse is continuous we can do the same thing since each of the coordinate functions of the inverse are of the form 
        $\frac{x_i}{a_i} + \frac{b_i}{a_i}$. 
    \end{proof}
\end{document}