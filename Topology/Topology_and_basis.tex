
\begin{chapter}{Topology and Basis}
    \section{Topology}

    
    \begin{defn}
        Let $X$ be a set. A $\emph{topology}$ on $X$ is a collection of subsets 
        $T$ such that 
        
        \begin{enumerate}
            \item $X$, $\varnothing \in \T$. 
            \item Closed under arbitrary unions.
            \item Closed under finite intersections. 
        \end{enumerate} 
    \end{defn}

    On a given set there may be many diffrerent topologies. That can be defined on that set. 
    Let $X = \{1, 2, 3, 4, 5\}$, we will write down several examples of topologies on $X$. 
    
    \begin{enumerate}
        \item $\T = \{\varnothing, X\}$
        \item $\T = \mathfrak{p}(X)$
        \item $\T = \{\varnothing, X, \{1\}\}$
        \item $\T = \{\varnothing, X, \{1\}, \{2\}, \{1,2\}\}$
    \end{enumerate}

    are all topologies on $X$. The first is called the trivial topology or the indescrete topology and the second is the descrete topology. We 
    will now give an example of another topology, the finite complement topology. 

    
    \begin{defn}{Finite complement Topology}
        Let $X$ be a set and let $\T$ consist of all subsets $U \subset X$ such that the complemnet of $U$ in $X$ is finite 
        or $X$. 
    \end{defn}

    
    \begin{prop}
        The finite complement topology is a topology
    \end{prop}

    
    \begin{proof}
        We must show all three conditions are true, first we show that $X$ and $\varnothing$ are in $\T$. Note, $X-\varnothing = X$ and $X - X = \varnothing$ and both of 
        these satisfy are conditions on $\T$. Now we show $\T$ is closed under arbitrary unions. Let $\{U_\alpha\}_{\alpha \in J} \subset \T$ and we want to show that 
        $(\bigcup_{\alpha} U_\alpha)^c \in \T$. We have by DeMorgans laws 
        \[(\bigcup_{\alpha \in J} U_\alpha)^c = \bigcap_{\alpha \in J} U_\alpha^c \]
        and since each $U_i \in \T$ we now that each $U_i^c$ is finite, thus since the intersection of finite sets are finite we are done. Now we must show closure under finite 
        intersections, so let $\{U_1, \dotsb, U_n \}$ be a subset of $\T$. Then we have 
        \[(\bigcap U_i)^c = \bigcup U_i^c \]
        and since each $U_i^c$ is finite, and we have a finite number of sets to union, the result is finite. Hence we have showed the finite complement topology is indeed a 
        topology. 
    \end{proof}

    We could replace the finite condidtion with countable and we would still have a topology since the union of countable sets is again 
    countable. 

    Given two topologies on a set we can also compare them. 
    
    \begin{defn}
        Let $X$ be a set and let $\T$ and $\T^\prime$ be topologies on $X$. We say $\T$ is finer than $\T^\prime$ if 
        $\T^\prime \subset \T$. We say $\T$ is corser in the reverse situation. 
    \end{defn}

    \section{Basis for a Toplogy}
    
    \begin{defn}{Basis}
        Let $X$ be a set, we say $\B$ is a $\emph{Basis}$ if 
        
        \begin{enumerate}
            \item For all $x \in X$ there exists $B \in \B$ such that $x \in B $.
            \item If $B_1$, $B_2 \in \B$ and $x \in B_1 \cap B_2$ then there exists $B_3 \in \B$ such that $x \in B_3 \subset B_1 \cap B_2$. 
        \end{enumerate}

        We define the topology generated by $\B$ as the collection $\T$ such that for any $U \subset X$, if 
        for all $x \in U$ there exists $B \in \B$ such that $x \in B \subset U$ then $U \in \T$. 
        For any $U \in \T$ we say that $U$ is open. 
    \end{defn}

    
    \begin{prop}
        The topology generated by a basis is a topology. 
    \end{prop}

    
    \begin{proof}
        The first condidtion for a basis gives us $X$ as an open set and the empty set satisfys our condition vacuosly. 
        Now we must prove closure under arbitrary unions. Let $\{U_\alpha\}_{\alpha \in J} \subset \T$ be a collection of 
        open sets and consider their union. Then for $x \in \bigcup U_\alpha$ we must have $x$ apearing in some $U_\alpha$ since it 
        is in the union, but $U_\alpha $ is by assumtion open so there exists $B \in \B$ such that 
        \[x \in B \subset U_\alpha \subset \bigcup_{\alpha \in J} U_\alpha\]
        as desired. Now we must show that the finite intersection of open sets is again open, for that let 
        $\{U_1, \dotsb, U_n\}$ be a collection of open sets. Then if $x$ lies in their intersection it must lie in each 
        $U_i$. Thus there exists a familiy of basis elements $\{B_1, \dotsb, B_n\}$ such that $x \in B_1 \subset U_i$. 
        It follows then that $x \in \bigcap_{i < n} B_i$. Now to see that this intersection must be a basis element, we use induction 
        on the second part of the definition of a basis. 
    \end{proof}

    We now look at another way to define the topology generated by a basis. 

    
    \begin{lem}
        Let $X$ be a topological space and let $\B$ be a basis for the topology on $X$. Then $\T$ is equal 
        to set containg all unions of elements of $\B$
    \end{lem}

    
    \begin{proof}
        Let $U \in \T$, we want to write $U$ as a union of basis elements. By definition we know that for each $x \in U$ there exists $B_x \in \B$ satisfying 
        $x \in B_x \subset U$. Taking the union over all $B_x$ gives us the desired result. Now since Basis elements are open, any union of them must be contained 
        in $\T$, by definition. 
    \end{proof}

    It may be helpful to be able to check whether or not a given set of subsets forms a basis for the topology. 
    
    \begin{lem}
        Let $X$ be a topological space. Let $\C$ be a collection of open sets such that 
        for all $U \in \T$ and all $x \in U$ there exists $C \in \C$ such that 
        \[x \in C \subset U\]
        Then $\C$ is a basis for the topology on $X$
    \end{lem}

    
    \begin{proof}
        The first condition of a basis is satisfied by assumption. Now suppose $x \in C_1 \cap C_2$ we must show there exists 
        $C_3 \in \C$ such that $x \in C_3 \subset C_1 \cap C_2$. We may use the fact that $\C$ is a collection of open sets together with our assumption 
        to produce such an element. 

        Now we must show that $\C$ generates the correct topology. Let $\C$ generate $\T^\prime$. If $U$ is open in $\T$ then by assumption it is open in $\T^\prime$. 
        If $U$ is open in $\T^\prime$ then it is a union of elements of $\C$, since $\C$ is a collection of open sets of $\T$, $U$ must be open in $\T$.
    \end{proof}

    Now we may wish to tell whether one topology is finer than another, we can use the following lemma. 

    
    \begin{lem}
        Let $X$ be a set and let $\T, \T^\prime$ be topologies on $X$ with basis's $\B$ and $\B^\prime $ respectively. Then the following are equivalent. 
        
        \begin{enumerate}
            \item $\T^\prime$ is finer than $\T$
            \item For all $B \in \B$ and $x \in B$ there exists $B^\prime \in \B^\prime$ such that $x \in B^\prime \subset B$.  
        \end{enumerate}
    \end{lem}

    
    \begin{proof}
        $(2) \implies (1)$: Let $U \in \T$, then for every $x \in U$ there exists $B \in \B$ with $x \in B \subset U$. Then by assumption we have there exists 
        $B^\prime \in \B^\prime$ with $x \in B^\prime \subset U$. Hence $U$ is open in $\T^\prime$. 

        $(1) \implies (2)$: Assume that $\T^\prime$ is finer than $\T$. Then since $\T$ is a topology and $\T \subset \T^\prime$, for all $x \in B$ 
        there must exist $B^\prime \in \B^\prime$ such that $x \in B^\prime \subset B$. 
    \end{proof}
    
    To end this section we discuss subbasises. 

    
    \begin{defn}
        Let $X$ be a set. A subbasis is a collection of subsets $A$ such that th union over $A$ is $X$. We define the topology generated 
        by the subbasis as the collection of all unions of all intersections of elements of $A$
    \end{defn}

    Of course we must prove that this is indeed a topology, but first notice that the definition of a subbasis is just the first axiom of a basis. Thus, 
    every basis is a subbais, and if $\B$ is a basis considering it as a subbasis will generate the same topology. So Subasises are a generalization of 
    of a basis. Can you give an example of a subbasis which is not also a basis? 

    
    \begin{prop}
        The topology generated by a subbasis is a topology
    \end{prop}
    
    \begin{proof}
        It is sufficient to show that the collection of all finite intersections of elements of $A$ is a basis. Since the union of all elements 
        of $A$ is $X$, the first condition of a basis is clearly satisfied. Now suppose $x \in a_1 \cap a_2$, where $a_1, a_2$ are finite intersections of elements of $A$.
        Then $a_1 \cap a_2$ is a finite intersection with length equal to the sum of the lengths of $a_1$ and $a_2$. 
    \end{proof}

    
    \section{Order Topology}
    
    \begin{defn}
        Let $X$ be a set, we define the order topology on $X$ as the topology generated by the basis $\B$ such that 
        $\B$ contains all elements of the form 
        
        \begin{enumerate}
            \item $(a, b)$ for $a, b \in X$. 
            \item $[a_0, b)$ where $a_0 $ is the minimal element of $X$. 
            \item $(a, b_0]$ where $b_0$ is the maximal element of $X$.
        \end{enumerate}
        If $X$ has no maximal or minimal elements, then $\B$ consists only of elements of the first type. 
    \end{defn}

    Now we must of course prove that this choice of $\B$ does indeed form the basis of a topology, but it is clear since 
    the intersection of any of these sets yields another set of the same type. There are several cases to check. 
    
    \begin{ex}
        Consider the order topology on $\R$, since $\R$ has no maximal or minimal elements the order topology is generated by 
        $\B = \{(a, b) | a, b \in \R\}$. Then it is clear that this coincides with the standard topology on $\R$
    \end{ex}

    
    \begin{ex}
        Consider the order topology on $\Z_{+}$ given the usual order. Then the order topology is equivalent to the discrete topology. 
    \end{ex}

    
    \begin{ex}
        Consider $\R \times \R$ equipped with the dictionary order, then the basis elements are of the form 
        $(a, b) \times (c, d)$ where $a < c$ or $(a = c) \wedge (b < d)$. 
    \end{ex}

    \section{Product Topology}
        Given two sets $A$ and $B$ we may want to define a topology on the cartesian product $A \times B$. 
    The cartesian product comes equipped with two functions $\pi_A(x): A \times B \to A$ and $\pi_B(x): A \times B \to B$ called 
    the projections of $A \times B$. 

    
    \begin{defn}
        Given topological spaces $A, B$ we define the basis for the topology on $A \times B$ as 
        \[\B = \{U \times V \}\]
        Where $U$ is open in $A$ and $V$ is open in $B$. 
    \end{defn}

    First we must prove that this is indeed a topology on $A \times B$. In fact not only does this define a topology, but it will satisfy certain universal properties that 
    we would like it too, given an arbitrary topological space $X$ with continuous maps to $A$ and to $B$ there is a unique map into $A \times B$ that makes the relevant diagrams commute.

    
    \begin{proof}
        place
    \end{proof}
    The next theorem is obvious and the proof is trivial

    
    \begin{thm}
        If $\B$ is a basis for $A$ and $\C$ is a basis for $B$ then $\B \times \C$ is a basis for the product topology. 
    \end{thm}

    There are plenty of good examples in the book. 


   
   \begin{thm}
    The projections $\pi_A$ and $\pi_B$ are both continuous maps 
   \end{thm}

   subbasis is formed by preimages of the projections. 


   \section{Subspace Topology}
   Given a topological space $X$ and $Y \subset X$ we may often wish to place some topology on $Y$ and consider it as a topological space. There is a very clear way of doing that, 
   and in this case we just define the open sets of $Y$ without bothering to define a basis. 

   
   \begin{defn}
    $U \subset Y$ is open in the topology of $Y$ if there exists an open set $V \subset X$ such that $U = V \cap Y$
   \end{defn}

   That is to say that the open sets of $Y$ are just the open sets of $X$ intersected with $Y$. Once again one must show that this defines a topology on $Y$. 

   
   \begin{thm}
    If $A \subset Y$ is open in $Y$ and $Y$ is open in $X$ then $A$ is open in $X$
   \end{thm}

   
   \begin{proof}
    Since $A$ is open in $Y$ there exists an open set $B$ such that $A = Y \cap B$, then since $Y$ is open in $X$ it follows that $A$ is open in $X$. 
   \end{proof}
   
   commuting theorems 
\end{chapter}
